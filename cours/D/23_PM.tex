% Domaine : Algèbre
% Niveau : Avancé
% Contenu : Exercices
% Auteur : Louise Gassot
% Stage : Montpellier 2016
%
% Inégalités

\documentclass[12pt,A4]{book}

\usepackage{geometry}           % Définir les marges
\usepackage[francais]{babel}

\usepackage[francais]{babel}
\usepackage[T1]{fontenc}
\usepackage[utf8]{inputenc}

\usepackage[tbtags]{amsmath}
\usepackage{amssymb,amsthm}
\usepackage{bbm,yhmath,stmaryrd}
\usepackage{array,multirow}
\usepackage[all]{xy}
\usepackage{url}
\usepackage{color}
\usepackage{import}
\usepackage{pdfpages}
\usepackage{fancyhdr}
\usepackage{titlesec,verbatim}
\usepackage{epsfig}
\usepackage{enumerate}
\usepackage{palatino}
\usepackage{pgf,tikz}
\usetikzlibrary{shapes,backgrounds,patterns,arrows}


\textwidth 173mm \textheight 235mm \topmargin -50pt \oddsidemargin
-0.25cm \evensidemargin -0.25cm

%%%%%
\def \N {\mathbb{N}}
\def \Z {\mathbb{Z}}
\def\Z{\mathbb Z}
\def\Q{\mathbb Q}
\def\R{\mathbb R}
\def\C{\mathbb C}
\def \K {\mathbb{K}}
\def \S {\mathfrak{K}}
\def\calC{\mathcal C}
\def \cyc {\textrm{cyc}}
\def \rec {\textrm{rec}}
\def \inv {\textrm{inv}}
\def \Card {\textrm{Card}}
\def \style {\displaystyle}

%%%%%
\newcommand{\A}{\mathcal{A}}
\newcommand{\fii}{\varphi}
%%%%

\makeatletter
\newcounter{exo}[subsection]
\newenvironment{exo}[1][]
  {\edef\@exer{#1}%
   \ifx\@exer\empty{}\else{\setcounter{exo}{#1}\addtocounter{exo}{-1}}\fi%
   \refstepcounter{exo}%
   \noindent {\bf Exercice \theexo}%
   \hspace{0.5em}}
  {\medskip}

\newcounter{sol}[subsection]
\newenvironment{sol}[1][]
  {\edef\@exer{#1}%
   \ifx\@exer\empty{}\else{\setcounter{sol}{#1}\addtocounter{sol}{-1}}\fi%
   \refstepcounter{sol}%
   \noindent {\it \underline{Solution de l'exercice \thesol}%
   \hspace{0.5em}}}
  {\medskip}
\makeatother

\theoremstyle{definition}

\newtheorem{theo}{Théorème}[section]
\newenvironment{preuve}{
  \noindent \textbf{Démonstration. }}{\hfill \(\square\) }

\newtheorem{defn}{Définition}
\newtheorem{rem}[defn]{Remarque}
\newtheorem{rems}[defn]{Remarques}
\newtheorem{ex}[defn]{Exemple}
\newtheorem{prop}[defn]{Proposition}
\newtheorem{dem}[defn]{Démonstration}
\newtheorem{corr}[defn]{Corollaire}
\newtheorem{thm}[defn]{Théoréme}
\newtheorem{lem}[defn]{Lemme}
\newtheorem{nott}[defn]{Notation}

%%%%%%


\begin{document}

\subsubsection{Rappel : liste d'inégalités à savoir (non exhaustive)}

\begin{itemize}
\item Un carré est toujours positif.
\item Inégalité arithmético-géométrique, et plus généralement les inégalités entre les moyennes. Voici les principales (\textbf{QM-AM-GM-HM}) :\\
Si $a_1,\dots,a_n$ sont des nombres positifs [strictement si on fait un inverse], alors
\[\sqrt{\frac{a_1^2+\dots+a_n^2}{n}}\geq\frac{a_1+\dots+a_n}{n}\geq\sqrt[n]{a_1\dots a_n}\geq\frac{n}{\frac{1}{a_1}+\dots+\frac{1}{a_n}},\]
avec égalité si et seulement si $a_1=\dots=a_n$.
On peut visualiser simplement comment utiliser l'IAG lorsqu'il faut prouver des inégalités homogènes en utilisant la méthode du triangle. Par exemple, pour une inégalité à trois variables $a,b,c$ de degré 6, on repère les termes de la façon suivante :
\begin{center}
\shorthandoff{:;!?}
\begin{tikzpicture}[scale=0.8]
\draw (3,5.2)-- (6,0)-- (0,0)-- cycle;
\draw (3.5,4.33)-- (1,0)-- (0.5,0.87)-- (5.5,0.87)-- (5,0)-- (2.5,4.33)-- cycle;
\draw (4,3.46)-- (2,0)-- (1,1.73)-- (5,1.73)-- (4,0)-- (2,3.46)-- cycle;
\draw (4.5,2.6)-- (3,0)-- (1.5,2.6)-- cycle;
\draw (0.1,-0.3) node {$b^6$};
\draw (5.9,-0.3) node {$c^6$};
\draw (3.1,5.6) node {$a^6$};
\begin{scriptsize}
\draw (3.76,4.54) node {$a^5c$};
\draw (1,-0.2) node {$b^5c$};
\draw (4.32,3.66) node {$a^4c^2$};
\draw (2,-0.2) node {$b^4c^2$};
\draw (4.82,2.8) node {$a^3c^3$};
\draw (3,-0.2) node {$b^3c^3$};
\draw (5.32,1.94) node {$a^2c^4$};
\draw (4,-0.2) node {$b^2c^4$};
\draw (5.76,1.06) node {$ac^5$};
\draw (5,-0.2) node {$bc^5$};
\draw (2.16,4.54) node {$a^5b$};
\draw (1.72,3.66) node {$a^4b^2$};
\draw (1.12,2.8) node {$a^3b^3$};
\draw (0.72,1.94) node {$a^2b^4$};
\draw (0.16,1.06) node {$ab^5$};
\draw (3.32,3.66) node {$a^4bc$};
\draw (2.88,2.8) node {$a^3b^2c$};
\draw (3.88,2.8) node {$a^3bc^2$};
\draw (4.38,1.94) node {$a^2bc^3$};
\draw (3.44,1.94) node {$a^2b^2c^2$};
\draw (2.38,1.94) node {$a^2b^3c$};
\draw (1.82,1.06) node {$ab^4c$};
\draw (2.88,1.06) node {$ab^3c^2$};
\draw (3.88,1.06) node {$ab^2c^3$};
\draw (4.82,1.06) node {$abc^4$};
\end{scriptsize}
\end{tikzpicture}
\end{center}
De façon plus rigoureuse, le terme $a^\alpha b^\beta c^\gamma$ est plac\'e au point de coordonn\' ees barycentriques $(\alpha, \beta, \gamma)$ par rapport aux sommets du triangle. Cette méthode permet de visualiser plus facilement quels sont les coefficients à utiliser pour prouver une inégalité avec l'IAG (cf exemples dans les exercices du TD).

\item \textbf{Schur}\\
Si $x,y,z$ sont des r\'eels positifs et  $r$ un nombre r\'eel, alors
\[ x^r(x-y)(x-z) + y^r(y-x)(y-z) + z^r(z-x)(z-y) \geqslant 0, \]
avec égalité lorsque $x = y = z$, ou bien lorsque deux r\'eels parmi $x,y,z$ sont \'egaux et le troisi\`eme est nul.

Rappel de la méthode du triangle pour la reconnaître (pour $r=1$ et $r=4$) :

\begin{center}
\shorthandoff{:;!?}
\begin{tikzpicture}[scale=0.8]
\draw (3,5.2)-- (6,0)-- (0,0)-- cycle;
\draw (4,3.46)-- (2,0)-- (1,1.73)-- (5,1.73)-- (4,0)-- (2,3.46)-- cycle;
\fill (0,0) circle (7pt);
\fill (6,0) circle (7pt);
\fill (3,5.2) circle (7pt);
\fill (3,1.73) circle (7pt);
\draw (3, 2.5) node {\Large{\textbf{3}}\small{$xyz$}};
\draw [fill=white] (2,0) circle (7pt);
\draw [fill=white] (4,0) circle (7pt);
\draw [fill=white] (1,1.73) circle (7pt);
\draw [fill=white] (5,1.73) circle (7pt);
\draw [fill=white] (2,3.46) circle (7pt);
\draw [fill=white] (4,3.46) circle (7pt);
\draw (0.1,-0.5) node {$y^3$};
\draw (5.9,-0.5) node {$z^3$};
\draw (3.1,5.8) node {$x^3$};

\begin{scope}[shift={(7,0)}]
\draw (3,5.2)-- (6,0)-- (0,0)-- cycle;
\draw (3.5,4.33)-- (1,0)-- (0.5,0.87)-- (5.5,0.87)-- (5,0)-- (2.5,4.33)-- cycle;
\draw (4,3.46)-- (2,0)-- (1,1.73)-- (5,1.73)-- (4,0)-- (2,3.46)-- cycle;
\draw (4.5,2.6)-- (3,0)-- (1.5,2.6)-- cycle;
\fill (0,0) circle (4pt);
\fill (6,0) circle (4pt);
\fill (3,5.2) circle (4pt);
\fill (1.5,0.87) circle (4pt);
\fill (4.5,0.87) circle (4pt);
\fill (3,3.46) circle (4pt);
\draw [fill=white] (1,0) circle (4pt);
\draw [fill=white] (5,0) circle (4pt);
\draw [fill=white] (0.5,0.87) circle (4pt);
\draw [fill=white] (5.5,0.87) circle (4pt);
\draw [fill=white] (2.5,4.33) circle (4pt);
\draw [fill=white] (3.5,4.33) circle (4pt);
\draw (0.1,-0.3) node {$y^6$};
\draw (5.9,-0.3) node {$z^6$};
\draw (3.1,5.6) node {$x^6$};
\end{scope}
\end{tikzpicture}
\shorthandon{:;!?}
\end{center}
En noir, on dessine les termes dominants, et en blanc les termes obtenus après minoration en appliquant l'inégalité.

\item \textbf{Cauchy-Schwarz}\\
Si $x_1, \dots, x_n, y_1, \dots, y_n$ sont des nombres r\'eels, alors
\[ \left|x_1y_1 + \dots + x_ny_n\right| \leqslant \sqrt{x_1^2+ \dots + x_n^2} \sqrt{y_1^2+ \dots y_n^2}, \]
avec \'egalit\'e si et seulement si $(x_1, \dots, x_n)$ est proportionnel \`a $(y_1, \dots, y_n)$.

\item \textbf{Réordonnement}\\
Si $a_1\leq\dots\leq a_n$ et $b_1\leq\dots\leq b_n$ sont des réels, alors pour toute permutation $\sigma$ de $\{1,2,\dots,n\},$
$$a_1b_1+\dots+a_nb_n\geq a_1b_{\sigma(1)}+\dots+a_nb_{\sigma(n)}\geq a_1b_n+a_2b_{n-1}+\dots+a_nb_1,$$
avec égalité si et seulement si la suite $(\sigma(1),\dots,\sigma(n))$ est croissante pour la première inégalité, décroissante pour la deuxième.

\item \textbf{Tchebychev}\\
Si $a_1\leq\dots\leq a_n$ et $b_1\leq\dots\leq b_n$ sont des réels, alors 
\[\frac{a_1b_1+\dots+a_nb_n}{n}\geq \frac{a_1+\dots+a_n}{n}\frac{b_1+\dots+b_n}{n}\geq \frac{a_1b_n+a_2b_{n-1}+\dots+a_nb_1}{n}\]

\item \textbf{Jensen}\\
Soit $I$ un intervalle. Un fonction $f:I\longrightarrow \mathbb{R}$ est convexe si et seulement si pour tout $n\geqslant 1$ et tous $x_1,\dots,x_n\in I$ et $\lambda_1,\dots,\lambda_n\in [0,1]$ tels que $\lambda_1+\dots+\lambda_n=1$, l'inégalité suivante est vérifiée:
\[ f\left( \sum_{i=1}^n\lambda_ix_i \right)\leqslant \sum_{i=1}^n\lambda_if(x_i). \]

\item \textbf{Hölder}\\
Si $a_1,\dots,a_n,b_1\dots,b_n,\dots,z_1,\dots,z_n$ sont des réels positifs ou nuls et $\lambda_a,\dots,\lambda_z$ sont des réels strictement positifs tels que $\lambda_a+\dots+\lambda_z=1,$ alors
$$(a_1+\dots+a_n)^{\lambda_a}(b_1+\dots+b_n)^{\lambda_b}\dots(z_1+\dots+z_n)^{\lambda_z}\geq a_1^{\lambda_a}b_1^{\lambda_b}\dots z_1^{\lambda_z}+\dots+a_n^{\lambda_a}b_n^{\lambda_b}\dots z_n^{\lambda_z}.$$
On l'utilise généralement avec deux suites :\\
Si $a_1,\dots,a_n,b_1\dots,b_n$ sont des réels positifs ou nuls, $p,q$ des réels supérieurs ou égaux à 1 tels que $\frac{1}{p}+\frac{1}{q}=1$, alors
\[ (a_1^p+\dots+ a_n^p)^{\frac{1}{p}}(b_1^q+\dots+ b_n^q)^{\frac{1}{q}}\geq a_1b_1+\dots+ a_nb_n. \]

On peut en déduire l'inégalité de \textbf{Minkowski} :\\
Si $p$, $x_1,\dots, x_n$ et $y_1,\dots, y_n$ sont des r\'eels strictement positifs tels que $p\geqslant1$, alors 
\[ \left(\sum_{i=1}^n(x_i+ y_i)^p\right)^{1/p} \leqslant \left(\sum_{i=1}^nx_i^p\right)^{1/p} + \left(\sum_{i=1}^ny_i^p\right)^{1/p}. \]


\item Changement de variables. Notamment, lorsque les variables sont les longueurs des côtés d'un triangle, penser à la \textbf{transformation de Ravi} :\\
Si $a,b,c$ sont les longueurs des côtés d'un triangle, alors il existe des réels positifs $u,v,w$ tels que $a=v+w$, $b=u+w$ et $c=u+v$.

\item Homogénéisation/Déshomogénéisation
\item etc.
\end{itemize}


\subsubsection{Énoncés}


\begin{exo}%01%Test Grésillon 2010/IAG
Montrer que pour tous $a,b,c\geq0$ tels que $abc=1$, $(a^2+1)(b^3+2)(c^6+5)\geq36.$
\end{exo}

\begin{exo}%02%Hong-Kong/IAG,savoir calculer
Trouver la plus grande constante C telle que pour tous $x,y,z\geq0$, $(yz+zx+xy)^2(x+y+z)\geq Cxyz(x^2+y^2+z^2).$
\end{exo}

\begin{exo}%03%Mildorf 18/IAG
Soit $a,b,c,d>0$. Montrer que $a^4b+b^4c+c^4d+d^4a\geq abcd(a+b+c+d)$.
\end{exo}

\begin{exo}%04%Chine 2008/Schur, IAG, cgt var
Soit $x,y,z$ des réels strictement positifs. Montrer :
$$\frac{xy}{z}+\frac{yz}{x}+\frac{zx}{y}>2\sqrt[3]{x^3+y^3+z^3}.$$
\end{exo}

\begin{exo}%05%IMO 1961/Cauchy-Schwarz
On suppose que $a,b,c$ sont les longueurs des côtés d'un triangle, $T$ son aire. Montrer que $a^2+b^2+c^2\geq 4\sqrt{3}T$. Quels sont les cas d'égalité ?
\end{exo}

\begin{exo}%06%Envoi 4 2012-2013/Cauchy-Schwarz
Soit $x,y,z$ des réels strictement positifs. On suppose que $x+y+z\geq \frac{1}{x}+\frac{1}{y}+\frac{1}{z}.$ Montrer :
$$\frac{x}{y}+\frac{y}{z}+\frac{z}{x}\geq \frac{1}{xy}+\frac{1}{yz}+\frac{1}{zx}.$$
\end{exo}


\begin{exo}%07%Grésillon 2012/Ravi
On suppose que $a,b,c$ sont les longueurs des côtés d'un triangle.
\begin{enumerate}
\item Montrer qu'il existe un triangle de côtés $\sqrt{a},\sqrt{b},\sqrt{c}.$
\item Montrer que $\sqrt{ab}+\sqrt{bc}+\sqrt{ca}\leq a+b+c\leq 2\sqrt{ab}+2\sqrt{bc}+2\sqrt{ca}.$
\end{enumerate}
\end{exo}

\begin{exo}%08%Envoi 1 2005-2006/Réordonnement
Soit $a,b,c>0$ tels que $ab+bc+ca=1$. Montrer :
$$\frac{1}{a+b}+\frac{1}{b+c}+\frac{1}{c+a}\geq\sqrt{3}+\frac{ab}{a+b}+\frac{bc}{b+c}+\frac{ca}{c+a}.$$
\end{exo}

\begin{exo}%09%Russie/ x+1/x>=2
Soit $x_0>\dots>x_n$ des réels. Montrer que $x_0+\frac{1}{x_0-x_1}+\frac{1}{x_1-x_2}+\dots+\frac{1}{x_{n-1}-x_n}\geq x_n+2n.$
\end{exo}

\begin{exo}%11%Poly 2011/Jensen
Montrer que :
\[\forall x,y,z\in [0;1],~~\frac{x}{y+z+1}+\frac{y}{z+x+1}+\frac{z}{x+y+1}\leq 1-(1-x)(1-y)(1-z).\]
\end{exo}

\begin{exo}%12%Poly 2011/Jensen
Soit $a,b,c$ des réels strictement positifs vérifiant l'identité $a+b+c=1$. Montrer que :
\[a\sqrt{b}+b\sqrt{c}+c\sqrt{a}\leq\frac{1}{\sqrt{3}}.\]
\end{exo}

\begin{exo}%13%Chine 2015/HM-GM,Chebychev
Soit $n\geq 1$ un entier, $x_1,\dots,x_n$ des réels strictement positifs tels que $x_1\leq x_2\leq\dots\leq x_n$ et $x_1\geq\frac{x_2}{2}\geq\dots\geq\frac{x_n}{n}$. Montrer :
$$\frac{\sum_{i=1}^nx_i}{n\left(\prod_{i=1}^nx_i\right)^{\frac{1}{n}}}\leq\frac{n+1}{2\sqrt[n]{n!}}.$$
\end{exo}

\begin{exo}%14%Chine 2011 simplifié/Minkowski, IAG, Hölder
Soit $n\geq1$ un entier, $x_1,\dots,x_{2n}$ des réels positifs. On pose
$$S=\left(\frac{1}{2n}\sum_{i=1}^{2n}(x_i+2)^n\right)^{\frac{1}{n}},~~~
S'=\left(\frac{1}{2n}\sum_{i=1}^{2n}(x_i+1)^n\right)^{\frac{1}{n}},~~~
P=\left(\prod_{i=1}^{2n}x_i\right)^{\frac{1}{n}}.$$
Montrer que si $S\geq P$, alors $S'\geq\frac{3}{4}P$.
\end{exo}

\begin{exo}%15%Polynômes !! et x^2>=0
Soit $n>1$ un entier. Si $x_1,\dots,x_n$ sont les racines du polynôme $X^n+a_1x^{n-1}+\dots+a_{n-1}$ et $y_1,\dots,y_{n-1}$ les racines du polynôme dérivé $nx^{n-1}+(n-1)a_1X^{n-2}+\dots+2a_{n-2}X+a_{n-1}$, montrer :
$$\frac{1}{n}\sum_{k=1}^nx_k^2>\frac{1}{n-1}\sum_{k=1}^ny_k^2.$$
\end{exo}



\subsubsection{Corrigés}

\begin{sol}%01
On utilise l'inégalité arithmético-géométrique de la façon suivante :
$$a^2+1\geq 2a, ~~~ b^3+2=b^3+1+1\geq 3b, ~~~ c^6+5=c^6+1+1+1+1+1\geq 6c,$$
et en multipliant les trois inégalités on obtient le résultat.
\end{sol}

\begin{sol}%02
En posant $y=z=1$, $\frac{(yz+zx+xy)^2(x+y+z)}{xyz(x^2+y^2+z^2)}=\frac{(2x+1)^2(x+2)}{x(x^2+2)}.$ Cette quantité tend vers $4$ lorsque $x$ tend vers $+\infty$, donc on va essayer de montrer l'inégalité avec $C=4.$

En développant, il faut donc montrer que
\[(x^3y^2+x^2y^3+y^3z^2+y^2z^3+z^3x^2+z^2x^3)+5(x^2y^2z+y^2z^2x+z^2x^2y)\geq2(x^3yz+y^3zx+z^3xy).\]
On dessine alors simplement le triangle :

\begin{center}
\begin{tikzpicture}
\begin{scope}[shift={(12.8,0)}]
\draw (2.5,4.33)-- (5,0)-- (0,0)-- cycle;
\draw (2.5,4.33) -- (3,3.46)-- (1,0)-- (0.5,0.87)-- (4.5,0.87);
\draw (3,3.46) -- (3.5,2.6) -- (2,0)-- (1,1.73)-- (4,1.73) -- (4.5,0.87)-- (4,0)-- (2,3.46)-- cycle;
\draw (3.5,2.6) -- (4,1.73) -- (3,0)-- (1.5,2.6)-- cycle;
\fill (3.5,2.6) circle (4pt);
\fill (1,1.73) circle (4pt);
\fill (3,0) circle (4pt);
\fill (4,1.73) circle (4pt);
\fill (1.5,2.6) circle (4pt);
\fill (2,0) circle (4pt);


\fill (2,1.73) circle (4pt);
\fill (3,1.73) circle (4pt);
\fill (2.5,0.87) circle (4pt);

\draw [fill=white] (1.5,0.87) circle (4pt);
\draw [fill=white] (3.5,0.87) circle (4pt);
\draw [fill=white] (2.5,2.6) circle (4pt);

\draw (-0.3,-0.3) node {$x^5$};
\draw (5.4,-0.3) node {$y^5$};
\draw (2.6,4.8) node {$z^5$};
%\begin{scriptsize}
\draw (4.5,2.7) node {$\mathbf{z^3y^2}$};
\draw (0,1.73) node {$\mathbf{x^3z^2}$};
\draw (3.2,-0.8) node {$\mathbf{x^2y^3}$};
\draw (5,1.83) node {$\mathbf{z^2y^3}$};
\draw (0.5,2.7) node {$\mathbf{x^2z^3}$};
\draw (1.8,-0.8) node {$\mathbf{x^3y^2}$};


\draw (1.5,1.73) node {$\mathbf{5}$};
\draw (3.5,1.73) node {$\mathbf{5}$};
\draw (2.5,0.37) node {$\mathbf{5}$};

\draw (1,0.87) node {$2$};
\draw (4,0.87) node {$2$};
\draw (2.5,3.1) node {$2$};

%\end{scriptsize}
\end{scope}
\end{tikzpicture}
\end{center}

Et on voit alors qu'il suffit d'appliquer trois fois l'IAG avec les points noirs extrémaux ($x^2z^3+z^3y^2\geq2z^3xy$, etc.) pour conclure.
\end{sol}

\begin{sol}%03
On a envie d'utiliser l'IAG, mais on ne peut pas dessiner de triangle parce qu'il y a $4$ variables.

On cherche donc des coefficients $x,y,z,t$ pour $a,b,c,d$ et on écrit :
\[\frac{xa^4b+yb^4c+zc^4d+td^4a}{x+y+z+t}\geq \sqrt[x+y+z+t]{a^{4x+t}b^{4y+x}c^{4z+y}d^{4t+z}}\]
On veut que le terme de droite soit, dans un premier temps, égal à $a^2bcd$.\\
On veut résoudre le système :
$\begin{cases}
x+y+z+t=1\\
4x+t=2\\
4y+x=1\\
4z+y=1
\end{cases},$
qui conduit à
$\begin{cases}
x=\frac{23}{51}\\
y=\frac{7}{51}\\
z=\frac{11}{51}\\
t=\frac{10}{51}
\end{cases}.$\\
On peut vérifier qu'on a bien :
\[\frac{23a^4b+7b^4c+11c^4d+10d^4a}{51}\geq \sqrt[51]{a^{102}b^{51}c^{51}d^{51}}=a^2bcd.\]
Si on fait de même avec les trois autres termes et que l'on somme le tout, on a l'inégalité voulue.
\end{sol}

\begin{sol}%04
On commence par faire le changement de variables :$a^2=\frac{xy}{z},b^2=\frac{yz}{x},c^2=\frac{zx}{y}.$ L'inégalité devient :\\
$a^2+b^2+c^2>2\sqrt[3]{a^3b^3+b^3c^3+c^3a^3}.$\\
On élève au cube et on développe :
$$\sum_{cyc}a^6+3\sum_{sym}a^4b^2+6a^2b^2c^2>8\sum_{cyc}a^3b^3.$$
On dessine un triangle :

\begin{center}
\begin{tikzpicture}
\begin{scope}[shift={(12.8,0)}]
\draw (3,5.2)-- (6,0)-- (0,0)-- cycle;
\draw (3.5,4.33)-- (1,0)-- (0.5,0.87)-- (5.5,0.87)-- (5,0)-- (2.5,4.33)-- cycle;
\draw (4,3.46)-- (2,0)-- (1,1.73)-- (5,1.73)-- (4,0)-- (2,3.46)-- cycle;
\draw (4.5,2.6)-- (3,0)-- (1.5,2.6)-- cycle;

\draw [fill] (0,0) circle (4pt);
\draw [fill] (6,0) circle (4pt);
\draw [fill] (3,5.2) circle (4pt);
\draw [fill=white] (4.5,2.6) circle (4pt);
\draw [fill=white] (3,0) circle (4pt);
\draw [fill=white] (1.5,2.6) circle (4pt);

\fill (4,3.46) circle (4pt);
\fill (1,1.73) circle (4pt);
\fill (4,0) circle (4pt);

\fill (5,1.73) circle (4pt);
\fill (2,3.46) circle (4pt);
\fill (2,0) circle (4pt);

\draw [fill] (3,1.73) circle (4pt);

\draw (0.1,-0.5) node {$\mathbf{b^6}$};
\draw (5.9,-0.5) node {$\mathbf{c^6}$};
\draw (3.1,5.6) node {$\mathbf{a^6}$};
%\begin{scriptsize}

\draw (0.27,1.99) node {$\mathbf{3a^2b^4}$};
\draw (4,-0.5) node {$\mathbf{3b^2c^4}$};
\draw (4.68,3.76) node {$\mathbf{3a^4c^2}$};
\draw (5.18,2.8) node {$8a^3c^3$};
\draw (3.08,-0.5) node {$8b^3c^3$};
\draw (0.68,2.84) node {$8a^3b^3$};
\draw (5.8,2) node {$\mathbf{3a^2c^4}$};
\draw (1.3,3.66) node {$\mathbf{3a^4b^2}$};
\draw (1.8,-0.5) node {$\mathbf{3b^4c^2}$};

\draw  (3.2,2.13) node {$\mathbf{6a^2b^2c^2}$};

%\end{scriptsize}
\end{scope}
\end{tikzpicture}
\end{center}

On ne peut pas appliquer simplement directement l'IAG, on s'en sort grâce à l'inégalité de Schur appliquée à $a^2,b^2$ et $c^2$ avec $r=1$ pour enlever tous les termes en $a^6,b^6,c^6$ :

\begin{center}
\begin{tikzpicture}
\begin{scope}[shift={(12.8,0)}]
\draw (3,5.2)-- (6,0)-- (0,0)-- cycle;
\draw (3.5,4.33)-- (1,0)-- (0.5,0.87)-- (5.5,0.87)-- (5,0)-- (2.5,4.33)-- cycle;
\draw (4,3.46)-- (2,0)-- (1,1.73)-- (5,1.73)-- (4,0)-- (2,3.46)-- cycle;
\draw (4.5,2.6)-- (3,0)-- (1.5,2.6)-- cycle;

\draw [fill] (0,0) circle (4pt);
\draw [fill] (6,0) circle (4pt);
\draw [fill] (3,5.2) circle (4pt);
%\draw [fill=white] (4.5,2.6) circle (4pt);
%\draw [fill=white] (3,0) circle (4pt);
%\draw [fill=white] (1.5,2.6) circle (4pt);

\draw [fill=white] (4,3.46) circle (4pt);
\draw [fill=white] (1,1.73) circle (4pt);
\draw [fill=white] (4,0) circle (4pt);

\draw [fill=white] (5,1.73) circle (4pt);
\draw [fill=white] (2,3.46) circle (4pt);
\draw [fill=white] (2,0) circle (4pt);

\draw [fill] (3,1.73) circle (4pt);

\draw (0.1,-0.5) node {$\mathbf{b^6}$};
\draw (5.9,-0.5) node {$\mathbf{c^6}$};
\draw (3.1,5.6) node {$\mathbf{a^6}$};
%\begin{scriptsize}

\draw (0.27,1.99) node {${a^2b^4}$};
\draw (4,-0.5) node {${b^2c^4}$};
\draw (4.68,3.76) node {${a^4c^2}$};
%\draw (5.18,2.8) node {$8a^3c^3$};
%\draw (3.08,-0.5) node {$8b^3c^3$};
%\draw (0.68,2.84) node {$8a^3b^3$};
\draw (5.8,2) node {${a^2c^4}$};
\draw (1.3,3.66) node {${a^4b^2}$};
\draw (1.8,-0.5) node {${b^4c^2}$};

\draw  (3.2,2.13) node {$\mathbf{3a^2b^2c^2}$};

%\end{scriptsize}
\end{scope}
\end{tikzpicture}
\end{center}

L'inégalité que l'on veut montrer maintenant se redessine :

\begin{center}
\begin{tikzpicture}
\begin{scope}[shift={(12.8,0)}]
\draw (3,5.2)-- (6,0)-- (0,0)-- cycle;
\draw (3.5,4.33)-- (1,0)-- (0.5,0.87)-- (5.5,0.87)-- (5,0)-- (2.5,4.33)-- cycle;
\draw (4,3.46)-- (2,0)-- (1,1.73)-- (5,1.73)-- (4,0)-- (2,3.46)-- cycle;
\draw (4.5,2.6)-- (3,0)-- (1.5,2.6)-- cycle;

\draw [fill=white] (4.5,2.6) circle (4pt);
\draw [fill=white] (3,0) circle (4pt);
\draw [fill=white] (1.5,2.6) circle (4pt);

\draw [fill] (4,3.46) circle (4pt);
\draw [fill] (1,1.73) circle (4pt);
\draw [fill] (4,0) circle (4pt);

\draw [fill] (5,1.73) circle (4pt);
\draw [fill] (2,3.46) circle (4pt);
\draw [fill] (2,0) circle (4pt);

\draw [fill] (3,1.73) circle (4pt);

\draw (0.27,1.99) node {$\mathbf{4a^2b^4}$};
\draw (4,-0.5) node {$\mathbf{4b^2c^4}$};
\draw (4.68,3.76) node {$\mathbf{a^4c^2}$};
\draw (5.18,2.8) node {$8a^3c^3$};
\draw (3.08,-0.5) node {$8b^3c^3$};
\draw (0.68,2.84) node {$8a^3b^3$};
\draw (5.8,2) node {$\mathbf{4a^2c^4}$};
\draw (1.3,3.66) node {$\mathbf{4a^4b^2}$};
\draw (1.8,-0.5) node {$\mathbf{4b^4c^2}$};

\draw  (3.2,2.13) node {$\mathbf{3a^2b^2c^2}$};


\draw (0.1,-0.5) node {${b^6}$};
\draw (5.9,-0.5) node {${c^6}$};
\draw (3.1,5.6) node {${a^6}$};
\end{scope}
\end{tikzpicture}
\end{center}
Il n'y a plus qu'à appliquer l'IAG ($4a^4b^2+4a^2b^4\geq 8a^3b^3$, etc.) et on a fini !
\end{sol}

\begin{sol}%05
À périmètre fixé, l'aire d'un triangle est maximale lorsque le triangle est équilatéral. On a donc :
\[4\sqrt{3}T\leq 4\sqrt{3} \frac{\sqrt{3}}{4}\left(\frac{a+b+c}{3}\right)^2=\frac{1}{3}(a+b+c)^2.\]
On applique maintenant l'inégalité deCauchy-Schwarz :
\[\frac{1}{3}(a+b+c)^2\leq \frac{1}{3}(a^2+b^2+c^2)(1+1+1)=a^2+b^2+c^2.\]
Dans les deux cas, il y a égalité si le triangle est équilatéral.
\end{sol}

\begin{sol}%06
En mettant au même dénominateur, l'inégalité revient à : $x^2z+y^2x+z^2y\geq x+y+z.$
Or d'après l'inégalité de Cauchy-Schwarz,
\[(x+y+z)^2=(x\sqrt{z}\frac{1}{\sqrt{z}}+y\sqrt{x}\frac{1}{\sqrt{x}}+z\sqrt{y}\frac{1}{\sqrt{y}})^2\leq (x^2z+y^2x+z^2y)\left(\frac{1}{z}+\frac{1}{x}+\frac{1}{y}\right).\]
En utilisant la condition de l'énoncé $x+y+z\geq \frac{1}{x}+\frac{1}{y}+\frac{1}{z},$ on obtient l'inégalité voulue.
\end{sol}

\begin{sol}%07
\begin{enumerate}
\item On utilise la transformation de Ravi $a=v+w, b=u+w, c=u+v$ où $u,v,w$ sont des réels positifs. L'inégalité triangulaire à vérifier $\sqrt{a}\leq \sqrt{b}+\sqrt{c}$ se réécrit :
$$\sqrt{v+w}\leq \sqrt{u+w}+\sqrt{u+v},$$
ce qui est clair en élevant les deux membres au carré.
\item L'inégalité de gauche se déduit facilement de l'inégalité arithmético-géométrique $\frac{a+b}{2}\geq\sqrt{ab}.$\\
Pour l'inégalité de droite, comme d'après la question précédente on a déjà l'inégalité triangulaire $\sqrt{a}\leq\sqrt{b}+\sqrt{c},$ on en déduit que $a\leq \sqrt{ab}+\sqrt{ac}$. On conclut en sommant les trois inégalités pour $a,b$ et $c$.
\end{enumerate}
\end{sol}

\begin{sol}%08
On commence par simplifier l'énoncé avec :
$$\frac{1}{a+b}-\frac{ab}{a+b}=\frac{1-ab}{a+b}=\frac{bc+ca}{a+b}=c.$$
Il faut donc montrer que $a+b+c\geq\sqrt{3}.$
Or d'après l'inégalité du réordonnement,
$$(a+b+c)^2=a^2+b^2+c^2+2\geq ab+bc+ca+2=3.$$
\end{sol}

\begin{sol}%09
Il suffit d'écrire :
\[x_0-x_n+\sum_{k=1}^n\frac{1}{x_{k-1}-x_k}=\sum_{k=1}^n x_{k-1}-x_k+\sum_{k=1}^n\frac{1}{x_{k-1}-x_k}.\]
Comme $x+\frac{1}{x}\geq 2$ pour tout $x>0$,
\[x_0-x_n+\sum_{k=1}^n\frac{1}{x_{k-1}-x_k}\geq 2n.\]
\end{sol}

\begin{sol}%11%Poly 2011/Jensen
La fonction qui à $x,y,z$ associe $\frac{x}{y+z+1}+\frac{y}{z+x+1}+\frac{z}{x+y+1}+(1-x)(1-y)(1-z)$ est clairement convexe (toute fonction affine ainsi que la fonction inverse sur les positifs étant convexes). En particulier, elle atteint son maximum sur ces bords, i.e. sur $\lbrace 0;1\rbrace^3$. Quitte à permuter les variables, les seuls cas à vérifier sont donc $x=y=z=1$, $x=y=1,z=0$, $x=1,y=z=0$, $x=y=z=0$. Or, on voit facilement qu'on a égalité dans chacun de ces cas, ce qui conclut cette preuve.
\end{sol}


\begin{sol}%12%Poly 2011/Jensen
La fonction racine est concave. En effet, sa dérivée seconde est l'application $x\longrightarrow -\frac{1}{4\sqrt{x^3}}$, à valeurs strictement négatives. Ainsi, l'inégalité de Jensen (pondérée avec les poids $a,b,c$) montre que :
\[\sum_{cyc}a\sqrt{b}\leq\sqrt{\sum_{cyc}ab}.\]
De plus, on sait que $\frac{1}{2}((a-b)^2+(b-c)^2+(c-a)^2)\geq 0$, ce qui se réécrit comme $\sum_{cyc}ab\leq \sum_{cyc}a^2$ puis comme $3\sum_{cyc}ab\leq \sum_{cyc}a^2+2\sum_{cyc}ab$ puis comme $3\sum_{cyc}ab\leq(a+b+c)^2=1$, d'où $\sum_{cyc}ab\leq\frac{1}{3}$.
Ainsi, on obtient que :
\[\sum_{cyc}a\sqrt{b}\leq\sqrt{\sum_{cyc}ab}\leq\frac{1}{\sqrt{3}}.\]
\end{sol}


\begin{sol}%13
On réécrit l'inégalité de façon un peu plus parlante :
\[\frac{2\sum_{i=1}^n x_i}{n(n+1)}\leq \frac{\left(\prod_{i=1}^nx_i\right)^{\frac{1}{n}}}{\sqrt[n]{n!}}=\left(\prod_{i=1}^n\frac{x_i}{i}\right)^{\frac{1}{n}}.\]
D'après l'inégalité GM-HM,
\[\left(\prod_{i=1}^n\frac{x_i}{i}\right)^{\frac{1}{n}}\geq\frac{n}{\sum_{i=1}^n\frac{i}{x_i}}.\]
On aimerait donc bien montrer :
\[\frac{n}{\sum_{i=1}^n\frac{i}{x_i}}\geq \frac{2\sum_{i=1}^nx_i}{n(n+1)},\]
soit
\[\frac{n+1}{2}\geq\frac{1}{n^2}\left(\sum_{i=1}^n\frac{i}{x_i}\right)\left(\sum_{i=1}^nx_i\right).\]
Or d'après l'inégalité de Tchebychev,
\[\frac{1}{n^2}\left(\sum_{i=1}^n\frac{i}{x_i}\right)\left(\sum_{i=1}^nx_i\right)\leq \frac{1}{n}\sum_{i=1}^ni=\frac{n+1}{2}.\]
\end{sol}

\begin{sol}%14
\begin{itemize}
\item On a envie d'utiliser l'inégalité de Minkowski :
\[S'+1=\left(\frac{1}{2n}\sum_{i=1}^{2n}(x_i+1)^n\right)^{\frac{1}{n}}+\left(\frac{1}{2n}\sum_{i=1}^{2n}1^n\right)^{\frac{1}{n}}\]
\[S'+1\geq \left(\frac{1}{2n}\sum_{i=1}^{2n}(x_i+2)^n\right)^{\frac{1}{n}}=S\geq P.\]
On a donc montré que $S'\geq\frac{3}{4}P$ dans le cas où $P\geq4$.
\item Supposons maintenant que $P\leq4$. D'après l'inégalité arithmético-géométrique,
\[S'\geq \left(\prod_{i=1}^{2n}(x_i+1)\right)^{\frac{1}{2n}}.\]
Maintenant, d'après l'inégalité de Hölder,
\[S'\geq \left(\prod_{i=1}^{2n}(x_i+1)\right)^{\frac{1}{2n}}\geq \left(\prod_{i=1}^{2n}x_i\right)^{\frac{1}{2n}}+\left(\prod_{i=1}^{2n}1\right)^{\frac{1}{2n}}=\sqrt{P}+1.\]
Il reste à vérifier que $\sqrt{P}+1\geq\frac{3}{4}P.$ Or c'est le cas parce que $\frac{3}{4}x^2-x-1=\frac{1}{4}(2-x)(2+3x)$, ce qui est positif sur $[0,2]$.
\end{itemize}
\end{sol}


\begin{sol}%15
On utilise les relations de Viète :
\[\sum_{k=1}^nx_k^2=a_1^2-2a_2 \text{~~~ et ~~~} \sum_{k=1}^{n-1}y_k^2=\left(\frac{n-1}{n}a_1\right)^2-2\frac{n-2}{n}a_2.\]
On en déduit :
\[n(n-1)\sum_{k=1}^nx_k^2-n^2\sum_{k=1}^{n-1}y_k^2=n(n-1)(a_1^2-2a_2)-(n-1)^2a_1^2+2n(n-2)a_2\]
\[n(n-1)\sum_{k=1}^nx_k^2-n^2\sum_{k=1}^{n-1}y_k^2=(n-1)a_1^2-2na_2=(n-1)\sum_{i=1}^nx_k^2-2\sum_{i<j}x_ix_j=\sum_{i<j}(x_i-x_j)^2\geq0.\]
\end{sol}

\end{document}
