\begin{exo}

\medskip

Montrer que pour tout entier strictement positif $n$ il existe $n$ entiers positifs consécutifs dont aucun n'est une puissance entière d'un nombre premier. 

\end{exo}

%\vspace{1cm}
\bigskip

\begin{sol}

\medskip

L'énoncé : "Montrer que pour tout entier strictement positif $n$ il existe $n$ entiers positifs consécutifs dont aucun n'est un nombre premier" est assez connu, et une solution classique consiste à considérer les entiers : $(n+1)! +k$ pour $k$ variant de $2$ à $n+1$. $k$ divise $(n+1)!$, donc $(n+1)! + k$, et comme $1 \neq k \neq (n+1)!+k$, $(n+1)! + k$ n'est pas premier. Mais l'énoncé du présent exercice, problème 5 de l'Olympiade Internationale 1989, nécessite un peu plus de finesse, car par exemple : $3! + 2 = 8$, $3! + 3 = 9$ et $4! + 3 = 27$ sont des puissances entières de nombres premiers. 

Une des solutions possibles consiste à considérer les $n$ entiers : $(n+1)!^2 + k$ pour $k$ allant de $2$ à $n+1$. Il est clair que $k$ divise $(n+1)!^2 + k$, donc si ce dernier est une puissance entière d'un nombre premier $p$, $k$ est lui aussi une puissance entière de ce même nombre premier : $k = p^{\alpha}$ avec $\alpha \geq 1$. Or $k^2 = p^{2\alpha}$ divise $(n+1)!^2$, de sorte que $(n+1)!^2 + k \equiv p^{\alpha} \pmod{p^{2\alpha}}$ et ne peut pas être une puissance entière de $p$. 

Yakob propose une solution différente et ingénieuse. Considérons $2n$ nombres premiers distincts : $p_1, \ p_2, \ \cdots p_{2n}$ et groupons-les deux par deux : $a_1 = p_1 \times p_2, \ a_2 = p_3 \times p_4, \ \cdots \ \ a_n = p_{2n-1} \times p_{2n}$. D'après le lemme chinois, il existe un entier $N$ tel que $N \equiv 1 \pmod{a_1}, \ N \equiv 2 \pmod{a_2}, \ \cdots \ N \equiv n \pmod{a_n}$. Chacun des $N$ entiers consécutifs $N-1, \ N-2, \ \cdots \ N-n$ a au moins deux facteurs premiers car pour $1 \leq k \leq n$, $N - k$ est divisible par $a_k$, donc aucun n'est une puissance entière d'un nombre premier.


\end{sol}

\bigskip

\begin{exo}

\medskip

Soit $p$ un nombre premier. Montrer que l'ensemble $X = \left\{ p-n^2  \mid n \in \mathbb{N}^* \right\}$ contient deux éléments distincts $x$ et $y$ tels que $x \neq 1$ et $x$ divise $y$.

\end{exo}

%\vspace{1cm}

\bigskip

\begin{sol}

\medskip

Cet énoncé est étrange, pourtant il paraît qu'il a été posé à l'Olympiade Balkanique 1996, pour $p$ premier strictement supérieur à $5$. En effet, tel qu'il est posé, il admet des solutions évidentes, que $p$ soit ou non premier, qu'il soit ou non supérieur à $5$. Par exemple, Yakob propose : $n = p$ et $m = p^2$, car $x = p - p^2$ divise toujours $y = p - p^4$. Linda fait remarquer que pour que $x = p - n^2$ divise $y = p - m^2$, il faut et il suffit qu'il divise la différence : $n^2 - m^2 = (n - m)(n+m)$. Il suffit par exemple que $p - n^2 = n - m$, de sorte que pour n'importe quel entier $n > \sqrt{p} > 1$, on peut trouver $m = n^2 + n - p > n > 1$ vérifiant : $x = p - n^2$ divise $y = p - m^2$. 

Il est vraisemblable que l'énoncé exact considérait l'ensemble $X = \left\{ p-n^2  \mid n \in \mathbb{N}^*, \ n < \sqrt{p} \right\}$ pour $p$ premier strictement supérieur à $5$. Avec cette restriction $n < \sqrt{p}$, le résultat n'est plus vrai pour $p = 2, \ 3,$ ou $5$, ni pour $p$ non premier. L'idée est de montrer que le plus petit élément de cet ensemble fini d'entiers divise un autre élément de $X$, sous réserve que ce plus petit élément de $X$ peut être égal à $1$, or $x = 1$ est exclu par hypothèse. 

Dans ce premier cas où $p = n^2 + 1$, $n$ est nécessairement pair, donc l'élément suivant : $x = p - (n-1)^2 = 2n > 1$ divise l'élément le plus grand : $y = p - 1 = n^2$, et ces deux éléments sont distincts car par hypothèse $n > 2$.

Dans le second cas, où $n^2 + 1 < p < (n+1)^2$ ($n \geq 2$), l'élément le plus petit : $x = p - n^2$ est compris entre $2$ et $2n$. Or il ne peut pas être égal à $2n$ ni à $n$ car $n^2 + 2n$ et $n^2 + n$ sont non premiers. Donc $0 < \mid x-n \mid < n$ de sorte que $y = p - (n-x)^2$ est bien un élément de $X$ distinct de $x$. Or $y = \left(p-n^2\right) + 2nx - x^2 = x(1+2n-x)$ est multiple de $x$ dans la mesure où $1+2n - x \geq 2$.

\end{sol}

\bigskip

\begin{exo}

\medskip

Déterminer tous les entiers $n > 1$ tels que $\dfrac{2^n+1}{n^2}$ soit un entier.

\end{exo}

%\vspace{1cm}

\bigskip

\begin{sol}

\medskip

Cet exercice est un grand classique : c'est le problème 3 de l'Olympiade Internationale 1990, celle de Pékin où la France s'est classée 5ème avec trois médailles d'or, notamment grâce à Vincent Lafforgue (petit frère du médaillé Fields Laurent Lafforgue), qui a obtenu, en 1990 et 1991, deux scores maximums de 42/42. Notons toutefois qu'il y avait moitié moins de pays candidats qu'aujourd'hui : plusieurs pays d'Europe se sont subdivisés par la suite.

Cet exercice utilise des techniques de raisonnement qui sont devenues classiques et utiles pour un grand nombre de problèmes. Tout d'abord, l'idée de chercher le plus petit facteur premier $p$ de $n$, en utilisant la notion d'ordre de $2$ modulo $p$. $n$ étant impair, $p \geq 3$. On appelle ordre de $2$ modulo $p$ le plus petit exposant non nul, $b$, tel que $2^b \equiv 1 \pmod{p}$, et on montre que $b$ divise tout autre exposant $a$ tel que $2^a \equiv 1 \pmod{p}$. Il suffit d'écrire la division euclidienne $a = bq + r$ et de remarquer que $2^{bq} \equiv 1 \pmod{p}$ pour en déduire $2^r \equiv 1 \pmod{p}$, avec $0 \leq r < b$. Comme $b$ est le plus petit entier strictement positif vérifiant cette relation, nécessairement $r = 0$ donc $a = bq$. 

Or d'après le théorème de Fermat, $2^{p-1} \equiv 1 \pmod{p}$, ce qui entraîne que l'ordre de $2$ divise $p-1$. Et comme $p$ divise $n$, donc $2^n+1$, donc $2^{2n}-1 = \left(2^n+1\right)\left(2^n-1\right)$, l'ordre de $2$ divise également $2n$. Or il est premier avec $n$, vu qu'il est strictement inférieur à $p$ et que $p$ est, par hypothèse, le plus petit diviseur de $n$ autre que $1$ (l'ordre de $2$ ne peut évidemment pas être $1$, car pour aucun nombre premier $p$  $2$ n'est congru à $1$ modulo $p$). Donc, d'après le théorème de Gauss, il divise $2$ : il ne peut valoir que $2$, de sorte que $p$ divise $2^2 - 1 = 3$, et le plus petit facteur premier $p$ de $n$ est nécessairement $3$. 

Si l'on cherchait les entiers $n$ tels que $n$ divise $2^n+1$, on en aurait beaucoup, et leurs facteurs premiers pourraient être nombreux, même si le plus petit d'entre eux est nécessairement 3. Par exemple, si $n$ divise $2^n+1 = nq$, $2^n+1$ divise $2^{nq}+1 = 2^{2^n+1}$ car $q$ est impair, donc $2^n+1$ est lui aussi solution. $3, \ 9$, mais aussi $513 = 2^9+1$ et même le tiers de $513$, $171 = 9 \times 19$ sont solutions. Seulement ces nombres ne vérifient pas : $2^n+1$ divisible par $n^2$, car par exemple, $2^9+1 = 3^3 \times 19$ n'est pas divisible par $9^2 = 3^4$. 

La seconde étape de la démonstration consiste donc à prouver que $n$ ne peut pas être divisible par $9$. Posons $n = 3^kd$, avec $k \geq 1$ et $d$ impair non divisible par $3$. Montrons par récurrence sur $k$ que la plus grande puissance de $3$ divisant $2^n+1$ est $3^{k+1}$. Pour $k = 1$, $2^{3d} + 1 = \left( 2^3+1 \right) \left( 2^{3(d-1)} - 2^{3(d-2)} + \cdots - 2^3 + 1 \right)$. La seconde parenthèse contient une somme de $d$ termes tous congrus à $1$ modulo $9$, vu que $2^3 \equiv -1 \pmod{9}$, elle est donc congrue à $d$ modulo $9$, ce qui entraîne qu'elle est non multiple de $3$, et la première parenthèse vaut précisément $9$, ce qui initialise la récurrence en prouvant le résultat pour $k = 1$, quel que soit le facteur $d$. Maintenant si le résultat est vrai pour un $k \geq 1$ donné, vu que $2^{3n}+1 = \left(2^n+1\right)\left(2^{2n}-2^n+1\right)$ et que $2^n \equiv -1 \pmod{3^{k+1}}$, $2^{2n}-2^n+1 \equiv 1+1+1=3 \pmod{9}$ est divisible par $3$ et pas par $9$, ce qui achève la démonstration. Il en résulte que si $k \geq 2$ et $n = 3^kd$, la plus grande puissance de $3$ divisant $2^n+1$ est $3^{k+1}$, alors que $n^2$ est divisible par $3^2k$. Donc une solution de notre problème est nécessairement multiple de $3$ et non multiple de $9$.

Ce résultat se généralise par une démonstration analogue : $p$ étant un nombre premier donné, si l'on note $v_p(n)$ l'exposant de la plus grande puissance de $p$ divisant $n$, et si $a$ et $b$ sont deux entiers tels que $p$ divise $a-b$ mais ne divise ni $a$ ni $b$, alors : $$v_p\left(a^n - b^n\right) = v_p(a-b) + v_p(n)$$ C'est le théorème LTE (Lifting The Exponent) qui permet de résoudre bien des exercices. Dans le présent problème, on applique ce théorème LTE pour $p = 3$,  $a = 2$ et $b = -1$. 

Que peut-on dire, maintenant, du facteur $d$ ? Pour quelles valeurs de $d$ non multiples de $3$ $2^n+1$ est-il divisible par $n^2$ avec $n = 3d$ ? Soit $p$ le plus petit facteur premier de $d$ ($p\geq 5)$. L'ordre de $2$ modulo $p$ est divise $p-1$ en vertu du théorème de Fermat, mais il divise aussi $2n = 6d$ vu que $2^{2n} \equiv 1 \pmod{p}$, et il ne divise par $n$ vu que $2^n \equiv -1 \pmod{p}$, donc il est pair. Comme il est strictement inférieur à $p$, plus petit facteur premier de $d$, il est premier avec $d$. Donc il divise $6$ (théorème de Gauss). Les seuls diviseurs pairs de $6$ sont $2$ et $6$, or $2$ ne peut pas être d'ordre $2$ car $2^2 - 1 = 3$ n'a pas de facteur premier autre que $3$, il ne peut pas non plus être d'ordre $6$ car $2^6 - 1 = 3^2 \times 7$ admet $p = 7$ pour unique facteur premier différent de $3$, or $7$ n'est pas d'ordre $6$ mais d'ordre $3$ (il divise $2^3-1$, donc il divise également $2^{3d}-1$ et non pas $2^{3d}+1$). $d$ ne pouvant pas avoir de plus petit facteur premier, $d = 1$ donc l'unique entier $n$ tel que $\dfrac{2^n+1}{n^2}$ soit un entier, outre la solution triviale $n = 1$, est $n = 3$.

\end{sol}

\bigskip

\begin{exo}

\medskip

Soit $p$ un nombre premier supérieur ou égal à $5$. On considère deux entiers strictement positifs $m$ et $n$ vérifiant : 
$$ 1 + \dfrac12 + \dfrac13 + ... + \dfrac1{p-1} = \dfrac{m}{n}$$
Montrer que $m$ est divisible par $p^2$. 

\end{exo}

%\vspace{1cm}

\bigskip

\begin{sol}

\medskip

Là encore, prouver que $m$ est divisible par $p$ est très classique et assez facile : il suffit de regrouper les termes : 
$$ \dfrac{m}{n} = 1 + \dfrac12 + \dfrac13 + ... + \dfrac1{p-1} = \left( 1 + \dfrac1{p-1} \right) + \left( \dfrac12 + \dfrac1{p-2} \right) + \cdots + \left( \dfrac1{k} + \dfrac1{p-k} \right) + \left( \dfrac1{\left(\frac{p-1}{2}\right)} + \dfrac1{\left(\frac{p+1}{2}\right)} \right) $$

$$ = \left( \dfrac{p}{1 \times (p-1)} \right) + \left( \dfrac{p}{2 \times (p-2)} \right) + \cdots + \left( \dfrac{p}{\left(\frac{p-1}{2}\right) \times \left(\frac{p+1}{2}\right)} \right) = p \left( \dfrac1{1 \times (p-1)} + \cdots +  \dfrac1{\left(\frac{p-1}{2}\right) \times \left(\frac{p+1}{2}\right)} \right)$$
 
 
Il est clair que $p$ ne se simplifie avec aucun des termes du dénominateur, puisqu'il est premier et que tous les facteurs du dénominateur lui sont strictement inférieurs. Donc $p$ divise $m$. Mais pour montrer que $p^2$ divise $n$, il faut en outre démontrer que le numérateur de  $\dfrac1{1 \times (p-1)} + \cdots +  \dfrac1{\left(\frac{p-1}{2}\right) \times \left(\frac{p+1}{2}\right)}$ est lui aussi divisible par $p$. 

\medskip

Réduisons ces fractions au même dénominateur : $1 \times 2 \times \cdots \times (p-1) = (p-1)!$, qui bien évidemment n'est pas divisible par $p$, il est même congru à $-1$ modulo $p$, d'après le théorème de Wilson. Au numérateur, nous aurons une somme de termes $a_k$ pour $1 \leq k \leq \frac{p-1}{2}$, avec : $a_k$ produit de tous les entiers entre $1$ et $p-1$ à l'exception de $k$ et $p-k$. En d'autres termes, $a_k \times k \times (p-k) = (p-1! \equiv -1 \pmod{p}$. Or pour tout entier $k$, $k \times (p-k) = kp - k^2 \equiv -k^2 \pmod{p}$. Donc $a_k \times k^2 \equiv 1 \pmod{p}$. 

\smallskip

Or à tout $k$ vérifiant $1 \leq k \leq \frac{p-1}{2} $ on associe un et un seul élément de ce même sous-ensemble de $\mathbb{Z}/p\mathbb{Z}$, élément que nous noterons $(\pm1/k)$, vérifiant : $k^2 \times (\pm1/k)^2 = 1$. En effet $k$ admet un inverse dans $\mathbb{Z}/p\mathbb{Z}$, que l'on notera $(1/k)$, vérifiant $k \times (1/k) \equiv 1 \pmod{p}$, et l'inverse de $k^2$ est le carré de cet inverse $(1/k)$, mais c'est aussi le carré de son opposé $(-1/k)$, car pour tout $x$, $x^2 = (-x)^2$. Un et un seul des deux éléments $(1/k)$ et $(-1/k)$ de $\mathbb{Z}/p\mathbb{Z}$, celui que je note $(\pm1/k)$, est compris entre $1$ et $\frac{p-1}{2}$ modulo $p$. Si bien que la somme des $a_k$ est congrue à la somme des $(\pm1/k)^2$ modulo $p$, donc à la somme des $k^2$ modulo $p$, puisque les $\frac{p-1}{2}$ éléments $(\pm1/k)$ sont tous les éléments $1, \ 2, \ \cdots \ \frac{p-1}{2}$ dans le désordre. En d'autres termes, la somme des $a_k$ vaut, modulo $p$, $1^2 + 2^2 + \cdots + \left(\frac{p-1}{2}\right)^2 = \dfrac{\frac{p-1}{2} \times \frac{p+1}{2} \times p}{6}$, vu que pour tout entier $n$, $1^2 + 2^2 + \cdots + n^2 = \dfrac{n(n+1)(2n+1)}{6}$ (ce qui se démontre classiquement par récurrence). Si $p \geq 5$, $p$ est premier avec $6$ donc il ne se simplifie pas et divise obligatoirement le numérateur de $\dfrac{\frac{p-1}{2} \times \frac{p+1}{2} \times p}{6}$, donc la somme des $a_k$, donc le numérateur de  $\dfrac1{1 \times (p-1)} + \cdots +  \dfrac1{\left(\frac{p-1}{2}\right) \times \left(\frac{p+1}{2}\right)}$, ce qui prouve bien que $p^2$ divise le numérateur de $1 + \dfrac12 + \dfrac13 + ... + \dfrac1{p-1}$.

%Or à tout élément $k \neq 0$ de $\mathbb{Z}/p\mathbb{Z}$ on associe bijectivement son inverse, que l'on notera $(1/k)$, à savoir l'unique élément non nul de $\mathbb{Z}/p\mathbb{Z}$ tel que $k \times (1/k) \equiv 1 \pmod{p}$. La somme des $k^2$ lorsque $k$ parcourt $\mathbb{Z}/p\mathbb{Z}^*$ vaut, modulo $p$, deux fois la somme des $k^2$ lorsque  

\end{sol}

\bigskip


\begin{exo}

\medskip

Résoudre l'équation diophantienne :
$$ x^2 = y^3 - 5 $$

\end{exo}

%\vspace{1cm}
\bigskip

\begin{sol}

\medskip

Commençons par chercher des informations modulo un entier donné, en l'occurrence modulo $4$ car $x^2$ ne prend que deux valeurs modulo $4$ : $0$ et $1$. Donc $y^3 = x^2 + 5$ doit être congru à $1$ ou $2$ modulo $4$. Le cube d'un nombre pair étant divisible par $8$, et le cube d'un nombre impair étant impair, $y^3$ ne peut pas être congru à $2$ modulo $4$. Donc $y^3 \equiv 1 \pmod{4}$, ce qui entraîne $y \equiv 1 \pmod{4}$ car $3^3 \equiv 3 \pmod{4}$.

Dès lors, l'équation peut s'écrire : $x^2 + 4 = y^3 - 1 = (y-1)\left( y^2 + y + 1 \right)$. Or $y^2 + y + 1 \equiv 3 \pmod{4}$ possède nécessairement au moins un facteur premier congru à $3$ modulo (4). Et si $p \equiv 3 \pmod{4}$ divise une somme de carrés $a^2 + b^2$, alors il divise $a$ et il divise $b$. Ce résultat classique  peut se démontrer par l'absurde ainsi : si $p = 4k + 3$ divise $a^2 + b^2$ et ne divise pas $a$, il ne divise pas non plus $b$, mais $a^4 \equiv b^4 \pmod{p}$, donc $a^{4k} \equiv b^{4k} \pmod {p}$ soit, en multipliant par $a^2b^2$, $a^{4k+2}b^2 \equiv b^{4k+2}a^2 \pmod{p}$. Or comme $a$ et $b$ sont premiers avec $p$, $a^{4k+2} \equiv b^{4k+2} \equiv 1 \pmod{p}$. Dès lors $b^2 \equiv a^2 \pmod{p}$, d'où $p$ divise $2a^2$ et comme $p$ est premier impair, $p$ divise $a$, ce qui contredit l'hypothèse. Donc $p$ divise $a$, ce qui entraîne qu'il divise également $b$. Or ici, $b = 2$ n'a pas de diviseur congru à $3$ modulo $4$ : contradiction. Cela prouve que l'équation n'admet pas de solutions. Cette solution a été proposée par Joachim.

\end{sol}

\bigskip


\begin{exo}

\medskip

Prouver qu'il n'existe pas d'entiers strictement positifs $a$ et $b$ tels que, pour tous nombres premiers distincts $p$ et $q$ supérieurs à 1000, le nombre $ap + bq$ soit lui aussi premier.

\end{exo}

\bigskip

\begin{sol}

\medskip

La solution de Yakob est différente de celle que j'avais préparée, et elle est plus élégante. Supposons qu'il existe de tels nombres $a$ et $b$. Soient $p$ et $q$ deux nombres premiers distincts supérieurs à 1000, tels que $p$ ne divise ni $a$ ni $a-1$ (choisissons par exemple $p > a$). Par hypothèse, $p_1 = ap + bq$ est lui aussi un nombre premier, et $p_1 > p > 1000$. Plus généralement, par une récurrence immédiate, la suite des $p_n = ap_{n-1} + bq$ est une suite croissante de nombres premiers supérieurs à 1000. Or, toujours par récurrence, $p_2 = a(ap+bq) + bq = a^2p + (a+1)bq = a^2p + \dfrac{a^2-1}{a-1}$, et plus généralement $p_n = a^np + \dfrac{a^n-1}{a-1}bq$. Comme $a$ est premier avec $p$, $a^{p-1} \equiv 1 \pmod{p}$, donc $p$ divise $a^{p-1} - 1$ et ne divise pas $a-1$ par hypothèse. Dès lors $p$ divise $p_{p-1} = a^{p-1}p + \dfrac{a^{p-1}-1}{a-1}bq$, et $p < p_{p-1}$, donc $p_{p-1}$ n'est pas premier, contradiction : on a donc prouvé par l'absurde que de tels $a$ et $b$ n'existent pas. 

\end{sol}

