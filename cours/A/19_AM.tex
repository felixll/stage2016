\subsubsection{Calculs basiques}

\begin{prop}[La factorisation simple] $k \times (a+b) = k \times a + k \times b$ \end{prop}

\begin{proof}
	\begin{center}
		
		\definecolor{qqqqff}{rgb}{0,0,1}
		\definecolor{dcrutc}{rgb}{0.86,0.08,0.24}
		\definecolor{ffqqqq}{rgb}{1,0,0}
		\begin{tikzpicture}[line cap=round,line join=round,>=triangle 45,x=1.0cm,y=1.0cm]
		\clip(-1.92,1.24) rectangle (6.1,5.88);
		\fill[color=dcrutc,fill=dcrutc,fill opacity=0.1] (-1,5) -- (3,5) -- (3,2) -- (-1,2) -- cycle;
		\fill[color=qqqqff,fill=qqqqff,fill opacity=0.1] (3,5) -- (5,5) -- (5,2) -- (3,2) -- cycle;
		\draw [color=dcrutc] (-1,5)-- (3,5);
		\draw [color=dcrutc] (3,5)-- (3,2);
		\draw [color=dcrutc] (3,2)-- (-1,2);
		\draw [color=dcrutc] (-1,2)-- (-1,5);
		\draw [color=qqqqff] (3,5)-- (5,5);
		\draw [color=qqqqff] (5,5)-- (5,2);
		\draw [color=qqqqff] (5,2)-- (3,2);
		\draw [color=qqqqff] (3,2)-- (3,5);
		\draw [color=ffqqqq](1.08,5.5) node[anchor=north west] {a};
		\draw [color=qqqqff](3.82,5.46) node[anchor=north west] {b};
		\draw (-1.52,3.66) node[anchor=north west] {k};
		\draw [color=ffqqqq](0.88,3.6) node[anchor=north west] {\textbf{ka}};
		\draw [color=qqqqff](3.7,3.56) node[anchor=north west] {\textbf{kb}};
		\begin{scriptsize}
		\fill [color=ffqqqq] (-1,5) circle (1.5pt);
		\fill [color=black] (3,5) circle (1.5pt);
		\fill [color=black] (3,2) circle (1.5pt);
		\fill [color=ffqqqq] (-1,2) circle (1.5pt);
		\fill [color=qqqqff] (5,5) circle (1.5pt);
		\fill [color=qqqqff] (5,2) circle (1.5pt);
		\end{scriptsize}
		\end{tikzpicture}
		
		
	\end{center}
\end{proof}

\begin{prop}[La factorisation du rectangle] $(a+b) \times (c+d) = ac + ad + bc + bd$ \end{prop}

\begin{proof}
	\begin{center}
		
		\definecolor{xfqqff}{rgb}{0.5,0,1}
		\definecolor{qqffqq}{rgb}{0,1,0}
		\definecolor{ffqqqq}{rgb}{1,0,0}
		\definecolor{ffxfqq}{rgb}{1,0.5,0}
		\begin{tikzpicture}[line cap=round,line join=round,>=triangle 45,x=1.0cm,y=1.0cm]
		\clip(-3.98,-0.64) rectangle (5.88,6.06);
		\fill[color=ffxfqq,fill=ffxfqq,fill opacity=0.1] (-3,5) -- (2,5) -- (2,2) -- (-3,2) -- cycle;
		\fill[color=ffqqqq,fill=ffqqqq,fill opacity=0.1] (2,5) -- (5,5) -- (5,2) -- (2,2) -- cycle;
		\fill[color=qqffqq,fill=qqffqq,fill opacity=0.1] (-3,2) -- (-3,0) -- (2,0) -- (2,2) -- cycle;
		\fill[color=xfqqff,fill=xfqqff,fill opacity=0.1] (2,2) -- (5,2) -- (5,0) -- (2,0) -- cycle;
		\draw [color=ffxfqq] (-3,5)-- (2,5);
		\draw [color=ffxfqq] (2,5)-- (2,2);
		\draw [color=ffxfqq] (2,2)-- (-3,2);
		\draw [color=ffxfqq] (-3,2)-- (-3,5);
		\draw [color=ffqqqq] (2,5)-- (5,5);
		\draw [color=ffqqqq] (5,5)-- (5,2);
		\draw [color=ffqqqq] (5,2)-- (2,2);
		\draw [color=ffqqqq] (2,2)-- (2,5);
		\draw [color=qqffqq] (-3,2)-- (-3,0);
		\draw [color=qqffqq] (-3,0)-- (2,0);
		\draw [color=qqffqq] (2,0)-- (2,2);
		\draw [color=qqffqq] (2,2)-- (-3,2);
		\draw [color=xfqqff] (2,2)-- (5,2);
		\draw [color=xfqqff] (5,2)-- (5,0);
		\draw [color=xfqqff] (5,0)-- (2,0);
		\draw [color=xfqqff] (2,0)-- (2,2);
		\draw (-0.98,5.54) node[anchor=north west] {a};
		\draw (3.24,5.54) node[anchor=north west] {b};
		\draw (-3.5,3.58) node[anchor=north west] {c};
		\draw (-3.52,1.2) node[anchor=north west] {d};
		\draw (-1,3.7) node[anchor=north west] {ab};
		\draw (-0.96,1.12) node[anchor=north west] {ad};
		\draw (3.24,3.68) node[anchor=north west] {bc};
		\draw (3.2,1.22) node[anchor=north west] {bd};
		\begin{scriptsize}
		\fill [color=black] (-3,5) circle (0.5pt);
		\fill [color=black] (5,5) circle (0.5pt);
		\fill [color=black] (2,5) circle (0.5pt);
		\fill [color=black] (2,2) circle (0.5pt);
		\fill [color=black] (-3,2) circle (0.5pt);
		\fill [color=black] (5,2) circle (0.5pt);
		\fill [color=black] (-3,0) circle (0.5pt);
		\fill [color=black] (2,0) circle (0.5pt);
		\fill [color=black] (5,0) circle (0.5pt);
		\end{scriptsize}
		\end{tikzpicture}
		
	\end{center}
\end{proof}

\begin{prop}[Identités remarquable] $$(a+b)^2 = a^2 +2ab + b^2$$
	$$(a-b)^2 = a^2 - 2ab + b^2$$
	$$(a+b)(a-b) = a^2 - b^2$$
	$$(a+b)^3 = a^3 +3a^{2}b + 3ab^{2} + b^3$$
\end{prop}
\begin{proof}
	$$(a+b)^2 = (a+b)(a+b) = a^2 +ab+ ba + b^2 = a^2 +2ab +b^2$$
\end{proof}

Plus généralement on peut montrer que le triangle de Pascal permet d'obtenir les coefficients de le développement remarquable de $(a+b)^n$

En pratique, ces formules sont aussi utiles dans les problèmes d'arithmétique avec les entiers : Une expression factorisée laisse apparaitre des diviseurs.

\par\medskip
\begin{prop}[Calculs avec puissances] $$a^m \times a^n = a^{m+n}$$
$$\frac{a^m}{a^n}=a^{m-n}$$
$$ (a^m)^n = a^{mn}$$
\end{prop}
\par\medskip

\subsubsection{Quelques sommes classiques}

\begin{exo}
	Calculer $$ S_1= 1+2+3+4+.....+2014+2015+2016$$
	et $$S_2 = 2016+2015+2014+2013+....+3+2+1$$
\end{exo}
\begin{sol}
	Il est clair par commutation que $$ S_1= S_2 $$ 
	De plus $$S_1 + S_2 = 2017+ 2017 + ... +2017= 2017\times 2016$$
	Donc $$S_1=\frac{2016\times2017}{2}$$
\end{sol}

On peut généraliser ce calcul à toute suite ayant une progression constante : 

\begin{prop}
	Soit $(u_n)$ une suite telle que pour tout rang n entier : $u_{n+1} = u_n + b$ avec b réel. 
	Alors :$$u_n = u_0 + nb$$
	et $$ u_0 + u_1 + ... + u_n = \frac{(n+1)(u_0 + u_n)}{2}$$
\end{prop}

\begin{exo}
	Calculer $$ S=7+10+13+16+...+994$$
\end{exo}
\begin{sol}
	Il y a 329 termes dans la somme
	$$S= \frac{(min + max)(nombre)}{2}=  \frac{(7 + 994)(329)}{2}=\frac{329329}{2}$$
\end{sol}
\begin{exo}
	Calculer $$ P= 1+ \pi^1 + \pi^2 + \pi^3 +...+ \pi^{10}$$
à l'aide de $$\pi\times P=\pi^1 + \pi^2 +...+ \pi^{11} $$
\end{exo}
\begin{sol}
	$$\pi\times P - P = \pi^{11} - 1 $$
	$$P = \frac{\pi^{11} - 1}{\pi - 1}$$
\end{sol}

Vous l'aurez compris, on peut remplacer $\pi$ par n'importe quel autre nombre entier, réel et (même complexe) à CONDITION que l'on ne divise pas par zéro. C'est à dire que l'on ne peux pas calculer $$1^0+1^1+1^2+1^3+...+1^n$$ avec cette technique. Cela dit, celle-ci n'est pas très difficile à calculer. 

De ces affirmations, on en déduit une factorisation non triviale : 
\begin{prop}
	$$(x^n-1)= (x-1)(1+x+x^2+x^3+...+x^{n-1})$$
	et même en posant $x=a/b$
	$$a^n - b^n= (a-b)(a^{n-1}+a^{n-2}b+...+ab^{n-2}+b^{n-1})$$
\end{prop}

 
 \begin{exo}
 	1) Trouvez tous les $(x,y)\in \mathbb{N}$ tels que $x^2+y^2 = 25$
 	
 	2) Trouvez tous les $(x,y)\in \mathbb{R}$ tels que $x^2+y^2 = 25$
 \end{exo}
 
 \par \medskip
 
 \begin{sol}
 	
 	1) $S=\{(0,5);(4;3);(3;4);(5;0)\}$
 	
 	2) $S=\{(x;\sqrt{25-x^2}) | x \leq 5\}$
 	
On remarque que l'ensemble des solutions n'est pas le même
 \end{sol}
 
Nous travaillons désormais uniquement avec des nombres réels

\subsubsection{L'\'equation du second degr\'e}
Consid\'erons une \'equation de la forme $ax^2+bx+c=0$ d'inconnue $x$, avec $a,b,c$ fix\'es et $a\ne 0$.
Comme
\begin{eqnarray*}
 ax^2+bx+c &=& a\left(x^2+\frac{b}{a}x+\frac{c}{a}\right)\\
&=& a \left(x^2+2\frac{b}{2a}x+(\frac{b}{2a})^2-(\frac{b}{2a})^2+\frac{c}{a}\right)\\
&=& a \left((x+\frac{b}{2a})^2-\frac{b^2-4ac}{a^2}\right),
\end{eqnarray*}
on est amen\'e \`a poser
$$\Delta=b^2-4ac.$$
On appelle $\Delta$ le discriminant. L'\'equation $ax^2+bx+c=0$ \'equivaut \`a $(x+\frac{b}{2a})^2=\frac{\Delta}{a^2}$.
\par\medskip

1) Si $\Delta>0$, alors $ax^2+bx+c=0$ \'equivaut \`a $x+\frac{b}{2a}=\pm \frac{\sqrt{\Delta}}{a}$, ce qui se simplifie en
$$x=\frac{-b\pm\sqrt{\Delta}}{2a}.$$
\par\medskip

2) Si $\Delta=0$, l'\'equation a une et une seule solution $x=-\dfrac{b}{2a}$.
\par\medskip

3) Si $\Delta<0$, l'\'equation n'a pas de solution.
\par\medskip

Le graphe d'un polyn\^ome du second degr\'e est une parabole, orient\'ee vers le haut ou vers le bas selon le signe de $a$.

Le signe de $\Delta$ est li\'e au nombre de points d'intersection de la parabole avec l'axe des abscisses.

\begin{exo}
	Pour quelles valeurs de $a$ l'équation suivante n'admet aucune solution réélle pour $x$ : 
	$$(a^{2}+2a)x^{2}+(3a)x+1=0$$
\end{exo}
\begin{sol}
	Il est important de considérer $a$ comme une constante qui ne change pas dans tous le problème. On est ramené à une équation de degré 2 en la variable $x$
	Dès que l'on a une équation de degré 2 en une variable on la résout
	$$\Delta  = (3a)^{2}-4(a^{2}+2a)$$
	$$\Delta  =  5a^{2}-10a$$
	On souhaite avoir aucune solution dans l'équation en $x$ donc $\Delta < 0$
	$$5a^{2}-10a < 0$$
	$$a(a-2)<0$$
	Donc $$0<a<2$$
\end{sol}


\begin{prop}[Somme et produit des racines]
 Consid\'erons une \'equation de la forme $x^2-Sx+P=0$ o\`u $S$ et $P$ sont deux r\'eels. Supposons qu'il y ait deux racines (une racine c'est une valeur de x qui annule le polynôme). Alors $S$ est la somme, et $P$ est le produit de ces deux racines.
\end{prop}

Il suffit en effet de calculer $x_1+x_2$ et $x_1x_2$ o\`u $x_1=\frac{S+\sqrt{S^2-4P}}{2}$ et $x_2=\frac{S-\sqrt{S^2-4P}}{2}$.

\begin{exo}
	Alice pense dans sa tête à deux nombres rééls.
	Elle écrit sur un papier leurs produit ainsi que leurs sommes.
	Alice à écrit les nombres $12$ et $7$.

Quels sont les deux nombres d'Alice ? 
\end{exo}


\begin{sol}
On résous d'une part l'équation : $$X^2 - 12X + 7$$
$X= 3$ et$ Y=4 $ est une solution 

MAIS on résous aussi l'équation : 
$$ X^2 - 7X + 12 = 0$$
$X= 6 + \sqrt{29}$ et $Y=6 - \sqrt{29} $
\end{sol}


\begin{exo}
 Calculer $x_1^2+x_2^2$ en fonction de $S^2$ et de $P$.
\end{exo}
 
\begin{sol}
 Comme $x_1^2=Sx_1-P$ et de m\^eme pour $x_2$, on a $x_1^2+x_2^2=S^2-2P$.
\end{sol}


\par\bigskip

La technique de r\'esolution des \'equations du second degr\'e permet de r\'esoudre certains types d'\'equations de degr\'e sup\'erieur.

\begin{exo}
 R\'esoudre l'\'equation $x^4-3x^2-1=0$.
\end{exo}

\begin{sol}
En posant $y=x^2$, on obtient $y=\frac{3\pm\sqrt{13}}{2}$. Une seule de ces solutions \'etant positive, on trouve
$x=\pm\sqrt{\frac{3+\sqrt{13}}{2}}$.
\end{sol}

\begin{exo}
 R\'esoudre l'\'equation $x^4-4x^3+3x^2-4x+1=0$.
\end{exo}

\begin{sol}
 $x=0$ n'\'etant pas solution, on peut diviser l'\'equation par $x^2$, ce qui donne
$$(x^2+\frac{1}{x^2})-4(x+\frac{1}{x})+3=0.$$
Soit $y=x+\frac{1}{x}$. On a $y^2-2-4y+3=0$, donc $y^2-4y+1=0$, ce qui donne
$y=2\pm\sqrt{3}$.

Comme $x^2-yx+1=0$, on peut retrouver $x$ \`a partir de $y$. Le discriminant de cette
derni\`ere \'equation est $y^2-4$, ce qui n\'ecessite $|y|\geqslant 2$. La seule solution
v\'erifiant cette condition est $y=2+\sqrt{3}$.

On trouve enfin $x=\dfrac{2+\sqrt{3}\pm\sqrt{3+4\sqrt{3}}}{2}$.
\end{sol}
