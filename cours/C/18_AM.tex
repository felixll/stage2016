		\begin{exo}
				Montrer que $1+3^m+3^n$, $m,n \in \N$ n'est jamais un carré parfait.
		\end{exo}

		\begin{exo}
				Trouver tous les entiers $m,n$ positifs tels que $3^m-2^n \in \left\{-1, 5, 7\right\}$.
		\end{exo}

		\begin{exo}
				Soit $p$ un nombre premier tel que $p^2+2$ soit premier. Montrer qu'il en est de même pour $p^3+2,\\ p^3+10, p^4+2, p^4-2$. 
		\end{exo}

		\begin{exo}
				Déterminer les entiers $n \in \N$ tels que $2^n+1$ soit un carré ou un cube.
		\end{exo}

		\begin{exo}
				Montrer que pour $x,y \in \N$, on n'a pas $x(x+1)=4y(y+1)$.
		\end{exo}

		\begin{exo}
				Montrer que pour tout $n \geq 2$, $n^4+4^n$ est composé. 
		\end{exo}

		\begin{exo}
				Montrer que $n \geq 2$ est premier si et seulement si $n$ divise $(n-1)!+1$. Pour quels $n$ cette dernière expression est-elle une puissance de $n$?
		\end{exo}

		\begin{exo}
				Montrer que le produit de quatre entiers consécutifs est le prédécesseur d'un carré parfait.
		\end{exo}

		\begin{exo}
				Montrer que si $a$ et $b$ sont somme de deux carrés, $ab$ l'est aussi. 
		\end{exo}

		\begin{exo}
				Montrer que $641|2^{32}+1$, que $421|1280000401$. 
		\end{exo}

		\begin{exo}
				Trouver tous les $n\in \N^*$ tels que $n^n+1$ soit premier inférieur à $10^{18}$. 
		\end{exo}

		\begin{exo}
				Pour $a,b,n \geq 2$ tels que $pgcd(a,b)=1$, montrer que $pgcd\left(\frac{a^n-b^n}{a-b},a-b\right)=pgcd(a-b,n)$.
		\end{exo}

		\begin{exo}
				Calculer, pour $a,b \in \N^*$ premiers entre eux, et $m,n \in \N^*$, $pgcd(a^n-b^n,a^m-b^m)$.
		\end{exo}

		\begin{exo}
				Donner le nombre de zéros à la fin de $1000!$. 
		\end{exo}

		\begin{exo}
				Montrer que $2^n$ ne divise jamais $n!$, mais que pour une infinité de $n$, $2^{n-1}|n!$.
		\end{exo}

		\begin{exo}
				On donne $\left(p_n\right)$ une suite d'entiers telle que pour tout $n \geq 2$, $p_n$ soit le plus grand diviseur premier de $p_{n-1}+p_{n-2}+2016$. Montrer qu'il existe un réel $M$ tel que pour tout $n \geq 0$, $p_n \leq M$. 
		\end{exo}

		\begin{exo}
				Trouver tous les entiers $n \in \N^*$ tels que $n|2^n-1$. 
		\end{exo}

		\begin{exo}
				Trouver tous les entiers $n$ qui sont premiers avec tout nombre de la forme $2^m+3^m+6^m-1$, $m \in \N$.  
		\end{exo}

		\begin{exo}
				Montrer que si $p$ est un premier impair, et $a$ un entier premier avec $p$, $a^{\frac{p-1}{2}}$ est congru à $1$ ou à $-1$ modulo $p$.  
		\end{exo}

		\begin{exo}
				Montrer que $n^5-4$ n'est jamais un carré. 
		\end{exo}

		\begin{exo}
				Trouver tous les triplets $(x,y,z)$ de rationnels tels que $11=x^5+2y^5+5z^5$.
		\end{exo}

		\begin{exo}
				Trouver tous les triplets $(x,y,z)$ d'entiers tels que $x^2+y^2+z^2=2xyz$.
		\end{exo}

		\begin{exo}
				Etant donnés $p,q$ deux entiers premiers entre eux, établir que $\sum \limits_{k=0}^{q-1}{\left\lfloor \frac{kp}{q}\right\rfloor}=\frac{(p-1)(q-1)}{2}$.
		\end{exo}

		\begin{exo}
				Montrer qu'il existe une infinité d'entiers $a$ tels que $a^2$ divise $2^a+3^a$. 
		\end{exo}

		\begin{exo}
				Trouver tous les entiers positifs $m,n$ tels que $7^m-3*2^n=1$.
		\end{exo}

		\begin{exo}
				Trouver tous les entiers positifs $m,n$ tels que $3^m-2^n=1$. 
		\end{exo}

		\begin{exo}
				Soient $a,b,n,c \in \N^*$, tels que $(a+bc)(b+ac)=19^n$. Montrer que $n$ est pair. 
		\end{exo}

		\begin{exo}
				Trouver tous les triplets d'entiers $(a,b,c)$ avec $2 \leq a < b < c$ tels que $(a-1)(b-1)(c-1)|abc-1$. 
		\end{exo}

		\begin{exo}
				Trouver tous les $(x,y) \in (\N^*)^2$ tels que $1+2^x+2^{2x+1}=y^2$. 
		\end{exo}

		\begin{exo}
				Montrer que pour $m,n$ deux entiers, $m!n!(m+n)!$ divise $(2m)!(2n)!$. 
		\end{exo}

		\begin{exo}
				Montrer que pour $n \geq 11$, il n'existe pas d'entier $m$ tel que $m^2+2*3^n=m(2^{n+1}-1)$. 
		\end{exo}
		
		\begin{sol}[1]
		On note que pour un entier $n$, $3^n$ est congru \`a $1$ ou \`a $3$ modulo $8$, donc l'expression est congrue modulo $8$ \`a $3,5$ ou $7$. Or, les carr\'es sont congrus modulo $8$ \`a $0,1,4$, ce qui conclut.
\end{sol}

\begin{sol}[2]
		Si $m,n$ sont tels que $3^m-2^n \in \{-1;5;7\}$, pour $n \geq 3$, en passant modulo $8$, on trouve que $3^m$ vaut $7$ ou $5$ modulo $8$, ce qui est impossible (voir ci-dessus). Pour $n=0$, il n'y a pas de solution. Pour $n=1$, $m=0$ convient pour $-1$, $m=2$ pour $7$. Pour $n=2$, $m=2$ convient pour $5$, $m=1$ pour $-1$. 
\end{sol}

\begin{sol}[3]
		Si $p \neq 3$, $p$ est congru \`a $1$ ou $2$ modulo $3$, donc $p^2+2$ vaut $0$ modulo $3$. Comme c'est un nombre premier, $p^2+2=3$, ce qui donne $p=\pm 1$, absurde. Donc $p=3$. On v\'erifie alors que $p^3+2=29$, $p^3+10=37$, $P^4+2=83$, $p^4-2=79$ sont premiers.  
\end{sol}

\begin{sol}[4]
		Si $2^n+1=a^2$, avec $a \geq 0$, on v\'erifie ais\'ement que l'on n'a pas $n=0$ et donc $a$ impair. On a alors $(a+1)(a-1)=2^n$, donc $a+1$ et $a-1$ sont deux puissances de $2$. Comme $a-1 < a+1$, $a-1|a+1$, et donc $a-1|2=(a+1)-(a-1)$, donc $a$ vaut $2$ ou $3$. On v\'erifie que la seule solution advient lorsque $n=3$, ie pour $a=3$. \\
		Si $2^n+1=a^3, a\in \Z$, on a $a > 0$; donc $2^n=a^3-1=(a-1)(a^2+a+1)$, donc $a^2+a+1$ est une puissance de $2$. Or, que $a$ soit pair ou impair, ce nombre est impair, donc $a^2+a+1=1$ et $a=0$, contradiction. Il n'y a pas de solution.
\end{sol}

\begin{sol}[5]
		Soient $x,y \in \N^*$ tels que $x(x+1)=4y(y+1)$: on a $x^2+x+1=4y(y+1)+1=(2y+1)^2$. Or, on a $x^2 < x^2+x+1 < x^2+2x+1=(x+1)^2$, donc $x^2+x+1$ n'est pas un carr\'e. Contradiction.
\end{sol}

\begin{sol}[6]
		Cet exercice provient de la muraille ; il n'est donc pas possible d'en donner un corrig\'e. 
\end{sol}

\begin{sol}[7]
		Si $n \geq 2$ est non premier, il poss\`ede un diviseur premier $d < n$, et $d|(n-1)!$ et $d|n$, donc $(n-1)!$ et $n$ sont non premiers entre eux, d'o\`u $n$ ne peut diviser $(n-1)!+1$. \\
		Si $n$ est premier, pour $1 \leq a \leq n-1$, si $a$ \'egale son inverse modulo $n$ (car $n$ \'etant premier, $a$ et $n$ sont premiers entre eux), $n$ divise $a^2-1=(a-1)(a+1)$ et $n$ divise donc $a-1$ ou $a+1$ (car n est premier, il est essentiel de comprendre pourquoi cette hypoth\`ese est indispensable), donc $a$ vaut $1$ ou $n-1$ (r\'eciproque imm\'ediate). Par cons\'equent, dans $\prod \limits_{i=1}^{n-1}{i}$, regard\'e modulo $n$, on peut regrouper les termes avec leur inverse, pour obtenir des $1$, et il reste $1$ et $n-1$, ce qui donne un total de $-1$ modulo $n$.\\
		Une autre preuve est possible, que je trouve plus jolie et plus f\'econde (mais plus compliqu\'ee, on peut l\'egitimement la sauter): on prend $p$ premier impair. Dans le cours de polyn\^omes, pour la construction de leur arithm\'etique, on a simplement utilis\'e le fait que dans $\R$ on peut additionner, soustraire, multiplier, diviser (sauf par $0$) dans l'ordre qui nous chante (on dit que $\R$ est un corps); mais, apr\`es tout, la m\^eme chose est vraie dans $\Z/p\Z$: on peut donc d\'efinir des polyn\^omes modulo $p$ (formels) qui ont les m\^emes propri\'et\'es que les polyn\^omes r\'eels. Consid\'erons alors le polyn\^ome "modulo $p$" $X^{p-1}-1-(X-1)(X-2)\ldots(X-(p-1))$. Ce polyn\^ome est de degr\'e au plus $p-2$ (les termes en $X^{p-1}$ s'annulent) et il a $p-1$ racines (les \'el\'ements de $(\Z/p\Z)^*$), donc il est nul, et on a formellement modulo $p$ $X^{p-1}-1=(X-1)\ldots(X-(p-1)$. On obtient l'\'egalit\'e souhait\'ee en \'evaluant en $0$ et en se rappelant que $(-1)^{p-1}=1$. \\
		Le reste est trop proche d'un exercice de la muraille pour \^etre corrig\'e ici.
\end{sol}

\begin{sol}[8]
		Il "suffit" de "remarquer" que $X(X+1)(X+2)(X+3)=(X^2+3X+1)^2-1$. 
\end{sol}

\begin{sol}[9]
		Il "suffit" de "remarquer" que $(a^2+b^2)(c^2+d^2)=(ac-bd)^2+(ad+bc)^2$ (formule de Lagrange). On peut l'interpr\'eter comme la multiplicativit\'e du module dans le monde des nombres complexes. 
\end{sol}

\begin{sol}[10]
		Les plus courageux des lecteurs mettront les nombres en base $2$: $641=2^9+2^7+1$. 
		On a $641|2^32+1 \Leftrightarrow 641|2^32+1-2^9-2^7-1 \Leftrightarrow 641|2^25-2^2-1$ (Gauss),\\ ce qui \'equivaut \`a $641|2^25-2^2-1+1+2^7+2^9 \Leftrightarrow 641|2^23+2^5+2^7-1$, etc en faisant diminuer les puissances de $2$ par lemme de Gauss. Le proc\'ed\'e est laborieux mais fonctionne. La preuve suivante est due \`a Euler: $641=2^4+5^4$ donc divise $2^28*5^4+2^28*2^4=2^32+5^4 2^28$; de plus, $641=5*2^7+1$ divise $(5*2^7)^4-1=5^4 2^28-1$ et en soustrayant la seconde de la premi\`ere on trouve le r\'esultat voulu.\\
		On note que $421=20^2+20+1$ et $1280000401=20^7+20^2+1$. Or, \\ $X^7+X^2+1=X(X^6-1)+X^2+X+1=X(X^3+1)(X-1)(X^2+X+1)+X^2+X+1=(X^2+X+1)(X(X-1)(X^3+1)+1)$. On \'evalue en $20$. 		
\end{sol}

\begin{sol}[11]
		On note que pour que $n^n+1$ soit inf\'erieur \`a $10^{18}$, il faut $n < 16$ (car $16^{16} + 1 > 10^{16}*(3/2)^{16} > 10^{16}* 2^8 > 10^{16}*100=10^{18}$). 
		$n=1$ convient. Si $n > 1$ poss\`ede un diviseur impair $d > 1$, $n^{n/d}+1|(n^{n/d})^n-(-1)^d=n^n+1$ et $1 < n^{n/d}+1 < n^n+1$, cela ne marche pas. Donc $n=2^k$, pour $k \geq 1$. Si $k$ poss\`ede un diviseur impair $d > 1$, $2^(nk/d)+1$ divise $(2^(nk/d))^d+1=n^n+1$, et est strictement compris entre $1$ et $n^n+1$. Donc $n=2^(2^p)$. Il reste \`a v\'erifier les cas $n=2$, $n=4$, qui marchent. 
\end{sol}

\begin{sol}[12]
		Modulo $a-b$, $\frac{a^n-b^n}{a-b}=\sum \limits_{k=0}^{n-1}{a^k b^{n-1-k}} \equiv \sum \limits_{k=0}^{n-1}{a^k a^{n-1-k}}=na^{n-1}$. Donc $pgcd\left(\frac{a^n-b^n}{a-b},a-b\right)=pgcd(na^{n-1},a-b)$. Soit $d$ divisant $na^{n-1}$ et $a-b$, $a$ et $b$ \'etant premiers entre eux, il en est de m\^eme ppour $a$ et $a-b$ donc $d$ est premier avec $a$ donc avec $a^{n-1}$, et par lemme de Gauss $d|n$ et $d|a-b$. La r\'eciproque est claire, ce qui prouve $pgcd(na^{n-1},a-b)=pgcd(n,a-b)$, ce qui fournit le r\'esultat.
\end{sol}

\begin{sol}[13]
		Soit $d$ un diviseur commun \`a $a^m-b^m$ et \`a $a^n-b^n$. A l'\'evidence, $d$ est premier avec $a$ et avec $b$. On peut donc envisager dans $\Z/d\Z$ des puissances n\'egatives de $a$ et de $b$. Soient $u,v \in \Z$ tels que $nu+mv=m\wedge n$. On a $a^n \equiv b^n [d]$ donc $a^{nu} \equiv b^{nu} [d]$; on a $a^m \equiv b^m[d]$ donc $a^{mv} \equiv b^{mv}[d]$, et en cons\'equence $a^{m\wedge n}=a^{mv+nu} \equiv a^{mv}a^{nu} \equiv b^{mv}b^{nu}\equiv b^{mv+nu}=a^{m \wedge n}$ (modulo $d$), et $d | a^{m \wedge n}-b^{m\wedge n}$.\\
		R\'eciproquement, $a^{m \wedge n}-b^{m\wedge n}$ divise $a^m-b^m$ et $a^n-b^n$. Le pgcd cherch\'e est donc $|a^{m \wedge n}-b^{m \wedge n}|$.
\end{sol}

\begin{sol}[14]
		A l'\'evidence, ce nombre $N$ est $\min(v_2(1000!),v_5(1000!))$. Or, la bien connue formule de Legendre stipule que pour $p$ premier, et $n$ entier, \[v_p(n!)=\sum \limits_{i=1}^{+\infty} \left\lfloor \frac{n}{p^i}\right\rfloor,\] la somme \'etant nulle \`a partir d'un certain rang (par exemple, pour $i \geq n$).
		Ainsi, $v_p(n!)$ d\'ecro\^it quand $p$ cro\^it, et $N=v_5(1000!)=\left\lfloor \frac{1000}{5}\right\rfloor+\left\lfloor \frac{1000}{25}\right\rfloor + \left\lfloor \frac{1000}{125}\right\rfloor + \left\lfloor \frac{1000}{625} \right\rfloor$, les termes suivants \'etant nuls. Donc $N=200+40+8+1=249$.
\end{sol}

\begin{sol}[15]
		Il s'agit encore d'une application de la formule de Legendre: pour $n \in \N^*$, 
		\[v_2(n!)=\sum_{i=1}^{+\infty}{\left\lfloor \frac{n}{2^i}\right\rfloor} < \sum_{i=1}^{+\infty}{\frac{n}{2^i}} = n,\]
		l'in\'egalit\'e se faisant terme \`a terme, et stricte car certains termes \`a gauche sont nuls alors qu'il n'y en a pas \`a droite, ce qui conclut. \\
		Si $n=2^p$, on a \[v_2(n!)=\sum_{i=1}^{+\infty}{\left\lfloor \frac{n}{2^i} \right\rfloor}=\sum_{i=1}^{p}{\frac{n}{2^i}}=\sum_{i=0}^{p-1}{2^i}=2^p-1=n-1, \]
		et donc $2^{n-1} | n!$, d'o\`u le r\'esultat.
\end{sol}

\begin{sol}[16]
		Il s'agit d'un exercice difficile. Si on se contente d'appliquer que $p_n \leq p_{n-1}+p_{n-2}+2016$, il n'est pas de voie vers la preuve. Soit $n \geq 2$, distinguons trois cas: si $p_{n-1}$ et $p_{n-2}$ sont pairs, $p_n$ est inf\'erieur \`a $2020$. Si un seul des deux est pair, $p_n$ est inf\'erieur \`a l'autre (ie le maximum des deux) plus $2018$. S'ils sont tous deux impairs, comme $p_{n-1}+p_{n-2}+2016 > 2$ est pair, $p_n \leq \frac{p_{n-1}+p_{n-2}+2016}{2} \leq \max(p_{n-1},p_n)+1008$. On voit donc qu'il est plus commode de raisonner avec $b_n=\max(p_n,p_{n-1})$. \\
		On dispose alors des propri\'et\'es suivantes sur $b_n$: pour $n \geq 2$, $b_n$ est premier et $b_n$ est inf\'erieur \`a $b_{n-1}+2018$. \\
		Supposons qu'on dispose d\'ej\`a d'une constante $M$ telle que $b_{n-1} \leq M$; alors $b_n$ est un premier inf\'erieur \`a $M+2018$. Pour prouver que $b_n \leq M$, il suffit de montrer qu'entre $M+1$ et $M+2018$ tous les entiers sont compos\'es. Or, le lecteur v\'erifiera que pour tout entier $a \geq 1$, $M=a*2019!+1$ satisfait cette propri\'et\'e. \\
		Par cons\'equent, avec un $a$ tel que $M=a*2019!+1 \geq b_1$, on a par r\'ecurrence $b_n \leq M$, et ainsi $p_n \leq M$ pour tout $n$.
		
\end{sol}

\begin{sol}[17]
		$n=1$ convient clairement, montrons que c'est le seul. Soit $n \geq 2$ divisant $2^n-1$, clairement $n$ est impair. Soit $p$ un diviseur premier de $n$, on a $2^n \equiv 1 [n]$ et $2^{p-1} \equiv 1 [n]$, donc $2^{n \wedge (p-1)} \equiv 1 [n]$. On aimerait avoir $n \wedge (p-1)=1$, c'est-\`a-dire que $n$ n'admette pas de diviseur premier divisant $p-1$. Il suffit pour cela que $n$ n'admette pas de diviseur premier inf\'erieur strictement \`a $p$, ie que $p$ soit le plus petit diviseur premier de $n$. Dans ce cas, on a bien $n \wedge (p-1)=1$ et $2^1 \equiv 1 [n]$, et donc $n$ divise $1$, contradiction. 
\end{sol}

\begin{sol}[18]
		On va montrer que seul $1$ convient. Pour cela, il suffit d'\'etablir que pour tout premier $p$, il existe un entier $m$ tel que $p | 2^m+3^m+6^m-1$. Pour $p=2$, tout $m$ convient. Pour $p=3$, tout $m$ pair (par exemple $2$) convient. Pour $p \geq 5$, $p$ est premier avec $2$,$3$,$6$. On laissera le soin au lecteur d'\'etudier les cas $p=5$, $p=7$, $p=11$ (en calculant les r\'esidus de $2^m+3^m+6^m-1$ modulo $p$) et remarquer qu'on a dans ces cas $p|2^{p-2}+3^{p-2}+6^{p-2}-1$. Montrons que ce r\'esultat est g\'en\'eral. \\
		Pour $p$ premier sup\'erieur \`a $5$, $2^{p-2}+3^{p-2}+6^{p-2} \equiv 6^{p-2}*3*2*2^{p-2}+6^{p-2}*2*3*3^{p-2} + 6^{p-2} [p]$, d'apr\`es le petit th\'eor\`eme de Fermat, et le second membre est congru modulo $p$ (toujours le petit th\'eor\`eme de Fermat) \`a $6^{p-2}(3*1+2*1+1) \equiv 6*6^{p-2} \equiv 1 [p]$, ce qu'on voulait d\'emontrer et qui cnclut la preuve.
\end{sol}

\begin{sol}[19]
		Soit $x=a^{\frac{p-1}{2}}$. On a $a^{p-1}-1$ divisible par $p$ donc $p$ divise $x^2-1=(x-1)(x+1)$ donc $p$ divise $x-1$ ou $x+1$, ce qui conclut. 
\end{sol}

\begin{sol}[20]
		Les puissances cinqui\`emes modulo $11$ \'etant $\pm 1,0$, pour un entier $n$, $n^5-4$ est congru modulo $11$ \`a $6,7,8$. Or, le lecteur v\'erifiera que les carr\'es modulo $11$ est $0,1,4,9,5,3$. 
\end{sol}

\begin{sol}[21]
		Soit $(x,y,z)$ triplet rationnel tel que $11=x^5+2y^5+5z^5$. Parmi les entiers positifs $d$ tels que $dx,dy,dz$ soient entiers, on en choisit un minimal. On \'ecrit $dx=a,dy=b,dx=c$. On a d\`es lors $11d^5=a^5+2b^5+5c^5$. En \'etudiant l'\'equation selon les restes de $a^5,b^5,c^5$ modulo $11$, le lecteur trouvera que $a,b,c$ sont divisibles par $11$ et donc $11d^5$ divisible par $11^5$, d'o\`u $11|d^4$ et $11|d$, et $0 < \frac{d}{11} < d$ v\'erifie la propri\'et\'e d\'efinissant $d$, ce qui contredit la minimalit\'e de $d$. Il n'y a donc aucune solution.
\end{sol}

\begin{sol}[22]
		Soit $(x,y,z)$ un triplet solution. On va montrer par r\'ecurrence sur $n \geq 1$ que $x,y,z$ sont divisibles par $2^n$, en rappelant le r\'esultat suivant (issu d'une \'etude des cas possibles modulo $4$): si une somme de trois carr\'es est divisible par $4$, les trois nombres sont pairs. Pour $n=1$, $x^2+y^2+z^2=2xyz$ est pair, donc l'un des $x^2,y^2,z^2$ est pair, donc de $x,y,z$ l'un est pair, donc $x^2+y^2+z^2=2xyz$ est divisible par $4$ et $x,y,z$ divisibles par $2=2^1$. \\
		Si $x,y,z$ divisibles par $2^n$ pour un $n \geq 1$, on note $x=2^nx',y=2^ny',z=2^nz'$. On a alors, en injectant dans l'\'equation et en simplifiant par $4^n$, $x'^2+y'^2+z'^2=2^{n+1}x'y'z'$ divisible par $4$ donc, $x',y',z'$ pairs et $x,y,z$ divisibles par $2^{n+1}$.  
\end{sol}

\begin{sol}[23]
		Comme $p,q$ sont premiers entre eux, (preuve laiss\'ee au lecteur) les nombres $0p, 1p, 2p, \ldots, (q-1)p$ couvrent chacun des restes diff\'erents modulo $q$, donc comme il y en a $q$, donc autant que de restes modulo $q$, chaque reste est atteint une unique fois. Il s'ensuit que la somme $S$ (membre de gauche) vaut:
			\[S=\sum_{k=0}^{q-1}{\frac{kp}{q}}-\sum_{k=0}^{q-1}{k/q}=\frac{pq(q-1)}{2q}-\frac{q(q-1)}{2q}=\frac{(p-1)(q-1)}{2}\]
\end{sol}

\begin{sol}[24]
		Soit un entier impair $a$ tel que $a^2|2^a+3^a$. Soit $p$ un premier divisant $N=\frac{2^a+3^a}{a}$ (divisible par $a$, et $p$ ne vaut ni $2$ ni $3$). 
		Par LTE (qu'on a le droit d'appliquer) on a $v_p(2^N+3^N)= v_p(2^a+3^a)+v_p(\frac{N}{a})=v_p(aN\frac{N}{a})=v_p(N^2)$. Donc $N$ convient. Il reste \`a prouver que $N > a$ et qu'un tel $a$ existe. Pour le premier point, le lecteur prouvera par r\'ecurrence que pour $n \geq 1$, $n^2 < 2^n+3^n$. Pour le second, on prend $a=1$. 
\end{sol}

\begin{sol}[25]
		Soient $m,n$ tels que $7^m-3*2^n=1$. On a donc $3*2^n=7^m-1$. $n=0$ ne donne pas de solution, donc $n \geq 1$. On a alors par une factorisation classique $2^{n-1}=1+7+\ldots+7^{m-1}$. $n=1$ fournit une solution si et seulement si $m=1$, on suppose $n \geq 2$. Alors la somme de droite est paire, donc compte un nombre pair de termes. Aussi, $m$ est pair. La somme de droite se factorise alors par $(1+7)$ (on regroupe ensemble les termes $7^{2k}$ et $7^{2k+1}$, $k \in \N$), ce qui prouve $n-1 \geq 3$ et en divisant par $8$ des deux c\^ot\'es on trouve $2^{n-4}=1+7^2+7^4+\ldots+7^{m-2}$. 
		$n=4$ donne une solution si et seulement si $m=2$, et pour $n \geq 5$, il y a un nombre pair de termes dans le membre de droite.  Similairement \`a ce qui pr\'ec\`ede, on peut factoriser \`a droite par $1+7^2$ qui est divisible par $5$: or $5$ ne peut diviser le membre de gauche, contradiction. Les seules solutions sont alors $n=m=1,n=2,m=4$.\\
		Un autre argument est possible. On traite au cas par cas la possibilit\'e $m < 5$ et pour $m \geq 5$, on constate que $7$ est d'ordre $4$ modulo $32$ donc que pour une solution $(m \geq 5,n)$, $4|n$ donc $5|7^2+1|7^4-1|7^m-1=2^n*3$. 
\end{sol}

\begin{sol}[26]
		$n=m=1$, $m=2,n=3$ donnent des solutions. On prend dor\'enavant $n \geq 4$. En passant modulo $8$ on trouve que $m$ est pair, et on a $2^n+1=(3^{m/2})^2$. On montre alors avec l'exercice $4$ qu'il n'y a pas d'autre solution.
\end{sol}

\begin{sol}[27]
		Avant tout, on note qu'on peut diviser simultan\'ement $a$ et $b$ autant de fois qu'on veut par $19$. On a alors $a$ et $b$ premiers entre eux (si $q$ premier divise $a$ et $b$, $q \neq 19$ et $q^2$ divise $(a+bc)(b+ac)=19^n$, ce qui est impossible). Supposons $a \leq b$. Alors $a+bc \geq b+ac$; comme ces deux nombres ont pour produit une puissance de $19$, ce sont des puissances de $19$ et $b+ac|a+bc$, donc $b+ac|a+bc-b-ac=(b-a)(c-1)$, $b+ac|a+bc+b+ac=(b+a)(c+1)$ Comme $b-a$ et $b+a$ ne sont pas simultan\'ement divisibles par $19$, sinon $a$ et $b$ le sont, $b+ac$ divise $c+1$ ou $c-1$ et donc $b+ac \leq c+1$ dans tous les cas. Cela force l'un des cas suivants: $a=0$, $a=1,b \leq 1$, $a=2, c=1, b=0$. Le troisi\`eme cas est impossible comme le premier, dans le deuxi\`eme cas, on a $a=b$ et la conclusion est imm\'ediate.
\end{sol}

\begin{sol}[28]
		Indication: remarquer que $1 < \frac{abc-1}{(a-1)(b-1)(c-1)} < \frac{a}{a-1}\frac{b}{b-1}\frac{c}{c-1} < (\frac{a}{a-1})^3$, donc qu'on doit avoir $(\frac{a}{a-1})^3 \geq 2$ ce qui force $a < 5$. Etudier les cas.
\end{sol}

Les probl\`emes suivants rel\`event de listes courtes IMO ou de probl\`emes d'IMO. Les solutions compl\`etes sont disponibles sur Internet, on ne donnera que des indications. Dans les deux \'equations diophantiennes, il s'agit beaucoup de triturer l'\'equation. 

\begin{sol}[29]
		Factoriser $y^2-1$, remarquer qu'exactement un des facteurs est divisible par $4$, \'ecrire la factorisation et noter que les deux facteurs sont distants de $2$.
\end{sol}

\begin{sol}[30]
		Penser \`a la formule de Legendre.
\end{sol}

\begin{sol}[31]
		Voir l'\'equation comme du second degr\'e en $m$, prendre les deux solutions et exploiter les relations coefficients-racines. Ajouter une pinc\'ee de LTE et touiller.
\end{sol}
