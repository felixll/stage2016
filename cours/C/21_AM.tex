\ifcourt
%% Cours de géométrie groupe C

\paragraph{Principe d'étude} Dans ce cours, on s'intéresse à différentes transformations géométriques qui envoient
les points sur des points, les droites sur des droites, les cercles sur des cercles. Le plan
d'étude est le suivant :
\begin{enumerate}
\item On considère une grandeur qu'on aimerait voir conservée par notre transformation 
(les longueurs, les rapports de longueur, les angles géométriques, les angles orientés) ;
\item On définit les transformations conservant cette grandeur, on étudie leur structure :
\begin{itemize}
\item[$\bullet$] centre d'une transformation : une telle transformation a-t-elle un point fixe ? si oui, comment le construire ? 
\item[$\bullet$] \og degrés de liberté\fg : peut-on remonter de la donnée
d'un certain nombre de points et de leurs images à la définition d'une transformation ?
inversement, pour quels $n$ peut-on fixer $A_1,\ldots,A_n$ et $B_1,\ldots,B_n$ tels qu'il
existe une transformation adéquate qui envoie $A_1$ sur $B_1$,\ldots,$A_n$ sur $B_n$ ?
(et quelles contraintes supplémentaires faut-il éventuellement imposer sur ces points pour
pouvoir augmenter le degré de liberté $n$ ?)
\item[$\bullet$] Composition de deux transformations d'un même type ;
\item[$\bullet$] Décomposition d'une transformation comme produit de facteurs \og élémentaires\fg.
\end{itemize}
\item Si on a deux types de transformations (celles qui conservent les schtroumpfs et celles
qui conservent les bagadas, par exemple), on étudie la position relative de l'ensemble
des transformations qui conservent les schtroumpfs et de l'ensemble des transformations qui 
conservent les bagadas, et la façon dont notre loi de composition se comporte par rapport 
à eux (voir le schéma infra).
\end{enumerate}

\paragraph{Points annexes traités/non traités} On a insisté sur la différence entre une 
transformation directe et une transformation indirecte (constat de
départ : la symétrie axiale est une isométrie, elle conserve les angles mais 
multiplie par $-1$ toutes les valeurs d'angles orientés), cette question jouant un
rôle important dans l'étude du nombre de degrés de liberté d'une transformation. 
On a étudié la notion de similitude intérieure. On a donné la construction du 
point de Miquel, dont on a ainsi pu déduire le théorème de concours des quatre 
cercles en le point de Miquel.

L'étude générale des symétries axiales (notamment leur composition)
n'a pas été menée. L'existence d'un centre pour toute similitude a été admise.

\paragraph{Schéma récapitulatif}


\paragraph{Exercices traités}
Outre l'étude des différentes transformations qu'on s'est efforcé de mener comme plusieurs
petits exercices, les exercices suivants on été posés :

\begin{exo}
Soit $ABCD$ un carré, $P$ à l'intérieur de $ABCD$ tel que $PA=1,PB=2,PC=3$.
Calculer $\widehat{APB}$.
\end{exo}

\begin{exo}
Un randonneur est dans une région rectangulaire $ABCD$ dont il possède une petite
carte $A'B'C'D'$. Quand il pose sa carte par terre, montrer qu'il existe un unique
point $X$ dont la position réelle et l'image sur la carte coïncident, et le déterminer
sa position à la règle.  
\end{exo}

\begin{sol}
Soit $\rho$ la rotation de centre $B$ et d'angle $90\degres$. Elle envoie $C$ sur $A$,
elle fixe $B$ et elle envoie $P$ sur un point $P'$ tel que $\widehat{PBP'}=90\degres$
et $P'B=PB=2$. On a de plus $P'A=PC=3$ (isométrie). D'après le théorème de Pythagore
dans $PBP'$, $PP'=2\sqrt{2}$, donc d'après la réciproque du théorème de Pytagore, 
le triangle $PAP'$ est rectangle en $P$. Comme $PBP'$ est rectangle isocèle en $B$,
on a finalement $\widehat{APB}=\widehat{APP'}+\widehat{PP'B}=90\degres+45\degres=135\degres$. 
\end{sol}

\begin{sol}
Comme $ABCD$ et $A'B'C'D'$ sont deux figures semblables, il existe une similitude qui envoie
l'une sur l'autre, et le point $X$ recherché est le centre (unique) de cette similitude.
On applique le leitmotiv :
$$\mbox{\og Quand je vois deux figures semblables, je pense à la similitude qui envoie l'une sur l'autre, et surtout je pense aux similitudes intétieures de mes figures.\fg}$$

On considère la similitude $s$ qui envoie $(AB)$ sur $(CD)$ et $(A'B')$ sur $(C'D')$. 
Elle est de centre $X$. Elle envoie $(AB)$ sur une parallèle à $(AB)$, c'est donc 
une similitude d'argument $0\degres$ ou $180\degres$, c'est-à-dire une 
homothétie. $s((AB)\cap (A'B'))=s((AB))\cap s((A'B'))= (CD)\cap (C'D')$ donc 
le centre de $s$ est sur la droite qui relie $(AB)\cap (A'B')$ et
$(CD)\cap (C'D')$. De même, $X$ est le centre de la similitude qui envoie $(AD)$ 
sur $(CB)$ et $(A'D')$ sur $(C'B')$, et il est sur la droite qui relie $(AD)\cap (A'D')$ et
$(CB)\cap (C'B')$. Donc $X$ est l'intersection de la droite qui relie $(AB)\cap (A'B')$ et
$(CD)\cap (C'D')$ et de la droite qui relie $(AD)\cap (A'D')$ et
$(CB)\cap (C'B')$. 
\end{sol}

\paragraph{Et ensuite ?}
Pour plus de précisions, la plupart des énoncés montrés se retrouvent dans l'excellent polycopié de
Thomas Budzinski sur les transformations (cf. cours de l'OFM), auquel on renvoie
également le lecteur désireux de plus d'exercices.
