Le cours a suivi le cours d'artihm\'etique qui se trouve sur le site d'Animath, parmi les cours de l'OFM (\url{http://www.animath.fr/IMG/pdf/cours-arith1.pdf}). Les points abord\'es ont \'et\'e les suivants (avec, entre parenth\`eses, la section correspondante du polycopi\'e du cours) :

\begin{itemize}
\item l'algorithme d'euclide (section 2.2) ;
\item le th\'eor\`eme de Bezout et sa preuve avec l'algorithme d'Euclide \'etendu (section 2.3) ;
\item le lemme de Gauss et ses cons\'equences (section 2.4) ;
\item les bases sur les congruences modulo, notamment la notion d'inverse (section 3.1) ;
\item la notion d'ordre (multiplicatif) d'un \'el\'ement (section 3.3) ;
\item les congruences modulo $p$ (section 3.5), avec notamment le petit th\'eor\`eme de Fermat et le th\'eor\`eme de Wilson ;
\item le th\'eor\`eme chinois des r\'esidus (section 3.4).
\end{itemize}

