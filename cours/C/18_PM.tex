
On suppose connues les in\'egalit\'es classiques suivantes : in\'egalit\'e arithm\'etico-g\'eom\'etrique (IAG), in\'egalit\'e de Jensen
et in\'egalit\'e de Cauchy-Schwarz.
De plus, on suppose connu que si $a_1,\ldots,a_n$ sont des nombres positifs, alors $k\mapsto \left(\dfrac{\sum_{i=1}^n a_i^k}{n}\right)^{1/k}$ est croissante.
Enfin, on suppose que le lecteur conna\^{\i}t les principes de base de la d\'erivation des fonctions.
Pour plus de d\'etails, on pourra consulter le cours d'alg\`ebre de l'OFM, ou bien le livre
``Analyse'' de Mohammed Aassila dont sont tir\'es un petit nombre d'exercices.

On utilisera des notations telles que $\sum_{cyc}ab^2=ab^2+bc^2+ca^2$ et $\sum ab^2=ab^2+ac^2+ba^2+bc^2+ca^2+cb^2$.

Si $a_1,\ldots,a_n$ sont des nombres r\'eels donn\'es, on notera \'egalement $S_k=a_1^k+\cdots+a_n^k$ et $\sigma_k=\sum_{i_1<\cdots<i_k}a_{i_1}\cdots a_{i_k}$.

\subsubsection{Exercices de base}

\begin{exo}
 D\'eterminer le minimum de l'expression $\frac{x}{\sqrt{1-x}}+\frac{y}{\sqrt{1-y}}$ o\`u $x$ et $y$ sont des r\'eels strictement positifs v\'erifiant $x+y=1$.
\end{exo}

\begin{exo}
 Soient $a,b,c$ les angles d'un triangle. Montrer que $\sin a+\sin b+\sin c\leqslant \frac{3\sqrt{3}}{2}$, $\cos\frac{a}{2}+\cos\frac{b}{2}+\cos\frac{c}{2}\leqslant \frac{3\sqrt{3}}{2}$, $\cos a\cos b\cos c\leqslant \frac{1}{8}$ et $\sin 2a+\sin 2b+\sin 2c\leqslant \sin a +\sin b+\sin c$.
\end{exo}

\begin{exo}
 Soient $a,b,c$ des nombres r\'eels strictement positifs. Montrer que $\frac{a}{b+c}+\frac{b}{c+a}+\frac{c}{a+b}\geqslant \frac{3}{2}$.
\end{exo}

\begin{exo}
Soit $n$ un entier. D\'eterminer la plus grande constante $C$ possible pour que pour tous $a_1\ldots,a_n\geqslant 0$ on ait $\sum a_i^2\geqslant C\sum_{i<j}a_ia_j$. 
\end{exo}


\begin{exo}
Soit $n$ un entier. D\'eterminer la plus grande constante $C$ possible pour que pour tous $a_1\ldots,a_n\geqslant 0$ on ait $(\sum a_i)^2\geqslant C\sum_{i<j}a_ia_j$. 
\end{exo}

\begin{exo} (non trait\'e en cours)
 D\'eterminer le minimum de $x^2+y^2+z^2$ sous la condition $x^3+y^3+z^3-3xyz=1$.
\end{exo}

\begin{exo}
 Soient $a_i$ et $A_i$ des r\'eels strictement positifs. Montrer que $(\sum a_iA_i)(\sum \frac{a_i}{A_i})\geqslant (\sum a_i)^2$.
\end{exo}

\begin{exo}
 Soient $a_i$ et $b_i$ des r\'eels strictement positifs. Montrer que $\sum\frac{a_i^2}{b_i}\geqslant \frac{(\sum a_i)^2}{\sum b_i}$.
\end{exo}


\subsubsection{Exercices moins faciles}

\begin{exo}
 Soient $a,b,c$ des r\'eels strictement positifs tels que $a+b+c=3$. Montrer que $\frac{a^2}{a+b}+\frac{b^2}{b+c}+\frac{c^2}{c+a}\geqslant\frac{3}{2}$.
\end{exo}

% \begin{exo}
%  Montrer que pour tous $x,y,z\geqslant 0$ on a $\sum_{cyc}\sqrt{x^2+xy+2y^2}\geqslant 2(x+y+z)$.
% \end{exo}

\begin{exo}
 Montrer que si $a,b,c$ sont des r\'eels strictment positifs tels que $ab+bc+ca=abc$ alors $\sum_{cyc}\frac{a^4+b^4}{ab(a^3+b^3)}\geqslant 1$.
\end{exo}

\begin{exo}
 Soit $k\geqslant 0$ donn\'e. D\'eterminer le minimum de l'expression $A=\frac{x}{1+y^2}+\frac{y}{1+x^2}+\frac{z}{1+t^2}+\frac{t}{1+z^2}$ sous
la condition $x+y+z+t=k$ et $x,y,z,t$ positifs.
\end{exo}

\begin{exo}
 Calculer le minimum de $\sqrt{(x-1)^2+y^2}+\sqrt{(x+1)^2+y^2}+|2-y|$ pour $x,y\in \R$.
\end{exo}

\begin{exo}
 Soient $a,b>0$. Montrer que $\frac{x}{ay+bz}+\frac{y}{az+bx}+\frac{z}{ax+by}\geqslant \frac{3}{a+b}$.
\end{exo}


\begin{exo} (non trait\'e en cours)
 D\'eterminer le minimum de l'expression
$$\frac{2}{|a-b|}+\frac{2}{|b-c|}+\frac{2}{|c-a|}+\frac{5}{\sqrt{ab+bc+ca}}$$
sous les conditions que $ab+bc+ca>0$, $a+b+c=1$ et $a,b,c$ distincts.
\end{exo}

\begin{exo}  (non trait\'e en cours)
 Montrer que si $a,b,c> 0$ sont des nombres r\'eels tels que $abc=1$ alors $(a+b)(b+c)(c+a)\geqslant 4(a+b+c-1)$.
\end{exo}


\subsubsection{Solutions abr\'eg\'ees des exercices de base}

\begin{sol}
 Soit $f(x)=\frac{x}{\sqrt{1-x}}$, alors on v\'erifie que $f''(x)=\frac{4-x}{4(1-x)^{5/2}}$ donc $f$ est convexe sur $]0,1[$. L'in\'egalit\'e de Jensen permet de conclure.
\end{sol}

\begin{sol}
Pour les trois premi\`eres in\'egalit\'es, d\'emontrer et utiliser la concavit\'e des fonctions $\sin$ sur $[0,\pi]$, $\cos$ sur $[0,\frac{\pi}{2}]$ et $\ln\cos$ sur $[0,\frac{\pi}{2}[$ (traiter s\'epar\'ement le cas o\`u le triangle n'est pas acutangle). Pour la troisi\`eme in\'egalit\'e on peut aussi utiliser l'IAG puis la concavit\'e du cosinus. Pour la quatri\`eme, d\'emontrer en utilisant les identit\'es $\sin x+\sin y=2\sin\frac{x+y}{2}\cos\frac{x-y}{2}$ et $\sin 2x=2\sin x\cos x$ que $\sin a+\sin b+\sin c=4\cos\frac{a}{2}\cos\frac{b}{2}\cos\frac{c}{2}$ et que $\sin 2a+\sin 2b+\sin 2c=4\sin a\sin b\sin c=32 \sin\frac{a}{2}\sin\frac{b}{2}\sin\frac{c}{2}\cos\frac{a}{2}\cos\frac{b}{2}\cos\frac{c}{2}$, puis utiliser l'in\'egalit\'e de convexit\'e $\sin\frac{a}{2}\sin\frac{b}{2}\sin\frac{c}{2}\leqslant\frac{1}{8}$.
\end{sol}

\begin{sol}
 On peut supposer $a+b+c=1$ et utiliser la convexit\'e de $x\mapsto \frac{x}{1-x}$, ou bien l'IAG :
$\sum(\frac{a}{b+c}+1)=(a+b+c)\sum\frac{1}{b+c}=\frac{1}{2}(\sum (b+c))\sum\frac{1}{b+c}\geqslant \frac{9}{2}$.
\end{sol}

\begin{sol}
L'in\'egalit\'e entre moyenne arithm\'etique et moyenne quadratique s'\'ecrit $(a_1+\cdots+a_n)^2\leqslant n (a_1^2+\cdots+a_n^2)$,
ou encore $\sum a_i^2+2\sum_{i<j}a_ia_j \leqslant n\sum a_i^2$. On a donc $C=\frac{2}{n-1}$.
\end{sol}

\begin{sol}
On ajoute $2\sum_{i<j}a_ia_j$ \`a l'in\'egalit\'e pr\'ec\'edente, on trouve $C=\frac{2n}{n-1}$.
\end{sol}

\begin{sol}
 On calcule que $S_3=\sigma_1S_2-\sigma_1\sigma_2+3\sigma_3$ donc $S_3-3\sigma_3=\sigma_1(S_2-\frac{3S_2-\sigma_1^2}{2})$.
On doit donc minimiser $S_2$ sous la condition que $\sigma_1(3S_2-\sigma_1^2)=2$.

Or, $3S_2=\frac{2}{\sigma_1}+\sigma_1^2\geqslant 3$ par l'IAG (ou par une \'etude de fonction), donc $S_2\geqslant 1$. La borne est atteinte lorsque
$\sigma_1=1$ (par exemple lorsque $x=1$ et $y=z=0$).
\end{sol}

\begin{sol}
 Cauchy-Schwarz avec $x_i=\sqrt{a_iA_i}$ et $y_i=\sqrt{a_i/A_i}$.
\end{sol}

\begin{sol}
 Cauchy-Schwarz avec $x_i=a_i/\sqrt{b_i}$ et $y_i=\sqrt{b_i}$.
\end{sol}

\subsubsection{Solutions abr\'eg\'ees des exercices moins faciles}
\begin{sol}
On utilise l'exercice pr\'ec\'edent. Le membre de gauche est sup\'erieur ou \'egal \`a $\dfrac{(a+b+c)^2}{(a+b)+(b+c)+(c+a)}=\frac{3}{2}$.
%  On cherche $u$ et $v$ tels que $\frac{a^2}{a+b}\geqslant ua+vb$ et on sommera les trois in\'egalit\'es analogues.
% Pour que la majoration soit utile, on prend $v=\frac{1}{2}-u$.
% Par homog\'en\'eit\'e, l'in\'egalit\'e devient $\frac{t^2}{t+1}\geqslant ut+(1-u)$.
% 
% Soit $f(t)=\frac{t^2}{t+1}$. G\'eom\'etriquement, cela signifie que le graphe de $f$ se situe au-dessus de la droite $t\mapsto ut+(1-u)$. De plus,
% cette droite touche le graphe de $f$ au point $t=1$. Par cons\'equent, si une telle \'egalit\'e est vraie alors la droite est la tangente en $t=1$,
% et donc $u=f'(1)$. On calcule que $f'(1)=\frac{3}{4}$. Il suffit donc de montrer que $t^2-(t+1)(\frac{3}{4}t+\frac{1}{4})\geqslant 0$,
% ce qui est vrai car l'in\'egalit\'e s'\'ecrit $(t-1)^2/4\geqslant 0$. (On peut aussi raisonner par convexit\'e de $f$.)
\end{sol}

% \begin{sol}
%  On cherche $u$ et $v$ tels que $\sqrt{x^2+xy+2y^2}\geqslant ux+vy$ de sorte que $u+v=2$.
% Soit $f(t)=\sqrt{t^2+t+2}$. G\'eom\'etriquement, cela signifie que le graphe de $f$ se situe au-dessus de la droite $t\mapsto ut+(1-u)$. De plus,
%cette droite touche le graphe de $f$ au point $t=1$. Par cons\'equent, si une telle \'egalit\'e est vraie alors la droite est la tangente en $t=1$,
%et donc $u=f'(1)$. On calcule que $f'(1)=\frac{3}{4}$, donc il suffit de montrer que $\sqrt{t^2+t+2}\geqslant (3t+5)/4$, i.e.
% $16(t^2+t+2)\geqslant 9t^2+30t+25$, ou encore $7t^2-14t+7\geqslant 0$, ce qui est vrai.
% \end{sol}

\begin{sol}
 Posons $x=1/a$, etc. On a $\sum x=1$ et on veut montrer que $\sum\frac{x^4+y^4}{x^3+y^3}\geqslant 1$. On cherche $a$ tel que $\frac{x^4+y^4}{x^3+y^3}\geqslant ax+(1-a)y$.
Pour des raisons de sym\'etrie, il est naturel d'essayer $a=1/2$. Il suffit alors de montrer que $t^4+1\geqslant(t^3+1)(t+1)/2$. Cela s'\'ecrit $(t-1)^2(t^2+t+1)/2\geqslant 0$, ce qui est vrai. (On peut aussi utiliser l'in\'egalit\'e de Tchebychev, qui donne $\frac{x^4+y^4}{2}\geqslant (\frac{x^3+y^3}{2})(\frac{x+y}{2})$.)
\end{sol}

\begin{sol}
 On applique $(\frac{x}{X}+\frac{y}{Y})(xX+yY)\geqslant (x+y)^2$ \`a $X=1+y^2$ et $Y=1+x^2$, ce qui donne $(\frac{x}{1+y^2}+\frac{y}{1+x^2})((x+y)(1+xy))\geqslant (x+y)^2$. Si $(x,y)\ne (0,0)$, ceci se simplifie en $\frac{x}{1+y^2}+\frac{y}{1+x^2}\geqslant \frac{x+y}{1+xy}$, in\'egalit\'e qui est toujours vraie pour $x=y=0$.
Comme $xy\leqslant (x+y)^2/4\leqslant k^2/4$, on en d\'eduit que $\frac{x}{1+y^2}+\frac{y}{1+x^2}\geqslant \frac{4(x+y)}{4+k^2}$, donc $A\geqslant \frac{4k}{4+k^2}$, borne qui est atteinte pour $x=y=k/2$ et $z=t=0$ par exemple.
\end{sol}

\begin{sol}
 Soient $A=(-1,0)$, $B=(1,0)$ et $D$ la droite $y=2$. On doit minimiser $MA+MB+d(M,D)$. Le point $M$ appartient n\'ecessairement \`a l'axe des ordonn\'ees. On se ram\`ene alors \`a minimiser $2\sqrt{1+y^2}+|2-y|$. On voit rapidement que $0\leqslant y\leqslant 2$. Une d\'erivation permet de montrer que le minimum est atteint en $y=1/\sqrt{3}$, et que ce minimum vaut $2+\sqrt{3}$.
\end{sol}

\begin{sol}
 \begin{eqnarray*}
  \sum_{cyc}\frac{x^2}{axy+bxz}&\geqslant& \frac{(\sum x)^2}{\sum_{cyc}(axy+bxz)}\\
&=& \frac{(\sum x)^2}{(a+b)\sum xy}\geqslant \frac{3}{a+b}.
 \end{eqnarray*}

(Voir exercices de base.)

\end{sol}

\begin{sol}
 Pour des raisons de sym\'etrie, on se ram\`ene au cas $a>b>c$. Alors l'expression vaut
\begin{eqnarray*}
 A &=& \frac{2}{a-b}+\frac{2}{b-c}+\frac{2}{c-a}+\frac{5}{\sqrt{ab+bc+ca}}\\
&\geqslant& \frac{10}{c-a}+\frac{5}{\sqrt{ab+bc+ca}}\quad \mbox{car } \frac{1}{m}+\frac{1}{n}\geqslant \frac{4}{m+n}\\
&=&10\left(\frac{1}{c-a}+\frac{1}{2{\sqrt{ab+bc+ca}}}\right)\\
&\geqslant& \frac{20\sqrt{2}}{\sqrt{(c-a)^2+4(ab+bc+ca)}}\quad \mbox{car }\frac{1}{m}+\frac{1}{n}\geqslant \frac{2\sqrt{2}}{\sqrt{m^2+n^2}}\\
&=& \frac{20\sqrt{2}}{\sqrt{{(a+c)^2+4b(a+c)}}}=\frac{20\sqrt{2}}{\sqrt{(1-b)(1+3b)}}.
\end{eqnarray*}

On minore cette expression par $10\sqrt{6}$, la borne \'etant atteinte pour $b=1/3$ et $a+c=2/3$ v\'erifiant $(c-a)^2=4(ab+bc+ca)$, i.e. $a=\frac{2+\sqrt{6}}{6}$ et $c=\frac{2-\sqrt{6}}{6}$.
\end{sol}

\begin{sol}
 Apr\`es simplification de l'in\'egalit\'e, on doit montrer que si $\sigma_3=1$ alors $\sigma_2+\frac{3}{\sigma_1}\geqslant 4$.

Comme $3(xy+yz+zx)\leqslant (x+y+z)^2$, en prenant $x=ab$, $y=bc$ et $z=ca$ on obtient que $\sigma_2^2\geqslant 3\sigma_1\sigma_3$ donc
$(\frac{\sigma_2}{3})^2\geqslant \frac{\sigma_1}{3}\geqslant 1$.
D'apr\`es l'IAG, on en d\'eduit que $\sigma_2+\frac{3}{\sigma_1}=3\frac{\sigma_2}{3}+\frac{3}{\sigma_1}\geqslant 4\sqrt[4]{(\frac{\sigma_ 2}{3})^3(\frac{3}{\sigma_1})}
\geqslant 4\sqrt[4]{(\frac{\sigma_ 2}{3})^2(\frac{3}{\sigma_1})}\geqslant 4$. 
\end{sol}
