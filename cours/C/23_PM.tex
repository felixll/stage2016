% largement inspiré de TD de Guillaume CK
\begin{exo}
Trouver les fonctions $f:\R \rightarrow \R$ telles que pour tous réels $x,y$, on ait
$$f(x-f(y))=1-x-y$$
\end{exo}
%Tst 2013, easy

\begin{exo}
Soit $f : \R \rightarrow \R$ vérifiant pour tout réel $x$ :
$$f(x+1)+f(x-1)= \sqrt{2}f(x),$$
montrer qu'elle est périodique. 
\end{exo}

\begin{exo}
Trouver les fonctions $f \colon \R \to \R$ continues telles que 
pour tous $x, y \in \R$, on ait :
\[f\left(\frac{x + y}{2}\right) = \frac{f(x) + f(y)}{2}\]
% Équation de Jensen
\end{exo}


\begin{exo}Trouver les fonctions $f \colon \Q_+^* \to \Q_+^*$ telles que pour tout $x \in  \Q_+^*$, on ait $f(x + 1) = f(x) + 1$ et $f(1/x) = 1/f(x)$.
\end{exo}



\begin{exo}Existe-t-il deux fonctions $f \colon \R \to \R$ et $g \colon \R \to \R$ telles que pour tout $x \in \R$ :

\[f(g(x)) = x^2 \text{ et } g(f(x)) = x^3\]
% proposition IMO
\end{exo}

\begin{exo}
Trouver toutes les fonctions $f : \N \rightarrow \N$ telles que pour tout naturel $n$,
$$f(n)+f(f(n))+f(f(f(n)))=3n.$$
\end{exo}

\begin{exo}
Trouver toutes les fonctions de $\R-\lbrace 0,1 \rbrace$ dans $\R$ telles que 
$$f(x)+f\left( \frac{1}{1-x}\right)=x.$$
\end{exo}


\begin{exo}(Arithmétique des fonctions)
Trouver toutes les fonctions $f : \N \rightarrow \N$ strictement croissantes telles que $f(2)=2$ et pour tous $m,n$ premiers entre eux, $f(mn)=f(m)f(n)$. 
\end{exo}


\begin{exo}Trouver toutes les fonctions $f \colon \R \to \R$ telles que pour tous $x, y \in \R$, on ait 
\[f(f(x) + y) = 2x + f(f(y) - x)\]
\end{exo}





\begin{exo}
Soit $f$ une fonction de $\N$ dans lui-même. Montrer que si pour tout naturel $n$, 
$$f(n+1)>f(f(n))$$
alors $f$ est l'identité.
% IMO 77
\end{exo}

\begin{exo}
Existe-t-il une fonction $f \colon \N \rightarrow \N$ telle que pour tout entier naturel $n$, 
$$f(f(n))=n+2015 \quad ?$$
% IMO 87 - réactualisé
\end{exo}
