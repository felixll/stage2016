\begin{exo}\\
Soient $p$ et $q$ deux nombres premiers impairs distincts. Déterminer l'ordre (multiplicatif) de $1+pq$ modulo $p^2 q^3$.
\end{exo}

\bigskip

\begin{exo}\\
Soient $a,b,c$ des nombres r\'eels positifs tels que $a+b+c=3$. Montrer que
$$\frac{a}{1+b^2c}+\frac{b}{1+c^2a}+\frac{c}{1+a^2b}\geqslant \frac{3}{2}.$$
\end{exo}

\bigskip

\begin{exo}\\
Déterminer toutes les solutions de l'équation
\[p^n=x^3+y^3\]
avec $p$ premier et $x,y,n \in \mathbb{N}^*$.
\end{exo}

\bigskip

\begin{exo}\\
Soit $P$ un polynôme de degré $n \geq 5$ avec $n$ racines entières distinctes tel que $P(0)=0$. Déterminer le nombre de racines entières de $P(P(X))$.
\end{exo}
