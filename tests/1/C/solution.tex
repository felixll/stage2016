\begin{sol}
On développe $(1+pq)^k$ avec le binôme de Newton. Tous les termes de degré au moins $3$ sont divisibles par $p^2 q^3$ donc
\[(1+pq)^k \equiv 1+kpq+\frac{k(k-1)}{2} p^2 q^2 \pmod{p^2 q^3}.\]
Si $(1+pq)^k \equiv 1 \pmod{p^2 q^3}$, alors $p^2 q^2$ divise $kpq+\frac{k(k-1)}{2} p^2 q^2$ donc il divise $k pq$, donc $pq$ divise $k$. Le terme $\frac{k(k-1)}{2} p^2 q^2$ est donc divisible par $p^2 q^3$ donc $p^2 q^3$ divise $kpq$ et $pq^2$ divise $k$. On vérifie réciproquement que si $pq^2$ divise $k$ alors $(1+pq)^k \equiv 1 \pmod{p^2 q^3}$, donc l'ordre recherché vaut $p q^2$.
\end{sol}

\begin{sol}
On utilise l'inégalité de Cauchy-Schwarz :
\begin{eqnarray*}
\frac{a}{1+b^2c}+\frac{b}{1+c^2a}+\frac{c}{1+a^2b} & \geqslant & \frac{(a+b+c)^2}{a(1+b^2 c)+b(1+c^2 a)+c(1+a^2 b)}\\
&=& \frac{(a+b+c)^2}{(a+b+c)(1+abc)}\\
&=& \frac{3}{1+abc}\\
& \geqslant & \frac{3}{2}
\end{eqnarray*}
où la dernière inégalité vient du fait que $abc \leqslant 1$ par IAG.
\end{sol}

\begin{sol}
Quitte à diviser $x$ et $y$ par une puissance de $p$ adéquate, on peut supposer que $x$ ou $y$ n'est pas divisible par $p$. Comme $p$ doit diviser $x^3+y^3$, si $p$ ne divise pas l'un des deux il ne divise pas l'autre. On commence par traiter à part le cas $p=2$ : on a $2^n=(x+y)(x^2-xy+y^2)$ mais comme $x$ et $y$ sont impairs, le deuxième facteur l'est aussi donc il vaut $1$, soit $(x-y)^2+xy=1$ donc $x=y=1$ et $n=1$. En multipliant $x$ et $y$ par une même puissance de deux, on obtient $x=y=2^k$ et $n=3k+1$ avec $k \geq 0$.

Si $p$ est impair, on utilise le LTE : on a $n=v_p(x^3+y^3)=v_p(x+y)+v_p(3)$. Si $p \ne 3$ alors $n=v_p(x+y)$ donc $x+y=p^n$ et $x^2-xy+y^2=1$. On obtient à nouveau $x=y=1$ et $p=2$, absurde. Si $p=3$, alors $n=v_3(x+y)+1$ donc $3(x+y)=p^n$ et $(x-y)^2+xy=3$, d'où $x=1$, $y=2$ ou l'inverse. On obtient donc comme solutions $x=3^k$, $y=2 \times 3^k$ et $n=3k+2$ ou bien $x=2 \times 3^k$, $y= 3^k$ et $n=3k+2$.
\end{sol}

\begin{sol}
On va montrer qu'il y en a $n$ : les racines de $P$ sont des racines de $P(P(X))$ car $P(0)=0$ donc il suffit de montrer que ce sont les seules. Notons $r_1=0, r_2, \dots, r_n$ les racines de $P(P(X))$ et supposons que $r$ est une racine de $P(P(X))$ différente des $r_i$. Alors $P(r)$ est une racine de $P$ et n'est pas $0$. On la note $r_k$. Sans perte de généralité (quitte à remplacer $P$ par $-P(-X)$), on peut supposer $r_k \geq 0$. On a donc
\[r(r-r_2) \dots (r-r_n)=r_k\]
et en particulier $r(r-r_k)$ divise $r_k$. En particulier $r \leq r_k$ et $r_k-r \leq r_k$ donc $0<r<r_k$. En posant $s=r_k-r$ on obtient $rs | r+s$ avec $r,s>0$ donc $r|s$ et $s|r$ donc $r=s=1$ ou $2$. Dans le premier cas on a $r_k=2$ donc 
\[-\prod_{i \ne 1,k} r-r_i =2 \]
mais ce produit contient au moins trois facteurs différents de $r$ et de $r-r_k$, donc différents de $1$ et $-1$, donc la valeur absolue du membre de gauche vaut au moins $8$, ce que st absurde.

Dans le second cas on a
\[2 \times (-2) \prod_{i \ne 1,k} r-r_i =4 \]
mais le produit à gauche contient au moins $3$ facteurs distincts, donc au moins un qui n'est pas $1$ ou $-1$, et on obtient une nouvelle contradiction.
\end{sol}
