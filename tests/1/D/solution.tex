\begin{sol}
Pour $p=2$ ou $p=3$ le seul résidu quadratique modulo $p$ est $1$ donc leur somme vaut $1$ modulo $p$ et n'est pas divisible par $p$.

Supposons maintenant $p \geq 5$. Alors il y a $\frac{p-1}{2} \geq 2$ résidus quadratiques modulo $p$ donc il existe un résidu $q \ne 1$. De plus, si $x$ est un résidu quadratique modulo $p$ alors $qx$ en est un aussi. Si on note $s$ la somme recherchée on a donc $s \equiv qs \pmod{p}$, soit $p | (q-1)s$. Comme $q \ne 1$ on en déduit $p | s$.
\end{sol}

\begin{sol}
On sait que $(AS)$ est la symédiane issue de $A$ dans $APQ$, donc il suffit de montrer que $H$ est sur la symédiane. De plus, comme $(AB)$ est l'axe radical de $\Gamma_1$ et $\Gamma_2$, elle recoupe $[PQ]$ en son milieu donc $(AB)$ est la médiane issue de $A$, donc il suffit de montrer $\widehat{PAH}=\widehat{QAB}$.

Or, par chasse aux angles on a $\widehat{PHQ}=\widehat{PBQ}=180^{\circ}-\widehat{PAQ}$ donc $A$, $P$, $Q$ et $H$ sont cocycliques. On en déduit 
\[\widehat{PAH}=\widehat{PQH}=\widehat{PQB}=\widehat{QAB}.\]
\end{sol}

\begin{sol}
 Soit $M$ le second point d'intersection du cercle $(ABE)$ avec $(EF)$. Soit $N$ le
second point d'intersection de $(ADF)$ avec $(EF)$. On a
$EA.ED=EN.EF$ et $FA.FB=FM.EF$ donc $EA.ED+FA.FB=(EN+FM)EF$. La solution \'equivaut
donc \`a $M=N$.

Si $ABCD$ sont cocycliques, alors $\widehat{ABC}=\widehat{ADF}$ et
$\widehat{ABC}=\widehat{AME}$ donc $\widehat{AME}=\widehat{ADF}$. Par cons\'equent
$AMFD$ sont
cocycliques, d'o\`u $M=N$.

La r\'eciproque se d\'emontre de mani\`ere analogue.
\end{sol}

\begin{sol}
On raisonne par récurrence forte sur $m$ : le résultat est évident pour $m=1$. Soit $m _geq 2$ tel que le résultat soit vrai pour tout $k$ avec $1 \leq k < m-1$.

Si $m$ et $a$ sont premiers entre eux, on utilise l'hypothèse de récurrence pour $\varphi(m)<m$ : à partir d'un certain rang $(u_n)$ est périodique modulo $\varphi(m)$ donc $(a^{u_n})$ est périodique modulo $m$ donc $(u_{n+1})$ est périodique modulo $m$, ce qui permet de conclure.

Sinon, posons $m=bc$ où $b$ est premier avec $a$ et tous les facteurs premiers de $c$ divisent $a$ (on écrit la décomposition en facteur premiers de $m$ et on met dans $b$ les facteurs premiers qui ne divisent pas $a$ et dans $c$ ceux qui le divisent), de sorte que $b$ et $c$ sont premiers entre eux. On a $b<m$ donc par hypothèse de récurrence $(u_n)$ est périodique à partir d'un certain rang modulo $b$. D'autre part, $(u_n)$ est une suite d'entiers strictement croissante (sauf si $a=1$, auquel cas le problème est trivial) donc tend vers $+\infty$, donc à partir d'un certain rang $c$ divise $a^{u_n}$, donc à partir d'un certain rang $u_n \equiv 0 \pmod{c}$. La suite est donc périodique à partir d'un certain rang modulo $b$ et $c$, et on peut conclure, par le théorème chinois par exemple.
\end{sol}
