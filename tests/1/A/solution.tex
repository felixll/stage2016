\begin{sol}
%Logique
Pour montrer $A \text{ ou } (B \text{ et } C) \iff (A \text{ ou } B) \text{ et } (A \text{ ou } C)$,
on remarque qu'il s'agit de la distributivité du \og ou\fg sur le \og et\fg. Les sceptiques
se feront un plaisir de redémontrer cette propriété usuelle à l'aide de tables de vérités.

Pour montrer $\text{non} \big [ \big ( A \text{\ ou\ } (\text{non\ } B) \big ) \text{\ ou\ } \big ( C \text{\ et\ } \big ( A \text{\ ou\ } (\text{non\ } B)
\big ) \big ) \big ]
\iff (\text{non\ } A) \text{\ et\ } B$, on distribue à l'aide des lois de 
de Morgan le \og non\fg du membre de gauche, et on obtient : 
non $A$ et $B$ et [non $C$ ou (non $A$ et $B$)]. Si non $C$, la seconde
moitié de l'expression est vraie, donc l'expression vaut non $A$ et $B$.
Si $C$, la seconde moitié devient non $A$ et $B$ (comme la première moitié),
d'où le résultat.
\end{sol}


\begin{sol}
En faisant le tour de la clairière, on décide d'appeler un arbre sur deux un chêne,
et l'autre arbre sur deux un bouleau. \`A chaque tour, les oiseaux magiques nichant
dans un bouleau partent s'installer dans un chêne, et inversement. Comme au début,
autant d'oiseaux nichent dans des bouleaux que dans des chênes, cette propriété est
un invariant. Si tous les oiseaux nichaient dans un même arbre, l'invariant ne serait pas
respecté. Donc Panoramix ne verra jamais tous les oiseaux magiques nicher dans un même arbre. 
\end{sol}

\begin{sol}
Pour $1\le k\le 77$, on note $a_k\in\N^*$ le nombre de parties jouées le 
$k$-ième jour. Pour $0\le k\le 77$, soit $s_k=a_1+a_2+\ldots+a_k$. Par
l'absurde, supposons qu'on ne peut pas trouver de période de jours consécutifs 
pendant laquelle exactement 21 parties sont jouées. Cela revient à dire 
que pour tous $i<j$, on a $s_j-s_i\ne 21$.

Sur une droite, on dessine une case vide à la place de chaque entier de 0 à
132 inclus. La $n$-ième case est marquée d'une croix s'il existe $k$ tel
que $s_k=n$. On place ainsi 78 croix. De plus, pour tout $0\le k\le 77$, 
on a soit $s_k + 21>132$, soit la $s_k+21$-ième case est libre. Soit $a$
le nombre de cases numérotées entre 112 et 132 inclus et marquées d'une croix.
Les cases numérotées entre 0 et 111 comprennent exactement $78-a$ cases marquées
d'une croix, et au moins autant de cases libres. On a donc l'inégalité :
$2(78-a)+a\le 133$, ie $a\ge 23$. Or $a\le 132 - 112 +1 = 21$, absurde !

Donc il existe bien une période de jours consécutifs lors de laquelle exactement
21 parties ont été jouées.


\end{sol}


\begin{sol}
%récurrence
Pour $n\in\N^*$, on note ${\cal P}_n :$ \og $2^n\ge n+1\fg$. Montrons par récurrence
que ${\cal P}_n$ est vraie pour tout $n\in\N^*$.

Initialisation : pour $n=1$, on a $2\ge 2$.

Hérédité : soit $n\in\N^*$. Supposons que ${\cal P}_n$ est vraie et 
montrons ${\cal P}_{n+1}$. On a $2^{n+1}\ge 2(n+1)$ par hypothèse de récurrence,
et $2n+2\ge n+2$. D'où ${\cal P}_{n+1}$.

Conclusion : on a bien montré que ${\cal P}_n$ est vraie pour tout $n\in\N^*$.
\end{sol}
