\begin{exo}
%Logique
Montrer que :
\begin{enumerate}
\item $A \text{ ou } (B \text{ et } C) \iff (A \text{ et } B) \text{ ou } (A \text{ et } C)$
\item $\text{non} \Bigg [ \bigg ( A \text{\ ou\ } (\text{non\ } B) \bigg ) \text{\ ou\ } \bigg ( C \text{\ et\ } \Big ( A \text{\ ou\ } (\text{non\ } B)
\Big ) \bigg ) \Bigg ]
\iff (\text{non\ } A) \text{\ et\ } B$
\end{enumerate}
\end{exo}



\begin{exo}
Dans une clairière magique, il y a six arbres disposés en cercle et six oiseaux enchantés. À la nouvelle lune, le druide Panoramix remarque que tous les oiseaux sont sur un arbre différent. De plus, il sait que chaque jour, chaque oiseau se déplace d'un arbre soit vers la gauche soit vers la droite sur le cercle. Panoramix pourra-t-il un jour observer tous les oiseaux sur le même arbre ?
\end{exo}

\begin{exo}
Un joueur d'échecs participe à un tournoi de 77 jours. Il joue au moins une partie chaque jour, et au plus 132 parties en tout. Montrer qu'il existe une période de jours consécutifs lors de laquelle il joue exactement 21 parties.
\end{exo}

\begin{exo}
%récurrence
Montrer que pour tout $n \geq 1$, 
$2^n\ge n+1.$
\end{exo}
