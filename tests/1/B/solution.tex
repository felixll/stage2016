\begin{sol}
D'apr\`es l'in\'egalit\'e arithm\'etico-g\'eom\'etrique (IAG),
$\frac{\frac{x}{y}+\frac{y}{z}+\frac{z}{x}}{3} \geq \sqrt[3]{\frac{x}{y}\times\frac{y}{z}\times\frac{z}{x}} = 1$,
d'o\`u l'in\'egalit\'e demand\'ee.
\end{sol}

\begin{sol}
En faisant le tour du cercle, on d\'ecide de colorer un nid sur deux en blanc, et l'autre nid sur deux en noir. Chaque jour, les hirondelles nichant sur un nid noir migrent vers un nid blanc, et inversement. Comme au d\'ebut,
autant d'oiseaux nichent dans des nids noirs que dans des nids blancs, cette propri\'et\'e est donc
un invariant. Si tous les oiseaux nichaient dans un m\^eme nid, l'invariant ne serait pas
respect\'e. Ainsi, cette situation n'arrivera jamais. 
\end{sol}

\begin{sol}
On peut une fois de plus appliquer IAG dans cet exercice :
$\sqrt[n]{n!} = \sqrt[n]{1 \times 2 \times ... \times n} \leq \frac{1+2+...+n}{n} = \frac{n+1}{2}$,
d'o\`u l'in\'egalit\'e demand\'ee.
\end{sol}

\begin{sol}
On proc\`ede par r\'ecurrence forte sur $n$. Le cas $n = 1$ est \'evident. Supposons l’assertion vraie pour $k$ temples pour tout $k < n$, et montrons-la pour $n$ temples. L’\'el\'ephant de gauche du premier temple et l’\'el\'ephant de droite du dernier temple sont inutiles, car ils ne peuvent que s'\'eloigner des temples : supprimons-les.

On consid\`ere l’\'el\'ephant le plus gros parmi les $2n-2$ \'el\'ephants centraux. Supposons WLOG qu’il s'agisse de l’\'el\'ephant de droite du temple $T_{k}$. On applique l’hypoth\`ese de r\'ecurrence aux temples  $T_{1}, ..., T_{k}$. Il existe un unique temple $T_{i}$ avec $i \leq k$ tel que aucun \'el\'ephant gardant l’un des temples  $T_{1}, ..., T_{k}$ ne puisse atteindre $T_{i}$.

Il est \'evident qu’aucun  \'el\'ephant gardant l’un des temples  \`a droite de $T_{k}$ ne pourra atteindre $T_{i}$ puisqu’il doit d’abord passer par $T_{k}$. Donc aucun \'el\'ephant ne pourra atteindre $T_{i}$ : ceci prouve l’existence.

Comme l’\'el\'ephant de droite gardant $T_{k}$ peut atteindre tous les temples \`a droite de $T_{k}$, ceci garantit l’unicit\'e.
\end{sol}
