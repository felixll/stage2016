Si une preuve avait pour vocation de persuader, on pourrait la définir comme "un blabla informel pour arriver à un consensus". Or, horreur, Matthieu Lequesne a failli nous prouver de cette manière-là que $1+1=0$ ! On comprend donc que cette première définition naïve n'est absolument pas adaptée. Plutôt que d'utiliser l'intelligence supposée de l'auditoire, on voudrait au contraire pouvoir confier la vérification d'une preuve à un opérateur ignorant qui vérifie étape par étape si des règles symboliques définies préalablement ont été bien utilisées, comme l'expliquent les logiciens Bertrand Russell et Alfred Whitehead dans les \emph{Principia mathematica} au début du vingtième siècle.


Néanmoins, cela suppose de fixer un certain nombre (fini) de règles à l'avance... peut-on toujours établir un tel formalisme ? En 1930, Bourbaki propose une première tentative de formalisation générale des mathématiques par la théorie des ensembles mais définir le nombre $1$ nécessiterait dans ce cas pas moins de quatre millions de livres de 400 pages remplis de symboles !

Dans les années 1950 apparaissent les premiers ordinateurs et grâce à la correspondance de Curry-Howard entre preuve et calcul (1958), on peut les utiliser pour faire des preuves. Actuellement, on dispose par exemple des assistants de preuve actuels HOL-Isabelle et Coq, qui ont déjà permis de prouver 87 des "100 théorèmes mathématiques les plus importants" (dont le fameux théorème des quatre couleurs).


Comment se convaincre de la nécessité des preuves formelles ? D'abord parce que certaines démonstrations récentes sont bien trop longues pour que les mathématiciens puissent les vérifier en se passant de l'outil informatique (par exemple la classification des groupes simples, qui nécessite plus de 10 000 pages). Ensuite parce que les "assistants de preuve" sont utilisés également pour la vérification des logiciels ou programmes. Par exemple, on peut prouver qu'un ascenseur fonctionne correctement si on peut prouver que quelle que soit la situation, un incident n'arrivera pas.

Sources :
\begin{itemize}
\item \emph{Du rêve à la réalité des preuves}, Jean-Paul Delahaye, \url{https://interstices.info/jcms/int\_63417/du-reve-a-la-realite-des-preuves}
\item \emph{La vérité et la machine}, Benjamin Werner,  \url{https://interstices.info/jcms/c\_42623/la-verite-et-la-machine}
\end{itemize}

