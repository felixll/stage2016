\subsubsection{jeudi 18 août : la pr\'esentation des olympiades par Lucie Wang}

Le but de la soir\'ee \'etait de pr\'esenter aux \'el\`eves les diff\'erentes olympiades auxquelles la France participe, et la d\'emarche \`a suivre pour pouvoir y participer.

\bigskip

\textbf{Les comp\'etitions pr\'esent\'ees :}

\smallskip

En olympiades, il y en a pour tout le monde, comme en t\'emoigne la liste des comp\'etitions auxquelles il est possible de participer :
\begin{itemize}
    \item les IMO (International Mathematical Olympiad), la comp\'etition de r\'ef\'erence, la plus ancienne et la plus prestigieuse des olympiades. Elle r\'eunit chaque \'et\'e des \'el\`eves d'une centaine de pays, qui viennent plancher deux matin\'ees cons\'ecutives sur six probl\`emes de type \textit{olympique}.
    \item les BMO (Balkan Mathematical Olympiad), une version r\'egionale des IMO \`a laquelle la France participe en tant que pays invit\'e. \`A la diff\'erence des IMO, le format des \'epreuves aux BMO est d'une matin\'ee avec quatre exercices \`a r\'esoudre.
    \item les JBMO (Junior Balkan Mathematical Olympiad), olympiade ouverte aux plus jeunes (moins de 15 ans et demi le jour de l’\'epreuve, habituellement en juin), sur le m\^ome format que les BMO.
    \item les EGMO (European Girls Mathematical Olympiad), cr\'eation r\'ecente. Il s'agit d'une olympiade 100\% f\'eminine, les d\'el\'egations nationales \'etant compos\'ees de quatre filles. L'id\'ee capitale est d'encourager davantage les filles \`a faire des math\'ematiques \`a un niveau avanc\'e, dans un milieu o\`u les in\'egalit\'es de repr\'esentation se font malheureusement toujours sentir.
    \item le RMM (Romanian Master of Mathematics), comp\'etition r\'eserv\'ee \`a une vingtaine de pays seulement dans le monde, se d\'eroulant chaque ann\'ee en Roumanie.
    \item le MYMC (Mediterranean Youth Mathematical Championship), comp\'etition en \'equipe tr\`es diff\'erente des olympiades, organis\'ee en Italie pour les pays du bassin m\'editerran\'een. Les d\'el\'egations, constitu\'ees de deux gar\c{c}ons et deux filles, s'affrontent dans la r\'esolution de probl\`emes appelant \`a la malice des participants plut\^ot qu'\`a des preuves formelles, contrairement aux olympiades.
\end{itemize}

\bigskip

\textbf{La d\'emarche \`a suivre pour y participer :}

\smallskip

L'OFM (Olympiade Fran\c{c}aise de Math\'ematiques) est la pr\'eparation fran\c{c}aise \`a toutes les comp\'etitions mentionn\'ees ci-dessus. Elle se fait par correspondance, avec des envois mensuels, i.e. des feuilles d'exercices \`a faire et \`a renvoyer. Cette pr\'eparation est compl\'et\'ee par un stage pendant les vacances d'hiver, ainsi que quelques tests au cours de l'ann\'ee qui visent \`a s\'electionner les \'equipes pour les diff\'erentes olympiades. Pour y participer, rien de plus simple : il suffit de s'inscrire en ligne au test d'entr\'ee qui a lieu chaque ann\'ee d\'ebut octobre !

En compl\'ement \`a l'OFM, il existe des clubs de math\'ematiques dans diff\'erentes villes. Rassemblant matheux et matheuses de France et de Navarre, ils donnent un cadre extra-scolaire pour suivre des cours r\'eguli\`erement, afin de pr\'eparer aux olympiades, mais pas que. Pour ne citer que les principaux, il existe :
\begin{itemize}
    \item le club de Parimaths
    \item le club de math\'ematiques de Lyon
    \item le cercle Sofia Kovalevska\"ia \`a Toulouse
    \item le cercle math\'ematique \`a Strasbourg.
\end{itemize}

\bigskip

Quoi qu'il en soit, l'adresse \`a retenir : celle du site d'Animath (\url{www.animath.fr}), pour les informations relatives aux olympiades et aux clubs, comme pour l'inscription au test d'entr\'ee de l'OFM.
