Les exercices ne sont pas classés par ordre de difficulté.

\begin{exo}
Soit $A$ le nombre $A=\dfrac{3}{4+(\sqrt{3})^6}+\dfrac{1}{1+\frac{1}{2+\frac{1}{3}}}$. On écrit $A$ sous
la forme d'une fraction irréductible $A=\frac{a}{b}$ (i.e. $a$ et $b$ entiers naturels et $a$ est le plus petit possible). Calculer $a+b$. 
\end{exo}




\begin{exo}
 Soient $a$ et $b$ des entiers naturels tels que $\frac{a}{b}$ soit une fraction irréductible.
On suppose que $c=\dfrac{(1+\sqrt{2})^3}{\frac{a}{b}+\sqrt{2}}$ est un entier naturel. Calculer $a+b+c$.
\end{exo}





\begin{exo}
 $p$ et $q$ sont deux nombres premiers tels que $q-p$ et $q+p$ sont premiers. Que vaut $p+q$ ?
\end{exo}



\begin{exo}
 Combien y a-t-il d'entiers $n$ compris entre $1$ et $1000$ tels que $3n+1$ soit divisible par $10$ ?
\end{exo}



\begin{exo}
 $100$ points sont disposés de manière régulière sur un cercle. On trace toutes les droites
reliant deux voisins quelconques. Combien de points d'intersection obtient-on au total (y compris
les 100 points de départ) ?
\end{exo}



\begin{exo}
On dispose $n$ personnes en cercle. Au départ,
l'une d'entre elles dispose de $n$ jetons, et les autres aucun. A chaque étape, une personne peut donner un
jeton au voisin de l'un de ses voisins. On suppose qu'il est possible que, après un certain nombre d'étapes,
chaque personne possède exactement $1$ jeton.

Déterminer le nombre de valeurs possibles de $n$, sachant que $1\leqslant n\leqslant 2016$.
\end{exo}



\begin{exo}
 $ABCD$ est un carré, et $EFG$ est un triangle équilatéral, tels que les points $A,E,B,F,C,D,G$ soient sur un
même cercle et dans cet ordre, et tels que $AE=EB$. Déterminer la valeur en degrés de l'angle $10\times \widehat{DAG}$.
\end{exo}




\begin{exo}
 Soit $ABC$ un triangle tel que $AB=5$, $AC=12$ et $BC=13$. Soit $H$ le pied de la hauteur issue de $A$.
On écrit $HA$ sous la forme d'une fraction irréductible $\frac{a}{b}$. Calculer $a+b$.
\end{exo}




%%%%%%%%%%%%%%%%%%%%%%%%%%%%%%%%%%%%%%%%%%%%%%%%%%%%%%%%


\begin{exo}
Une plante $A$ grandit de $2$ cm par jour, et une plante $B$ de $3$ cm par jour. Aujourd'hui, $B$ est $6$ fois
plus grande que $A$. Dans $20$ jours, elle sera $4$ fois plus grande que $A$. Déterminer la somme des
âges actuels de $A$ et de $B$ (exprimée en jours). 
\end{exo}



\begin{exo}
 Déterminer la somme de tous les entiers naturels $a$ tels que $\dfrac{2\times 16^{95}}{2^{13a}}$ soit la puissance $17$-ième
d'un entier.
\end{exo}



\begin{exo}
 Un pion se déplace sur un échiquier $6\times 6$. Il part de la case en bas à gauche. A chaque étape, il peut sauter soit
sur la case juste à droite, soit sur la case juste en haut, et doit rejoindre la case en haut à droite, de sorte qu'il ne se
trouve jamais strictement au-dessus de la diagonale reliant la case de départ et la case d'arrivée. Déterminer le nombre
de chemins possibles.
\end{exo}


\begin{exo}
Soient $ABC$ et $ABD$ deux triangles rectangles en $C$ et $D$ respectivement, tels que $C$ et $D$ ne soient pas dans le même
demi-plan délimité par $(AB)$. On note $E$ le milieu de $[AB]$. Calculer la mesure en degrés de
l'angle (saillant) $\widehat{CED}$ sachant que $\widehat{BAC}=14^\circ$
et $\widehat{BAD}=17^\circ$.
\end{exo}

