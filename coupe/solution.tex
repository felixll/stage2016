\begin{sol}
\begin{center}
\begin{tikzpicture}[line cap=round,line join=round,>=triangle 45,x=0.3cm,y=0.3cm]
\clip(-5.74,-2.26) rectangle (15.28,26.88);
\draw (5,-2.26) -- (5,26.88);
\draw (0,0)-- (5,25);
\draw (10,0)-- (5,25);
\draw (10,10)-- (0,10);
\draw (0,0)-- (10,0);
\draw (10,0)-- (10,10);
\draw (0,10)-- (0,0);
\begin{scriptsize}
\fill [color=black] (0,0) circle (1.5pt);
\draw[color=black] (0.56,0.96) node {$A$};
\fill [color=black] (10,0) circle (1.5pt);
\draw[color=black] (10.59,0.96) node {$B$};
\fill [color=black] (10,10) circle (1.5pt);
\draw[color=black] (10.59,10.92) node {$C$};
\fill [color=black] (0,10) circle (1.5pt);
\draw[color=black] (0.56,10.92) node {$D$};
\fill [color=black] (5,10) circle (1.5pt);
\draw[color=black] (6,10.92) node {$G$};
\fill [color=black] (5,0) circle (1.5pt);
\draw[color=black] (5.54,0.96) node {$X$};
\fill [color=black] (5,25) circle (1.5pt);
\draw[color=black] (5.54,25.93) node {$Y$};
\fill [color=black] (2,10) circle (1.5pt);
\draw[color=black] (2.61,10.92) node {$E$};
\fill [color=black] (8,10) circle (1.5pt);
\draw[color=black] (8.54,10.92) node {$F$};
\end{scriptsize}
\end{tikzpicture}
\end{center}

 Notons $E$ et $F$ les points d'intersection de $(AY)$ et de $(BY)$ avec $(CD)$.
Soit $G$ le milieu de $[CD]$.

Pour des raisons de sym\'etrie, l'aire de $ADE$ vaut $\frac{1}{2}(100-99)=1/2\,cm^2$.
On en d\'eduit que $\frac{1}{2}\times AD\times DE=\frac{1}{2}$, ce qui donne $DE=\frac{1}{10}$,
et donc $EG=\frac{49}{10}$. Or, d'apr\`es le th\'eor\`eme de Thal\`es on a $\frac{ED}{EG}=\frac{DA}{GY}$, donc
$GY=DA\times \frac{EG}{ED}=10\times 49=490$, ce qui donne $XY=500 \, cm$.
\end{sol}

\paragraph{Remarque}
Si on ne conna\^it pas le th\'eor\`eme de Thal\`es, on peut raisonner comme suit : soit $x=XY$. L'aire du triangle $ABY$ vaut $\frac{1}{2} \times AB \times XY=5x$ et celle de $EFY$ vaut $\frac{1}{2} \times EF \times GY=\frac{49}{10} (x-10)$. L'aire du quadrilat\`ere $ABEF$ est la diff\'erence des deux, d'o\`u $99=5x-\frac{49}{10} (x-10)=\frac{1}{10} x + 49$, donc $\frac{1}{10}x=50$ et finalement $x=500 \, cm$.

\bigskip

\begin{sol}
 Soit $d$ la dur\'ee du trajet exprim\'ee en heures. On a $55d=(100c+10b+a)-(100a+10b+c)=99(c-a)$.
Ceci se simplifie en $5d=9(c-a)$. Ceci implique que $c-a$ est un multiple de $5$.
D'autre part, $c-a=5d/9>0$ et $c-a\leqslant c<10$, donc $c-a=5$.

Si $a=1$ alors $c=6$, et comme $a+b+c\leqslant 7$ on en d\'eduit que $abc=106$.

Si $a>1$ alors $c=5+a>6$ donc $a+b+c>7$, ce qui contredit l'hypoth\`ese.

Donc l'unique valeur possible de $abc$ est $106$.
\end{sol}

\begin{sol}
 Si la personne $A$ d\'esigne la personne $B$, et si elle dit ``$B$ est un chevalier'' alors
$A$ et $B$ sont dans le m\^eme groupe. Inversement, si elle dit ``$B$ est un truand'',
alors $A$ et $B$ sont dans des groupes diff\'erents.

Num\'erotons les personnes $P_1,\ldots,P_{2016}$, la personne $P_1$ \'etant tout devant.

$P_2$ est oblig\'ee de d\'esigner $P_1$, donc l'observateur sait
si $P_2$ est dans le m\^eme groupe que $P_1$.

$P_3$ est oblig\'ee de d\'esigner, soit $P_1$, soit $P_2$, donc l'observateur sait
si $P_3$ est dans le m\^eme groupe que $P_1$.

En continuant le raisonnement, l'observateur peut d\'eterminer exactement quelles sont
les personnes qui sont dans le m\^eme groupe que $P_1$ et celles qui ne le sont pas.

Comme le groupe le plus nombreux est constitu\'e de truands, l'observateur en d\'eduit
si $P_1$ est un truand ou un chevalier, puis d\'etermine qui est un chevalier et qui est un truand.
\end{sol}

\begin{sol}
 D'apr\`es l'in\'egalit\'e triangulaire, on a $MA\leqslant MC+CA=MC+BD\leqslant MC+BM+MD$.
L'\'egalit\'e a lieu si et seulement si $C\in [MA]$ et $M\in [BD]$. Si ces deux conditions sont
v\'erifi\'ees, n\'ecessairement $M$ se trouve \`a la fois sur $(AC)$ et $(BD)$, donc
$M$ est l'intersection des diagonales. Mais dans ce cas, la condition $C\in [MA]$ ne
peut pas \^etre satisfaite. On en d\'eduit que $MA<MB+MC+MD$.

Par sym\'etrie des r\^oles de $A,B,C,D$, les autres in\'egalit\'es similaires sont \'egalement
satisfaites.
\end{sol}

\begin{sol}
Les deux plus petites sommes sont $a+b<a+c$ donc $a+b=32$ et $a+c=36$. De m\^eme, les deux plus grandes sont $c+e<d+e$ donc $c+e=48$ et $d+e=51$. La troisi\`eme plus petite somme peut \^etre soit $a+d$ soit $b+c$. Dans le premier cas, on a $a+d=37$ donc d'une part $a+c+d+e=(a+c)+(d+e)=36+51=87$ et d'autre part $a+c+d+e=(a+d)+(c+e)=37+48=85$, ce qui est absurde.

On est donc dans le deuxi\`eme cas donc $b+c=37$. On en d\'eduit $2(a+b+c)=(a+b)+(a+c)+(b+c)=105$ donc $c=\frac{105}{2}-32=\frac{41}{2}$, puis $e=48-\frac{41}{2}=\frac{55}{2}$.
\end{sol}

\begin{sol}
 Soit $a$ le nombre de triangles dont les trois c\^ot\'es sont des diagonales, $b$ le nombre de triangles
dont deux c\^ot\'es sont des diagonales et $c$ le nombre de triangles dont un c\^ot\'e est une diagonale.
Par hypoth\`ese, on a $a+b+c=2014$.

Comme une ar\^ete du polygone est un c\^ot\'e d'un triangle et un seul, et comme le polygone a 2016 ar\^etes,
on a $2016=b+2c$, donc $b+2c>a+b+c$. On en d\'eduit que $c>a$, donc $2a<a+c\leqslant a+b+c=2014$.
Par cons\'equent, on ne peut pas avoir $a=2014/2$, autrement dit la r\'eponse \`a la question est non. 
\end{sol}

\begin{sol}
Notons $s(n)$ la somme des chiffres de $n$.
 Si un entier $n$ est inf\'erieur ou \'egal \`a 2016, alors $s(n)\leqslant 28$. En effet, si $n<2000$ alors $s(n)\leqslant 1+9+9+9=28$, et si $2000\leqslant n\leqslant 2016$ alors
$s(n)\leqslant 2+0+1+9=12$.

Or, parmi $18$ entiers cons\'ecutifs, il y a un entier $n$ divisible par $18$.
Comme $n$ est divisible par $9$, la somme de ses chiffres est divisible par $9$, donc
$s(n)=9$ ou $s(n)=18$ ou $s(n)=27$.

Si $s(n)=9$ ou $s(n)=18$, alors $n$ est bien divisible par la somme de ses chiffres.

Si $s(n)=27$, alors $n$ est l'un des nombres $999,1899,1989,1998$. Or, $n$ est
pair donc $n=1998$, et $n$ est bien divisible par $27$.
\end{sol}

\begin{sol}
 a) C'est possible avec une classe de trois \'el\`eves $A,B,C$ et un contr\^ole de trois questions, en supposant que $A$ a r\'eussi les questions $1$ et $2$, $B$ a r\'eussi les questions $1$ et $3$ et $C$ n'a r\'eussi aucune question.

 b) et c) On montre que la r\'eponse est non. Plus pr\'ecis\'ement, pour tout $p>2/3$ on
montre qu'il n'est pas possible que la proportion d'\'el\`eves ayant bien r\'eussi soit $\geqslant p$ et que la proportion de questions difficiles (r\'eussies par une proportion d'\'el\`eves
$\leqslant 1-p$) soit $\geqslant p$.

Supposons en effet le contraire. Soit $n$ le nombre d'\'el\`eves et $m$ le nombre de questions. Soit $A$ l'ensemble des \'el\`eves ayant bien r\'eussi,
% $B$ l'ensemble des \'el\`eves n'ayant pas bien r\'eussi,
$D$ l'ensemble des questions difficiles et $F$ l'ensemble des questions faciles.

Pour tout ensemble $X$, on notera $|X|$ le nombre d'\'el\'ements de $X$. Soit $a$ tel que $|A|=an$. Supposons $a\geqslant p\geqslant 2/3$.

%Quitte \`a remplacer les r\'eponses de tous les \'el\`eves de $B$ par des copies blanches (ce qui
%ne peut qu'augmenter le nombre de questions difficiles),
%on peut supposer que les \'el\`eves de $B$ n'ont r\'eussi aucune question.

Consid\'erons l'ensemble $E$ des couples $(x,y)$ o\`u $x$ est un \'el\`eve de $A$ ayant r\'eussi la question $y$. 

Pour tout $x\in A$, $x$ a r\'esolu au moins $pm$ questions, donc $|E|\geqslant pamn$.

Pour tout $y\in D$, $y$ a \'et\'e r\'esolue par au plus $(1-p)n$ \'el\`eves, et pour tout $y\in F$, $y$ a \'et\'e r\'esolue par au plus $an$ \'el\`eves de $A$, donc $|E|\leqslant |D|\times (1-p)n+ |F|\times an =(m-|F|)\times (1-p)n+|F|\times an=(1-p)mn+|F|\times (a+p-1)n\leqslant (1-p)(a+p)mn$.

On en d\'eduit que $pamn\leqslant (1-p)(a+p)mn$, donc $a(2p-1)\leqslant p(1-p)$. Par cons\'equent, $2p-1\leqslant 1-p$. En simplifiant,
cela donne $p\leqslant 2/3$.
\end{sol}
