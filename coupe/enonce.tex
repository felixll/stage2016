\subsubsection*{Instructions}\renewcommand{\labelitemi}{$\triangleright$}
\begin{itemize}
%\item Il est \textbf{imp\'eratif} de rendre une feuille simple s\'epar\'ee sur laquelle vous \'ecrirez votre nom, pr\'enom, adresse email, nom de l'\'etablissement et sa ville ainsi que votre classe. 
%\medskip

 \item \textbf{R\'edigez les diff\'erents probl\`emes sur des copies distinctes. Sur chaque copie, \'ecrivez en lettres capitales
vos nom et pr\'enom en haut \`a gauche ainsi que votre classe, et le num\'ero du probl\`eme en haut \`a droite.}
\medskip
\item On demande des solutions \textbf{compl\`etement r\'edig\'ees} (sauf pour l'exercice 1), o\`u toute affirmation est soigneusement \textbf{justifi\'ee}. La
notation tiendra compte de la \textbf{clart\'e} et de la \textbf{pr\'ecision} de la copie.\\
\quad Travaillez d'abord au brouillon, et r\'edigez ensuite au propre votre solution, ou une tentative, r\'edig\'ee, de
solution contenant des r\'esultats significatifs pour le probl\`eme.\\
\quad Ne rendez pas vos brouillons : ils ne seraient pas pris en compte.
\medskip
\item Une solution compl\`ete rapportera plus de points que plusieurs tentatives inachev\'ees. Il vaut mieux
terminer un petit nombre de probl\`emes que de tous les aborder.

\medskip\item R\`egles, \'equerres et compas sont autoris\'es. Les rapporteurs sont interdits.\\
Les calculatrices sont interdites, ainsi que tous les instruments \'electroniques.

%\vspace {-0.2cm}


 %\begin{center}
%\rput(0,-8pt){\psvectorian[scale=0.3]{88}}
%\end{center}


\bigskip
\textbf{Les coll\'egiens traitent les exercices 1 \`a 5. Les lyc\'eens traitent les exercices 4 \`a 8.}

\textbf{Chaque exercice est not\'e sur 7 points.}

%\item[]
%Le groupe B est constitu\'e des \'el\`eves n\'es en 1999 ou apr\`es, avec les exceptions suivantes :
%
%* les \'el\'eves de Terminale sont dans le groupe A,
%
%* les \'el\`eves de Seconde et Premi\`ere qui \'etaient \`a l'OFM en 2012-2013 sont dans le groupe A.
%
%\quad Les autres \'el\`eves sont dans le groupe A.


%\smallskip\item[-] Les exercices class\'es \og Groupe B\fg\ ne sont \`a chercher que par les \'el\`eves du groupe B.
%\item[-] Les exercices class\'es \og communs\fg\  sont \`a chercher par tout le monde. 
%\item[-] Les exercices class\'es \og Groupe A\fg\  ne sont \`a chercher que par les \'el\`eves du groupe A.




\end{itemize}

\pagebreak


%\begin {center}
%{ \Large \'Enonc\'es des exercices}
%\end {center}

%\bigskip

\emph {Merci de bien vouloir respecter la num\'erotation des exercices. R\'edigez les diff\'erents probl\`emes sur des copies distinctes. Sur chaque copie, \'ecrivez en lettres capitales
vos nom et pr\'enom en haut \`a gauche ainsi que votre classe, et le num\'ero du probl\`eme en haut \`a droite.}

\bigskip
{\bf\large \'Enonc\'es coll\`ege}
\par\medskip



\begin{exo}
\emph{N.B. Dans cet exercice, et uniquement celui-ci, on demande une r\'eponse sans justification.}

Soit $ABCD$ un carr\'e de c\^ot\'e $10 \, cm$. On note $X$ le milieu de $[AB]$. On place un point $Y$ tel que le triangle $ABY$ est isoc\`ele en $Y$, et la zone qui est \`a la fois \`a l'int\'erieur de $ABY$ et de $ABCD$ a une aire de $99 \, cm^2$. Combien vaut $XY$ ?
\end{exo}

\begin{exo}
Anne a conduit sa voiture pendant un nombre entier (et non nul) d'heures, et a parcouru un nombre entier de kilom\`etres, \`a la
vitesse de $55$ $km/h$. Au d\'ebut du trajet, le compteur indiquait $abc$ kilom\`etres, o\`u $abc$ est un nombre de $3$ chiffres
tel que $a\geqslant 1$ et $a+b+c\leqslant 7$. A la fin du trajet, le compteur indiquait $cba$ kilom\`etres. D\'eterminer toutes les valeurs possibles
du nombre $abc$.
\end{exo}

\begin{exo}
$2016$ personnes sont en file indienne. Chacune d'elles est soit un truand (qui ment toujours), soit un chevalier (qui dit toujours la v\'erit\'e).
Chacune des 2016 personnes sait qui sont les truands et les chevaliers.
Chaque personne, except\'e celle qui est tout devant, d\'esigne l'une des personnes devant elle et dit l'une des deux phrases : ``cette personne est un truand'' ou ``cette personne est un chevalier''.

Sachant qu'il y a strictement plus de truands que de chevaliers, comment un observateur peut-il d\'eterminer qui est un truand et qui est un chevalier ?
\end{exo}




\bigskip
{\bf\large \'Enonc\'es communs}
\par\medskip

\begin{exo}
 Dans le plan, on consid\`{e}re un trap\`{e}ze $ABCD$ dont les diagonales sont de m\^{e}me longueur. Prouver que, pour tout point $M$ du plan, la somme des distances de $M$ \`{a} trois sommets quelconques du trap\`{e}ze est toujours strictement plus grande que la distance de $M$ au quatri\`{e}me sommet.
\end{exo}

\begin{exo}
Soient $a<b<c<d<e$ des nombres r\'eels. On calcule toutes les sommes possibles de deux nombres distincts parmi ces cinq nombres. Les trois plus petites valent $32$, $36$ et $37$ et les deux plus grandes valent $48$ et $51$. Trouver toutes les valeurs possibles de $e$.
\end{exo}

\bigskip
{\bf\large \'Enonc\'es lyc\'ee}
\par\medskip

\begin{exo}
 Dans un polygone convexe \`a 2016 c\^ot\'es, on trace certaines diagonales, qui ne se
coupent pas \`a l'int\'erieur du polygone. Ce dessin d\'ecompose le polygone en 2014 triangles. Est-il
possible qu'exactement la moiti\'e de ces triangles aient leurs trois c\^ot\'es qui soient des diagonales ?
\end{exo}

\begin{exo}
 Montrer que parmi $18$ entiers cons\'ecutifs inf\'erieurs ou \'egaux \`a 2016, il en existe au moins un qui est divisible
par la somme de ses chiffres.
\end{exo}

\begin{exo}
 a) Un contr\^ole a eu lieu dans une classe. On sait qu'au moins les deux tiers des questions de ce
contr\^ole \'etaient difficiles : pour chacune de ces questions difficiles, au moins les deux tiers des
\'el\`eves n'ont pas su r\'epondre. On sait aussi qu'au moins les deux tiers des \'el\`eves ont bien r\'eussi
le contr\^ole : chacun d'eux a su r\'epondre \`a au moins deux tiers des questions. Est-ce possible ?

b) La r\'eponse \`a la question pr\'ec\'edente serait-elle la m\^eme si l'on rempla\c{c}ait partout deux tiers
par trois quarts ?

c) La r\'eponse \`a la premi\`ere question serait-elle la m\^eme si l'on rempla\c{c}ait partout deux tiers par
sept dixi\`emes ?
\end{exo}

