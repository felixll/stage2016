\emph{La préparation du stage s'est achevée dans la précipitation, et nous n'avons pas pris le temps de réétudier attentivement l'évaluation initiale, servant à répartir les élèves en quatre groupes de niveau. A vrai dire, nous étions satisfaits du questionnaire proposé l'an passé, qui alliait des questions d'auto-évaluation et quelques exercices simples pour vérifier la justesse de cette estimation. C'était déjà l'aboutissement de plusieurs années de tâtonnement. Mais dans la mesure où un certain nombre de stagiaires étaient déjà là l'an passé, il aurait été judicieux de changer au moins les quelques exercices proposés. Voici, hormis l'en-tête (Nom, prénom, classe), le questionnaire soumis aux élèves, avec les solutions des exercices en italique.}


%\begin{center}
%{\Large \textsc{Stage Olympique Animath -- Montpellier  2016}}
%\end{center}



%\bigskip

%Classe (2015 - 2016): \qquad \qquad\qquad \qquad  Prénom: \qquad \qquad\qquad \qquad\qquad Nom:  
\medskip

Le but de ce petit questionnaire est de nous aider à choisir le groupe le plus adapté à tes connaissances et dans lequel tu pourras évoluer à ton rythme. Réponds si possible sur la feuille, éventuellement au dos... et si c'est vraiment nécessaire, ajoute une feuille. Ne rends pas tes brouillons.



%\bigskip

Les quatre groupes du stage sont les suivants:

\begin{enumerate}
\item[\textbf{Groupe A:}] Groupe prioritairement destiné aux élèves sortant de 4ème et de 3ème qui ne sont pas encore familiers avec les exercices de type ``olympiades''.
\item[\textbf{Groupe B:}] Groupe prioritairement destiné aux élèves sortant de 2nde et de 1ère qui ne sont pas encore familiers avec les exercices de type ``olympiades''. Le groupe B abordera grosso modo les mêmes choses que le groupe A, mais avancera plus vite.
\item[\textbf{Groupe C:}] Groupe destiné aux élèves familiers avec les connaissances et outils de base du programme des Olympiades Internationales de Mathématiques.
\item[\textbf{Groupe D:}] Groupe destiné aux élèves les plus avancés.
\end{enumerate}
\ligne

\textbf{Important:} ce n'est pas du tout un souci si tu n'es pas familier avec les éléments ci-dessous qui ne sont pas abordés dans le cadre scolaire.

\ligne

\begin{itemize}


\item As-tu déjà participé à des stages Animath ?  \qquad \fbox{OUI}  \qquad \fbox{NON}

Si oui, quand et lesquels ? Dans quel groupe étais-tu ?
%\bigskip
%\bigskip

\item As-tu déjà participé à des clubs de mathématiques?  \qquad \fbox{OUI}  \qquad \fbox{NON}

Si oui, quand et lesquels ?
%\bigskip
%\bigskip



\item As-tu déjà participé à des compétitions comme le TFJM, Kangourou,  \qquad \fbox{OUI}  \qquad \fbox{NON}

Al Kindi, Castor... ?  Si oui, quand ? lesquelles ? résultats ?  
%\bigskip
%\bigskip


\item As-tu déjà été élève à l'Olympiade Fran\c caise de Mathémiques (OFM) ?  \qquad \fbox{OUI}  \qquad \fbox{NON}
\end{itemize}


\ligne

As-tu un avis sur le groupe dans lequel tu devrais être ?
\ligne
{\Large\textbf{Géométrie}}
\bigskip

\begin{itemize}
\item
Connais-tu le théorème de l'angle inscrit ?  \qquad \fbox{OUI}  \qquad \fbox{NON}


Si oui, résous l'exercice suivant :

Soient $\mathscr{C}_1$ et $\mathscr{C}_2$ deux cercles qui se coupent en deux points $A$ et $B$. Soient $C$ et $D$ deux points de $\mathscr{C}_1$. Les droites $(AC)$ et $(BD)$ recoupent $\mathscr{C}_2$ en $E$ et $F$. Montrer que $(CD)$ et $(EF)$ sont parallèles.
%
%ALTERNATIVE (PLUS DURE) :
%
%Soient $\mathscr{C}_1$ et $\mathscr{C}_2$ deux cercles qui se coupent en deux points $A$ et $B$. Soit $C$ un point de $\mathscr{C}_1$. Les droites $(AC)$ et $(BC)$ recoupent $\mathscr{C}_2$, respectivement en $D$ et $E$. On note $O$ le centre de $\mathscr{C}_1$.
%
%Montrer que $(OC)$ et $(DE)$ sont perpendiculaires.
%\bigskip
%\bigskip
%\bigskip
%\bigskip
%\bigskip
%\bigskip
%\bigskip
%\bigskip
\bigskip

\emph{D'après le théorème de l'angle inscrit, les angles $\widehat{ACD}$ et $\widehat{ABD}$ sont supplémentaire, de même que les angles $\widehat{ABF}$ et $\widehat{AEF}$. Or $\widehat{ABD}$ et $\widehat{ABF}$ sont supplémentaires : il en résulte que $\widehat{ACD}$ et $\widehat{AEF}$ sont eux aussi supplémentaires, donc que $(CD) \sslash (EF)$. }

\bigskip

\item
Connais-tu la puissance d'un point par rapport à un cercle ?  \qquad \fbox{OUI}  \qquad \fbox{NON}

Si oui, quel est la nature de l'ensemble des points ayant même puissance par rapport à deux cercles différents?

%\bigskip
\bigskip

\emph{C'est une droite appelée axe radical des deux cercles. Si les cercles se coupent en $A$ et $B$, c'est la droite $(AB)$.}

\bigskip

\item
Sais-tu ce qu'est une homothétie ?  \qquad \fbox{OUI}  \qquad \fbox{NON}

\item
Sais-tu ce qu'est une similitude ?  \qquad \fbox{OUI}  \qquad \fbox{NON}
\end{itemize}

\bigskip

{\Large\textbf{Combinatoire}}
\bigskip

\begin{itemize}
\item Connais-tu le principe des tiroirs ? \qquad \fbox{OUI}  \qquad \fbox{NON}
\item Sais-tu ce qu'est une démonstration par récurrence ?  \qquad \fbox{OUI}  \qquad \fbox{NON}


Si oui, résous l'exercice suivant :

Soit $n \geq 1$ un entier. Montrer que $1^2+2^2+ \cdots +n^2=\frac{n(n+1)(2n+1)}{6}$.

\bigskip

\emph{\textbf{Initialisation} : Le résultat est vrai pour $n = 1$ : $1^2 = \dfrac{1 \times 2 \times 3 }{6}$. }

\smallskip

\emph{\textbf{Hérédité} : Si le résultat est vrai pour $n-1$, donc si $1^2+2^2+\cdots+(n-1)^2 = \dfrac{(n-1)n(2n-1)}{6} = \dfrac{2n^3-3n^2+n}{6}$, alors $1^2 + 2^2 + \cdots + (n-1)^2 + n^2 = \dfrac{2n^3 - 3n^2 + n}{6} + n^2 = \dfrac{2n^3 + 3n^2 + n}{6} = \dfrac{n(n+1)(2n+1)}{6}$, donc le résultat est encore vrai pour $n$, ce qui achève la démonstration}

\bigskip
%\bigskip
%\bigskip
%\bigskip
%\bigskip
%\bigskip
%\bigskip
%\bigskip


\item
Sais-tu ce qu'est un coefficient binomial ?


Si oui, que vaut ${n \choose k} +{n \choose k+1}$ ?

\bigskip
\emph{${n \choose k} +{n \choose k+1} = {n+1 \choose k+1}$, cela résulte de la formule du binôme et permet de construire le triangle de Pascal}

\end{itemize}

\bigskip
{\Large\textbf{Algèbre}}
\bigskip

\begin{itemize}

\item Sais-tu ce que signifient  les symboles $\displaystyle \sum_{i=1}^{n}$ et $\displaystyle\prod_{i=1}^{n}$?  \qquad \fbox{OUI}  \qquad \fbox{NON}

\item Connais-tu la formule du binôme de Newton ?  \qquad \fbox{OUI}  \qquad \fbox{NON}

Si oui, écris-la :
\bigskip
%\bigskip
\emph{Pour tous $a$ et $b$ réels, et tout entier $n > 0$, $(a+b)^n = \displaystyle \sum_{i=0}^{n} {n \choose i}a^ib^{n-i} $}

\bigskip
\item
Sais-tu effectuer une division euclidienne polynomiale ? \qquad \fbox{OUI}  \qquad \fbox{NON}

\item
Connais-tu l'inégalité arithmético-géométrique ? \qquad \fbox{OUI}  \qquad \fbox{NON}

Si oui, essaie de résoudre l'exercice suivant :

Soient $a,b>0$ deux nombres réels tels que $a^3b^2=1$. Montrer que $(1+a)(2+b)\geq 6$.
\end{itemize}

\bigskip

\emph{D'après l'inégalité arithmético-géométrique, $\dfrac{1+a}{2} \geq (1 \times a)^\frac12 = \sqrt{a}$ et $\dfrac{1 + 1 + b}{3} \geq (1 \times 1 \times b)^\frac13 = \sqrt[3]{b} $, donc $\dfrac{(1+a)(2+b)}{6} \geq a^{\frac12}b^{\frac13} = \left( a^3b^2 \right) ^\frac16 = 1$.}

\bigskip


%\bigskip
%\bigskip
%\bigskip

\bigskip
{\Large\textbf{Arithmétique}}
\bigskip


\begin{itemize}

\item
Sais-tu ce qu'est le PGCD de deux entiers ?  \qquad \fbox{OUI}  \qquad \fbox{NON}

Si oui, calcule le PGCD de $132$ et de $63$.

\bigskip

%\bigskip
%\bigskip
%\bigskip
\emph{Par l'algorithme d'Euclide : $132 = 2 \times 63 + 6$, $63 = 10 \times 6 + 3$, $6 = 2 \times 3 + 0$, donc le PGCD est le dernier reste non nul, à savoir $3$.}
\bigskip
\item
%Sais-tu prouver qu'il existe une infinité de nombres premiers ? OUI NON
As-tu déjà manipulé des congruences ?  \qquad \fbox{OUI}  \qquad \fbox{NON}

Si oui, est-il vrai que si $a \equiv b \mod n$ et $c \equiv d \mod n$, alors $a^{c} \equiv b^{d} \mod n$ (justifie ta réponse)?

\bigskip

\emph{Non ! C'est faux. Pour $n = 3$ par exemple, $2 \equiv 5 \pmod{3}$ et $4 \equiv 1 \pmod {3}$, mais $2^4 = 16$ n'est pas congru à $5^1 = 5$ modulo $3$, car manifestement $16 - 5 = 11$ n'est pas multiplie de $3$.}
%\bigskip
%\bigskip
%\bigskip
%\bigskip
\bigskip


\item Connais-tu le petit théorème de Fermat ?  \qquad \fbox{OUI}  \qquad \fbox{NON}

Si oui, écris-le :

\bigskip
%\bigskip
\emph{Soit $p$ un nombre premier. Pour tout entier $a$ non divisible par $p$, $a^{p-1} \equiv 1 \pmod{p}$. Ce qui peut encore s'écrire : pour tout entier $a$, divisible ou non par $p$, $a^p \equiv a \pmod p$.}
\bigskip
\item Essaie de résoudre l'exercice suivant :
\begin{enumerate}
\item[(i)] Trouver tous les entiers $n$ tels que $2^n+1$ soit un carré. 
\item[(ii)] Trouver tous les entiers $n$ tels que $2^n+1$ soit un cube. 
\end{enumerate}
\end{itemize}

\bigskip
\emph{Si $2^n+1 = k^2$ alors $2^n = k^2-1 = (k-1)(k+1)$, donc $k-1$ et $k+1$ sont tous deux des puissances de $2$, comme ils sont distants de $2$, un seul des deux est divisible par $4$, d'où $n-1 = 2$, $n+1 = 4$, $n = 3$ et l'on a bien $2^3+1 = 3^2$.}

\emph{Si $2^n+1 = k^3$, $2^n = k^3-1 = (k-1)\left(k^2+k+1\right)$. Or pour tout entier $k > 0$, $k(k+1)$ est pair donc $k^2+k+1$ est impair et supérieur ou égal à $1+1+1 = 3$. Comme $2^n$ n'admet pas de diviseur impair $\geq 3$, et que $k = 0$ est lui aussi exclu, il n'existe pas d'entier $n$ tel que $2^n+1$ soit un cube.}
%\bigskip
%\bigskip
%\bigskip
%\bigskip
%\bigskip
%\bigskip
\ligne
Si (et SEULEMENT si) tu as déjà fait tout ce que tu pouvais pour les questions précédentes, traite ces exercices (sur une feuille séparée):

1) On écrit un entier strictement positif sur chaque face d'un dé. On attribue à chaque sommet le produit des trois faces qui l'entourent. La somme des valeurs des sommets vaut $105$, quelle est la somme des nombres écrits sur les faces ?

\bigskip

\emph{Appelons $a, \ b, \ c, \ d, \ e, \ f$ les nombres inscrits sur les faces, de sorte que $a$ et $d$ soient écrits sur des faces opposées, $b$ et $e$ sur d'autres faces opposées, et $c$ et $f$ sur les deux dernières faces, opposées. Si l'on développe : $(a+d)(b+e)(c+f)$, on obtient une somme de huit termes qui sont précisément les nombres attribués aux $8$ sommets du cube. Il faut donc décomposer $105$ en un produit de trois facteurs chacun supérieur ou égal à $2$, ce qui n'est possible que d'une seule manière car $105 = 3 \times 5 \times 7$. La somme des six nombre écrits sur les faces vaut donc $(a+d)+(b+e)+(c+f) = 3 + 5 + 7 = 15$.}

\bigskip

2) Trouver les entiers naturels $n$ tels que le produit de leurs chiffres soit égal à $n^2-15n-27$.

\bigskip

\emph{Pour $n \leq 16$, $n^2 - 15n - 27 = (n + 1)(n - 16) - 11 < 0$. Pour $n = 17$, on a bien $1 \times 7 = 17^2 - (15 \times 17) - 27$, donc $17$ est la plus petite solution. Montrons que c'est la seule. Pour $18 \leq n \leq 20$, le produit des chiffres vaut $8, \ 9$ et $0$ alors que $n^2 - 15n - 27 = (n+1)(n-16) - 11 \geq (19 \times 2 ) - 11 = 27$. Pour $21 \leq n \leq 99$, le produit des deux chiffres est inférieur ou égal à $9 \times 9 = 81$ alors que $n^2 - 15n -27 = n(n-15) -27 \geq (21 \times 6) - 27 = 99$, donc l'égalité n'est pas possible. Pour $10^k \leq n < 10^{k+1}$, avec $k \geq 2$, le produit des $k+1$ chiffres est majoré par $9^{k+1}$ et $n^2 - 15n - 27 = (n-8)^2 + (n-91) \geq (n-8)^2$ car $n>99$, et $(n-8)^2 > \left(10^k - 10\right)^2 \geq (9 \times 10^{k-1})^2$. Il reste à prouver que $9^2 \times 10^{2k-2} > 9^{k+1}$, donc que $100^{k-1} > 9^{k-1}$, ce qui est évident car $100 > 9$. Calcul laborieux, certes, mais les majorations sont très larges, et prouvent bien que l'unique entier naturel solution est $17$.} 

