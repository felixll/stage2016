Les exercices ne sont pas classés par ordre de difficulté.

%%%%%%%%%%%%%%%%%%%%%%%%%%%%%%%%%%%%%%%%%%%%%%%%%%%%%%%%


\begin{exo}
Une plante $A$ grandit de $2$ cm par jour, et une plante $B$ de $3$ cm par jour. Aujourd'hui, $B$ est $6$ fois
plus grande que $A$. Dans $20$ jours, elle sera $4$ fois plus grande que $A$. Déterminer la somme des
âges actuels de $A$ et de $B$ (exprimée en jours). 
\end{exo}



\begin{exo}
 Déterminer la somme de tous les entiers naturels $a$ tels que $\dfrac{2\times 16^{95}}{2^{13a}}$ soit la puissance $17$-ième
d'un entier.
\end{exo}



\begin{exo}
 Un pion se déplace sur un échiquier $6\times 6$. Il part de la case en bas à gauche. A chaque étape, il peut sauter soit
sur la case juste à droite, soit sur la case juste en haut, et doit rejoindre la case en haut à droite, de sorte qu'il ne se
trouve jamais strictement au-dessus de la diagonale reliant la case de départ et la case d'arrivée. Déterminer le nombre
de chemins possibles.
\end{exo}


\begin{exo}
Soient $ABC$ et $ABD$ deux triangles rectangles en $C$ et $D$ respectivement, tels que $C$ et $D$ ne soient pas dans le même
demi-plan délimité par $(AB)$. On note $E$ le milieu de $[AB]$. Calculer la mesure en degrés de
l'angle (saillant) $\widehat{CED}$ sachant que $\widehat{BAC}=14^\circ$
et $\widehat{BAD}=17^\circ$.
\end{exo}


%%%%%%%%%%%%%%%%%%%%%%%%%%%%%%%%%%%%%%%%%%%


\begin{exo}
Soit $u_1,u_2,u_3,\ldots$ une suite de nombres réels tels que $u_1=7$, $u_{2016}=6102$ et $u_{n+2}+u_n=2u_{n+1}$ pour tout $n$.
Déterminer la valeur $u_{4031}$. 
\end{exo}



\begin{exo}
 $m,n,p,q$ sont des entiers tels que $a^4+ma^3+na^2+pa+q=0$, où $a=\sqrt{2}+\sqrt{3}$. Déterminer $m^2+n^2+p^2+q^2$.
\end{exo}



\begin{exo}
 Déterminer le nombre d'entiers $n\geqslant 1$ vérifiant les deux conditions suivantes:

(i) le produit de tous les entiers naturels qui divisent $n$ est égal à $n^2$ ;

(ii) $n$ n'est divisible par aucun nombre premier $\geqslant 10$.
\end{exo}




\begin{exo}
 Quel est le plus grand entier qui divise tous les entiers de la forme $a(a+2)(a+4)$ ($a\in \N^*$) ?
\end{exo}



\begin{exo}
 Soit $a_n$ le nombre de manières de colorier chaque case d'un échiquier $2\times n$ en bleu ou en rouge,
de sorte que deux cases bleues ne soient jamais adjacentes (deux cases sont dites adjacentes si elles
ont une arête commune). On a par exemple $a_1=3$. Déterminer $a_{10}$.
\end{exo}



\begin{exo}
 Soit $S=\{1,2,3,\ldots,2016\}$. Trouver le plus grand entier $n$ vérifiant la propriété suivante :
il existe un sous-ensemble $A$ de $S$ possédant $n$ éléments tel que la différence de deux éléments
quelconques de $A$ ne divise jamais leur somme.
\end{exo}



\begin{exo}
 $ABC$ est un triangle isocèle en $A$ tel que $AB$ et $BC$ soient des entiers. On suppose que $BI=8$,
où $I$ est l'intersection des bissectrices de $\widehat{B}$ et $\widehat{C}$. Calculer $AB+BC$.
\end{exo}



\begin{exo}
 Quatre points $A,O,B,O'$ sont alignés dans cet ordre sur une droite. Soit $C$ le cercle de centre $O$ et
de rayon $2015$, et $C'$ le cercle de centre $O'$ et de rayon $2016$. On suppose que $A$ et $B$ sont des
intersections de deux tangentes communes aux deux cercles. Calculer $AB$ sachant que $AB$ est un entier $<10^7$
et que $AO$ et $AO'$ sont des entiers.
\end{exo}

