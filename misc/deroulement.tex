
C'est la troisième fois, après 1999 et 2006, qu'Animath organise son stage au Centre International de Valbonne, mais le stage est beaucoup plus important qu'en 2006 : il dure dix jours (du 17 août après midi au 27 août au matin) et accueille 63 stagiaires (un peu moins qu'en 2014). Pour des raisons financières, nous avons dû réduire les dépenses et, notamment, n'imprimer qu'une version réduite du polycopié. La version complète sera néanmoins accessible, au format pdf, sur notre site. 

La plupart des quelque 450 candidats à la coupe Animath ont passé une épreuve éliminatoire, et 361 d'entre eux ont atteint l'épreuve finale, le 2 juin 2015 au matin. Avec des critères de sélection un peu modifiés par rapport à l'an dernier, nous avons retenu 63 stagiaires de 11 à 17 ans (âge moyen 15,7 ans), dont 9 filles (en prévision des Olympiades Européennes de Filles : même proportion que l'an passé), 9 jeunes nés en 2001 ou après (en prévision des Olympiades Balkaniques Junior : moins que l'an passé). Nous avons fixé des quotas par classe, de sorte que nous avions comme l'an passé 49\% d'élèves de première et moins d'élèves des Académies de Paris et Versailles (37\%). 21 animateurs, pour la plupart d'anciens stagiaires (dont certains très jeunes : âge moyen 25 ans), ont assuré les 168 heures de cours, les soirées, les tests, la muraille, le polycopié...

Le stage était structuré comme celui de l'an dernier : deux périodes de quatre jours (18 - 21 août et 22 - 25 aout), trois de cours / exercices, un test le matin du quatrième jour (8h30 à 12h30 pour le groupe D, 9h à 12h pour les autres), une excursion à Biot (musée Fernand Léger et village) le vendredi 21 après-midi et une après-midi libre le mardi 25. Enfin, le mercredi 26 aout, dernier jour du stage, était consacré à des cours non sanctionnés par un test, sur des sujets plus larges que la stricte préparation olympique. C'est la journée d' "ouverture". Les tests étaient corrigés le soir même, les soirées étaient libres les veilles de tests ainsi que le dernier jour (soirée spéciale sans contrôle d'heure de coucher). Même les autres soirs, l'heure de coucher n'était guère contrôlée. 

Le deuxième soir, mardi 18 août, Jean-Louis Tu, responsable des activités olympiques d'Animath et président du jury, a remis la coupe Animath aux 5 lauréats, avant une conférence de géométrie de Xavier Caruso. Joseph Najnudel, double médaillé d'or des Olympiades Internationales (1997 et 1998), a présenté une conférence sur les permutations le samedi 22 août. Les 19 et 23 août ont eu lieu la présentation par Thomas Budzinski des différentes Olympiades Internationales et la soirée ITYM - TFJM, coordonnée par Mathieu Barré. Nous n'étions pas seuls sur le campus, il y avait différents groupes pour un total de 250 personnes environ. Nous avions trois étages d'un bâtiment pour nous (contre quatre prévus initialement), presque tous les stagiaires et animateurs étaient en chambre individuelle avec sanitaire dans la chambre. Chaque participant avait sa carte magnétique de chambre et sa carte de restauration, plusieurs cartes de chambre ont dû être remagnétisées en cours de stage, et quelques cartes de restauration ont été perdues. L'horaire des repas était un peu plus large que d'habitude : 7 h à 9 h pour le petit déjeuner, 12 h à 13 h 30 pour le déjeuner, 19 h à 20 h pour le dîner.  

Le lundi 17 août, en raison d'un problème de photocopieuse, les livrets d'accueil (programme, plan du campus, liste des stagiaires et animateurs...) n'étaient pas prêts à l'arrivée des élèves, ils ont été distribués pendant la présentation du stage (17 h). Puis, les élèves ont répondu à un questionnaire d'évaluation en une heure et demie, afin d'être répartis dans quatre groupes de niveaux autant que possible homogènes (les groupes valaient pour l'ensemble du stage). Ce n'est que le mercredi 19 août que nous avons distribué les tee-shirts ainsi que les bics Animath. 



\bigskip

Quelques liens utiles pour poursuivre le travail réalisé pendant ce
stage:
%\refstepcounter{dummy}
% \label {liensutiles}
\begin{itemize}
\item Le site d'Animath : \href {http://www.animath.fr} {http://www.animath.fr}
\item Le site MathLinks: \href {http://www.mathlinks.ro} {http://www.mathlinks.ro}
\item Les polycopiés de stages olympiques précédents:  
\\
\href{http://www.animath.fr/spip.php?article260}{http://www.animath.fr/spip.php?article260}
\item Les cours de l'Olympiade Française de Mathématiques: 
\\
\href{http://www.animath.fr/spip.php?article255}{http://www.animath.fr/spip.php?article255}
\end{itemize}

\begin{center}
\resizebox{17cm}{!}{
\begin{tabular}{|l|c|c|c|c|c|}
\cline{3-6}
\multicolumn{2}{c|}{} & Groupe A  & Groupe B  & Groupe C & Groupe D\\
\hline
\small{Mardi 16/08} &  & \multicolumn{4}{c|}{Arriv\'ee, accueil des \'el\`eves, présentation du stage (17 h 00) et premi\`ere évaluation} \\
%& 20h30 -21h30 & \multicolumn{4}{c|}{Remise de la coupe Animath et conférence : Les médailles Fields (Martin)}\\
\hline
& 9h - 12h & stratégies de base & logique & arithmétique & arithmétique \\
& &  (Joon) &  (Antoine) & (Giancarlo)  & (M. Lequesne)\\
\cline{2-6}
\small{Mercredi 17/08} & 14h - 17h$+ \varepsilon$ & logique & stratégies de base & arithmétique & arithmétique\\
& & (Victor) & (Clara) & (Mehdi) & (François)\\
\cline{2-6}
& 20h30 -21h30 & \multicolumn{4}{c|}{Conférence : Qu'est-ce qu'une preuve ? (Matthieu Lequesne)}\\
\hline
& 9h - 12h & stratégies de base & stratégies de base & arithmétique &  arithmétique\\
&& (Eva) & (Lucie) & (Élie) & (Igor) \\
\cline{2-6}
\small{Jeudi 18/08}  & 14h - 17h$+ \varepsilon$ & jeu de go & stratégies de base & inégalité &  géométrie\\
&& & (Joon) & (Jean-Louis) & (Thomas) \\
\cline{2-6}
& 20h30 -21h30 & \multicolumn{4}{c|}{Conférence : les Olympiades de Mathématiques (Lucie)}\\
%\cline{2-5}
%& 17h$+ \epsilon$30-18h30 & \multicolumn{3}{c|}{Correction du Test}\\
\hline
& 9h - 12h & stratégies de base & algèbre & polynômes & géométrie\\
&& (Wassim) & (Félix) & (Igor) & (François) \\
\cline{2-6}
\small{Vendredi 19/08}  &14h - 17h$+ \varepsilon$& stratégies de base & algèbre & polynômes & géométrie\\
&& (Mehdi) & (Giancarlo) & (Thomas) & (Jean-Louis) \\
\cline{2-6}
& 20h30 - 21h 30 & \multicolumn{4}{c|}{Soirée libre} \\
\hline
& 9h - 12h & \multicolumn{4}{c|}{Test de mi-parcours} \\
\cline{2-6}
\small{Samedi 20/08} & Après-midi & \multicolumn{4}{c|}{Chasse au trésor} \\
\cline{2-6}
& 20h30 - 21h 30 & \multicolumn{4}{c|}{Correction du test} \\
\cline{2-6}
\hline
& 9h - 12h & géométrie & géométrie & géométrie & combinatoire\\
&& (Élie) & (Mathieu Barré) & (Cécile) & (Colin) \\
\cline{2-6}
\small{ Dimanche 21/08} & 14h - 17h$+ \varepsilon$ & arithmétique & géométrie & géométrie & combinatoire\\
&& (Antoine) & (Alexander)  & (François) & (Thomas) \\
\cline{2-6}
& 20h15 - 21h & \multicolumn{4}{c|}{Conférence : TFJM$^2$} \\
\hline
& 9h - 12h & géométrie & géométrie & géométrie & combinatoire \\
& &  (Wassim) &  (Élie) & (Mathieu Barré)  & (Vincent Bouis)\\
\cline{2-6}
\small{Lundi 22/08 } & 14h - 17h$+ \varepsilon$ & & arithmétique & géométrie & polynômes \\
& &  &  (Cécile) & (Alexander)  & (Florent)\\
\cline{2-6}
& 20h - 21h & \multicolumn{4}{c|}{Conférence : tresses et treillis (Vincent Jugé)} \\
\hline
& 9h - 12h & géométrie & arithmétique & éq. fonctionnelles & inégalités \\
& &  (Lucie) &  (Colin) & (Clara)  & (Vincent Jugé)\\
\cline{2-6}
\small{Mardi 23/08} & 14h - 17h$+ \varepsilon$ & arithmétique & arithmétique & éq. fonctionnelles & inégalités \\
& &  (Victor) &  (Florent) & (Félix)  & (Louise)\\
%\cline{2-6}
%& 20h - 23h30 & \multicolumn{4}{c|}{Anniversaire Pierre-Alexandre + Soirée libre} \\
\hline
& 9h - 12h & \multicolumn{4}{c|}{Test de fin de parcours} \\
\cline{2-6}
\small{Mercredi 24/08} & Après-midi & \multicolumn{4}{c|}{Après-midi libre  - activité proposées} \\
\cline{2-6}
& 20h30 - 21h 30 & \multicolumn{4}{c|}{Correction du test} \\
\cline{2-6}
\hline
& 9h - 12h & théorie des graphes & théorie des graphes & groupes & partitions entières \\
& &  (Romain) &  (Vincent Bouis) & (Vincent Jugé)  & (M. Piquerez)\\
\cline{2-6}
\small{Jeudi 25/08} & 14h - 17h$+ \varepsilon$ & géom. projective réelle & groupes & plateforme FUN & dim. fractale \\
& &  (Cécile) &  (Eva) & (Martin Andler)  & (Louise)\\
\cline{2-6}
& 20h - 23h30 $+ \varepsilon$ & \multicolumn{4}{c|}{Soirée/nuit libre} \\\hline
Vendredi 26/08 & Matinée & \multicolumn{4}{c|}{Petit déjeuner et départ} \\
\hline
\end{tabular}}
\end{center}
\vspace{\fill}

%\begin{center}
%\rput(0,-2pt){\psvectorian[scale=0.4]{68}}
%\end{center}
%\vspace{\fill}

