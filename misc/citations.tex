\noindent - Matthieu L "Le problème c'est pas qu'il y a un problème"

\noindent - "C'est un peu en mode fakir"

\noindent - Étienne M "L'arithmétique c'est plus simple quand c'est facile"

\noindent - Lucie W "Ça existe la notion d'angles pointus ?"

\noindent - Victor "Vous pouvez l'utiliser pour les brouillons… comme ce cours"

\noindent - Antoine : "Si tu es la reine d'Angleterre, je suis le premier ministre. Tu es la reine d'Angleterre ?"
\noindent - Olivine : "oui !"

\noindent - Matthieu L : "Un avion, c'est comme un ascenseur en un peu plus compliqué"

\noindent - Noémie : "François, vous avez combien d'oreilles ?"

\noindent - "Est-ce que François a une oreille coupée ?"

\noindent - Victor : "Clara ? C'est une des deux chinoises en bas ?"

\noindent - Antoine : "Dieu a créé le monde et la décomposition en nombres premiers"

\noindent Savinien : "Soient $a_1, a_2, \ldots, a_4$ des réels"

\noindent Thomas : "Il faut vérifier que ça marche avec n'importe quelle valeur de 2016"

\noindent Mathias : "$\sqrt{4} \equiv 3 [5]$" (convaincu de ce qu'il affirme)

\noindent Clara : "Je veux bien vivre toute ma vie avec Lucie mais pas avec un enfant"
\noindent Lucie : "Moi avec un enfant oui mais pas avec Clara"

\noindent Clément : "Moi je connais pas les chevaliers de la table ronde, à part D'Artagnan"

