\documentclass[12pt,francais]{article}
\PassOptionsToPackage{dvipsnames,svgnames}{xcolor}
%\usepackage[applemac]{inputenc}
\usepackage{amssymb,amsmath,amsthm,epsfig,mathrsfs}
\usepackage{fancyhdr}


\usepackage[utf8]{inputenc}
\usepackage[T1]{fontenc}
\usepackage[french]{babel}

\usepackage{palatino,graphicx}
\usepackage[top=1.5cm,left=2cm,right=2cm,bottom=2cm]{geometry}


\newcommand{\ligne}{\begin{center}\vrule depth 0pt height 0.4pt
		width 5cm\end{center}}

\newcommand{\etoile}{\begin{center}$\ast\ast\ast$\end{center}}

%\usepackage[object=vectorian]{pgfornament}
%\usetikzlibrary{shapes.geometric,calc}


%\usepackage{auto-pst-pdf}
\usepackage{tikz}
\usetikzlibrary{arrows}
%\usepackage{psvectorian}
%\usepackage{lettrine}
%\renewcommand{\LettrineFontHook}{\fontfamily{pzc}}

%       \renewcommand*{\psvectorianDefaultColor}{NavyBlue}%
%\tikzset{pgfornamentstyle/.style={draw=NavyBlue, fill = red}}

% --------------------------------------------------------------------------------------------------------------
% --------------------------------------------------------------------------------------------------------------
\theoremstyle{definition}
\newtheorem{thm}{Th\'eor\`eme}[section]
\newtheorem{prop}[thm]{Proposition}
\def \style {\displaystyle}
% --------------------------------------------------------------------------------------------------------------
% --------------------------------------------------------------------------------------------------------------

\def\ind{{\mathbbm{1}}_}
\def\al{\alpha}
\def\a{{\mathcal A}}
\def\vep{\varepsilon}
\newcommand{\R}{\mathbb{R}}
\newcommand{\K}{\mathbb{K}}
\newcommand{\Z}{\mathbb{Z}}
\renewcommand{\P}{\mathbb{P}}
\newcommand{\Q}{\mathbb{Q}}
\newcommand{\E}{\mathbb{E}}
\newcommand{\C}{\mathbb{C}}
\newcommand{\D}{\mathbb{D}}
\newcommand{\N}{\mathbb{N}}
\newcommand{\M}{\mathbb{M}}
\newcommand{\T}{\mathbb{T}}
\newcommand{\A}{\mathbb{A}}
\renewcommand{\Z}{\mathbb{Z}}
\renewcommand{\L}{\mathbb{L}}

\def\f{{\mathcal F}}
\def\sg{\sigma}
\def\noi{\noindent}
\def\b{{\mathcal B}}
\def\dl{\delta}
\def\vph{\varphi}
\def\lm{\lambda}
\def\build#1_#2^#3{\mathrel{
		\mathop{\kern 0pt#1}\limits_{#2}^{#3}}}

\def\la{\longrightarrow}

%\newtheorem{exo}{Exercice}

\makeatletter
\newcounter{exo}[subsection]
\newenvironment{exo}[1][]
{\edef\@exer{#1}%
	\ifx\@exer\empty{}\else{\setcounter{exo}{#1}\addtocounter{exo}{-1}}\fi%
	\refstepcounter{exo}%
	\noindent { \textcolor{NavyBlue}{\fontfamily{pzc}\selectfont \large {\LARGE\textsc{E}}xercice \theexo.}}%
	\hspace{0.5em}}
{\medskip}

\newcounter{sol}[subsection]
\newenvironment{sol}[1][]
{\edef\@exer{#1}%
	\ifx\@exer\empty{}\else{\setcounter{sol}{#1}\addtocounter{sol}{-1}}\fi%
	\refstepcounter{sol}%
	\noindent {\it \underline{Solution de l'exercice \thesol}%
		\hspace{0.5em}}}
{\medskip}
\makeatother


\setlength{\parindent}{0pt}

%\pagestyle{empty}

\newenvironment{cor}[1]
{\medskip\noindent \textcolor{red}{\underline{\emph{Solution.}}}
	
	{  #1}}
{%\vspace {-0.6cm}
	\begin{center}
		\rput(0,-8pt){\psvectorian[scale=0.3]{49}}
	\end{center}
	\vspace {0.6cm}
	
	\pagebreak}

\newenvironment{corrige2}[1]
{\medskip\noindent \textcolor{red}{\underline{\emph{Solution.}}}
	
	{  #1}}
{}


\title{Equations}
\date{}
\author{}

\begin{document}
	\begin{sol}
		\definecolor{uuuuuu}{rgb}{0.26666666666666666,0.26666666666666666,0.26666666666666666}
		\definecolor{qqqqff}{rgb}{0.,0.,1.}
		\begin{tikzpicture}[line cap=round,line join=round,>=triangle 45,x=0.5cm,y=0.5cm]
		\clip(-4.8866410764292,-5.572904036609516) rectangle (24.92335892357077,7.055095963390474);
		\draw (3.38,4.66)-- (4.6,-3.1);
		\draw (4.6,-3.1)-- (13.6,-3.2);
		\draw (3.38,4.66)-- (13.6,-3.2);
		\draw(6.6073797978803945,-0.7786041073342618) circle (1.171777726594992cm);
		\draw (3.8617737962416823,1.595602738659465)-- (7.415842784909827,1.556113083229819);
		\draw (1.0600044164045799,-3.0606667157378284)-- (5.638808290575755,1.575857910944642);
		\draw (1.0600044164045799,-3.0606667157378284)-- (6.581341825154507,-3.1220149091683833);
		\draw (1.0600044164045799,-3.0606667157378284)-- (8.036095081092272,1.0790893016257088);
		\begin{scriptsize}
		\draw [color=qqqqff] (3.38,4.66)-- ++(-2.5pt,-2.5pt) -- ++(5.0pt,5.0pt) ++(-5.0pt,0) -- ++(5.0pt,-5.0pt);
		\draw[color=qqqqff] (3.5393589235707914,5.053095963390476) node {$A$};
		\draw [color=qqqqff] (4.6,-3.1)-- ++(-2.5pt,-2.5pt) -- ++(5.0pt,5.0pt) ++(-5.0pt,0) -- ++(5.0pt,-5.0pt);
		\draw[color=qqqqff] (4.199358923570791,-2.712904036609518) node {$B$};
		\draw [color=qqqqff] (13.6,-3.2)-- ++(-2.5pt,-2.5pt) -- ++(5.0pt,5.0pt) ++(-5.0pt,0) -- ++(5.0pt,-5.0pt);
		\draw[color=qqqqff] (13.747358923570783,-2.800904036609518) node {$C$};
		\draw [color=uuuuuu] (8.036095081092272,1.0790893016257088)-- ++(-2.5pt,-2.5pt) -- ++(5.0pt,5.0pt) ++(-5.0pt,0) -- ++(5.0pt,-5.0pt);
		\draw[color=uuuuuu] (8.445358923570787,1.3350959633904789) node {$E$};
		\draw [color=uuuuuu] (4.292261123652807,-1.1425789504473618)-- ++(-2.5pt,-2.5pt) -- ++(5.0pt,5.0pt) ++(-5.0pt,0) -- ++(5.0pt,-5.0pt);
		\draw[color=uuuuuu] (3.913358923570791,-0.9309040366095195) node {$F$};
		\draw [color=uuuuuu] (6.581341825154507,-3.1220149091683833)-- ++(-2.5pt,-2.5pt) -- ++(5.0pt,5.0pt) ++(-5.0pt,0) -- ++(5.0pt,-5.0pt);
		\draw[color=uuuuuu] (6.729358923570788,-2.734904036609518) node {$D$};
		\draw [color=uuuuuu] (1.0600044164045799,-3.0606667157378284)-- ++(-2.5pt,-2.5pt) -- ++(5.0pt,5.0pt) ++(-5.0pt,0) -- ++(5.0pt,-5.0pt);
		\draw[color=uuuuuu] (0.7233589235707943,-2.690904036609518) node {$T$};
		\draw [color=uuuuuu] (6.6334177706062825,1.5648066944998584)-- ++(-2.5pt,-2.5pt) -- ++(5.0pt,5.0pt) ++(-5.0pt,0) -- ++(5.0pt,-5.0pt);
		\draw [color=uuuuuu] (3.8617737962416823,1.595602738659465)-- ++(-2.5pt,-2.5pt) -- ++(5.0pt,5.0pt) ++(-5.0pt,0) -- ++(5.0pt,-5.0pt);
		\draw[color=uuuuuu] (4.023358923570791,1.9950959633904781) node {$H$};
		\draw [color=uuuuuu] (7.415842784909827,1.556113083229819)-- ++(-2.5pt,-2.5pt) -- ++(5.0pt,5.0pt) ++(-5.0pt,0) -- ++(5.0pt,-5.0pt);
		\draw[color=uuuuuu] (7.719358923570788,2.1930959633904776) node {$G$};
		\draw [color=uuuuuu] (5.638808290575755,1.575857910944642)-- ++(-2.5pt,-2.5pt) -- ++(5.0pt,5.0pt) ++(-5.0pt,0) -- ++(5.0pt,-5.0pt);
		\draw[color=uuuuuu] (5.783358923570789,1.9730959633904783) node {$M$};
		\draw [color=uuuuuu] (4.939888935469334,0.8681279248895978)-- ++(-2.5pt,-2.5pt) -- ++(5.0pt,5.0pt) ++(-5.0pt,0) -- ++(5.0pt,-5.0pt);
		\draw[color=uuuuuu] (4.595358923570791,1.3790959633904787) node {$T_1$};
		\end{scriptsize}
		\end{tikzpicture}
		
		(Résolu par Baptiste Serraille)
		
		La solution utilise deux résultats de géométrie projective avancés : 
		
		i) Pour un point et un cercle donnés, les deux points de contact avec les tangentes par ce point et deux points sur une sécantes sont en division harmonique
		
		ii) Le birapport de quatre points sur un cercle est égal au birapport des tangentes en ces points
		
		iii) Soient $A,B$ des points et $M$ le milieu de $[AB]$. Alors $A,B,M,\infty$ sont harmoniques. Réciproquement, si $A,B,M,\infty$ sont harmoniques, alors $M$ le milieu de $[AB]$.
		
		Début de la solution :
		
		Soit $T_1$ le point de contact de la tangente (autre que $(TC)$) au cercle inscrit. Et soit $M_1$ le point d'intersection de $TT_1$ avec $(GH)$. Montrons que $M_1$ est le mileu de $(GH)$.
				
		D'après i) $T_1 F E D$ sont en division harmonique
		
		D'après ii) $$-1 =b_{T_1,D,F,E}=b_{T_1M_1,DB,FH,EG}=b_{M_1,\infty,H,G}$$
		
		Donc, d'après iii), $M_1$ est bien le milieu de $[HG]$.
				
		
	\end{sol}
	
\end{document}
